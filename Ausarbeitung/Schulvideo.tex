\section{Schulklassenvideo}
\label{Schulvideo}
Aus Datenschutzgründen kann kein originales Bild veröffentlicht werden, daher wurde ein Bild anderes verwendet. Die Bildaufteilung, Kameraausrichtung und Auflösung ist ähnlich, um die Problematik zu visualisieren, wenn auf solchen Daten gearbeitet wird. Zur Verfügung steht nur ein einfaches Video der Schulklasse, ohne Ground-Truth Daten. Außerdem wurde es mit einer unbekannten Videokamera aufgezeichnet, daher sind nur die Parameter des Filmes $(640 \times 480$ Pixel mit $25Fps)$ bekannt.\\
Die Hauptproblematik ist die Bildauflösung, sie ist sehr gering und die Gesichter sind nur durch entsprechend wenige Pixel dargestellt. Außerdem ist die Distanz zwischen den Schülern und Kamera sehr unterschiedlich wodurch verscheiden Größen entstehen.\\
\begin{figure}
	\centering
	\includegraphics[width=0.8\linewidth]{img/Schulklasse}
	\caption{Eine Screenshot des YouTube-Videos \glqq Maxi Beister als Herr Müller überrascht eine Schulklasse\grqq \cite{Schulklasse_Video}}
	\label{fig:schulklasse}
\end{figure}
