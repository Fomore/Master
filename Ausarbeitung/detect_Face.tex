\section{Gesichtserkennung}
\label{detection_Gesicht}
Da nur eine einzige fest montierte Kamera ohne Zoom eingesetzt wird, muss diese eine entsprechend hohe Auflösung besitzen damit alle Personen zu erkennen sind. Allerdings machen die eigentlichen Bereiche der Gesichter nur einen sehr geringen Anteil des gesamten Bildes aus und diese relevanten Bildausschnitte müssen für die spätere Anwendung noch aufbereitet werden, siehe \autoref{skalierung}\\
Für die automatische Detektion wird MTCNN-Face eingesetzt, da dieses Verfahren im Vorabtests auf Probebildern einen sehr guten Eindruck gemacht hat und die meisten Gesichtern mit verschieden Größen und Blickrichtungen finden konnte. Sogar recht kleine mit $20\times 20$ Pixeln soll laut Beschreibung des Verfahrens \autoref{MTCNN} möglich sein. Bei diesem Schritt müssen alle Gesichert gefunden werden, auf denen die Berechnung stattfinden soll. Dabei muss das gesamte Gesicht in der Box sein, weitere Besonderheiten gibt es nicht, da OpenFace einen eigenen Facedetector besitzt.\\
Die von den beiden Methoden (OpenFace und MTCNN-Face) ausgegebenen Boxen sind allerdings in ihren Ausmaßen nicht identisch. Je nach verwendetem Trainingsdatensatz und darin enthaltener Annotation werden z.B. Kinn und Haaransatz noch als Gesichtsbereich oder schon als außerhalb betrachtet. Da die folgende Verarbeitung eine OpenFace-skalierte Box erwartet, hat sich eine Vergrößerung der Box um $30\%$ als sinnvoll erwiesen bei Verwendung des MTCNN-Face Detektors.\\
Ebenfalls in diesem Schritt werden die einzelnen Boxen den Personen zugeordnet, damit im späteren Verlauf das korrekte CLNF für die Person verwendet werden kann. Für die Zuordnung reicht es meist aus, jene Box zu wählen die am ehesten den selben Bereich wie im vorigen Frame einnimmt. Dabei wird einfach für jede Box im neuen Frame die Box Im vorigen Frame gesucht die den selben Bildausschnitt repräsentiert. Dies ist ausreichend, da die Gesichter sich meist weder groß Bewegen noch sich die einzelnen Boxen der anderen überlappen.\\
Damit sicher auf allen Gesichter gerechnet werden kann, ist eine semiautomatische Korrektur erforderlich um Falsch-Detektionen zu entfernen und fehlende Boxen der Gesichtern ergänzen zu können.\\
Die gefundenen 5 Landmarks von MTCNN-Face Detection sind für die nachfolgende Berechnung nicht relevant, da sie gerade bei kleinen Gesichtern zu ungenau sind. Daher kann dieser Bereich auch von anderen Verfahren übernommen werden, da es sich hierbei nur um ein Vorverarbeitungsschritt handelt und zur Beschleunigung sowie Stabilität des späteren Berechnung beitragen soll.
