\section{Detektion der Gesichter}
\label{detection_Gesicht}
Da nur eine einzige fest montierte Kamera ohne Zoom eingesetzt wird, muss sie eine entsprechend hohe Auflösung besitzen damit alle Personen zu erkenne sind. Allerdings machen die eigentlichen Bereiche der Gesichter nur einen sehr geringen Anteil aus und diese müssen noch Nachbearbeitet werden. Siehe \ref{skalierung}\\
Für die automatische Detektion wird Face-MTCNN  \ref{MTCNN} eingesetzt, da dieses Verfahren die meisten Gesichtern mit verscheiden Größen im selben Bild findet, sogar recht kleine mit $20\times 20$ Pixeln. Bei diesem Schritt müssen alle Gesichert gefunden werden, auf denen die Berechnung stattfinden soll. Dabei muss das gesamte Gesicht in der Box sein, ansonsten muss es nicht sehr exakt sein, da OpenFace einen eigenen Facedetector besitzt. Wird MTCNN-Face dedector eingesetzt hat sich eine Vergrößerung der Box um $30\%$ als sinnvoll erwiesen, damit sichergestellt wird, dass alle Merkmale wie Nasenspitzen, Kinn, Augenbrauen usw. sicher im Bildausschnitt enthalten sind.\\
Ebenfalls in diesem Schritt werden die einzelnen Boxenden Personen zugeordnet, damit im späteren Verlauf das korrekte CNN verwendet wird. Für die Zuordnung reicht meist einen einfache Übereinstimmung der aktuellen Box zum vorigen Frame, da die Gesichter sich meist weder groß Bewegen noch sich die Boxen überlappen.\\
Damit auf allen Gesichter gerechnet werden kann, Ist eine Semiautomatische Korrektur erforderlich damit Falsch-Detectionen entfernt und fehlende Boxen ergänzt werden können. Alle nicht gefundenen Gesichtern können manuelle gesetzt oder zwischen dem letzten und nächsten Frame interpoliert werden.\\
Die gefundenen 5 Landmarks sind für die nachfolgende Berechnung nicht relevant, da sie gerade bei kleinen Gesichtern zu ungenau sind.