\section{Bestimmung einer Position auf der die Aufmerksamkeit liegt}
Ist bekannt wohin die Personen alle Blicken, kann dies aus ihrer Blickrichtung bestimmt werden.
\subsection{Schnittpunkt berechnen}
\label{calc_Center}
Verwende Blickrichtung mit Linie $L_i = s \cdot n_i+ p_i$ mit $s\in \mathbb{R}$ und $n_i,p_i \in \mathbb{R}^3$
\begin{align*}
c&=(\sum_{i} I -n_in_i^T)^{-1}
(\sum_{i} (I -n_in_i^T)p_i)
\end{align*}
\subsection{Mittelwert}
Ermittle aus den bestimmten Winkel die Blickrichtung mit $O= R_{a,b,c}\cdot (0,0,-1)^T$ und deren Durchschnitt $O_{avg}$ und die Durchschnittliche Position $P_{avg}$ der Personen. Bei der Bestimmung der Tiefe muss nun geschätzt werden, Wenn das Ziel die Tafel ist und die Kamera an der Tafel platziert wurde, kann die Tiefe im Bild verwendet werden um somit die Position der Tafel zu ermitteln.\\
Dies muss angewandt werden, wenn die Blickrichtungen zu sehr parallel verlaufen oder zu verrauscht sind um ein sinnvolles Ergebnis mit den Schnittpunkt \autoref{calc_Center} zu erhalten.