\section{Hardware}
\label{hardware}
Als Messinstrument für die Versuche wurden verschiedenen Farbkameras eingesetzt.\\
Das Videomaterial der Schulklasse wurde mit einer unbekannten Videokamera aufgezeichnet, daher sind nur die Parameter des Filmes $(640 \times 480$ Pixel mit $25Fps)$ bekannt.\\
Für die Messungen im Versuch wurde die Explorer 4K Action Camera verwendet, sie besitzt ein $170^\circ$ Weitwinkel-Linse mit großer Schärfentiefe. Mit ihrer 2.7K Einstellung wird ein $2688 \times 1520$ Video mit 30FPS aufgezeichnet. Leider ist das Bild stark von Pixelrauschen betroffen.\\
Außerdem die Logitech c920 HD Pro Webcam, diese liefert ein $15Fps$ Video mit einer Auflösung von $1600\times 896$ Pixel. Die Kamera besitzt einen horizontalen Blickwinkel von etwa $70^\circ$.
%\subsection{Auswirkung von Pixelrauschen}
%Durch Aufnahme eines Schwarzbildes der Actioncam zeigt sich, dass das Pixelrauschen recht hoch ist, siehe \autoref{img_noishight}. Das Rauschen hat keine Normalverteilung, sondern es besteht aus kleinen Bereiche, die den selben fehlerhaften Farbwert besitzen.
%\begin{figure}
%	\centering
%	\fbox{\includegraphics[width=1\linewidth]{img/NoisHight}}
%	\caption{Aufnahme eines Schwarz-Bildes $(2688\times 1520)$ der Actioncam um den Faktor 7 verstärkt und invertiert.}
%	\label{img_noishight}
%\end{figure}
