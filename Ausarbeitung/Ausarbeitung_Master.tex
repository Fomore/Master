\documentclass{scrreprt}
\usepackage[utf8]{inputenc}
\usepackage[english,ngerman]{babel}
\usepackage{lscape}
\usepackage{eurosym}
\usepackage{amsmath}
\usepackage{graphicx} 
\usepackage[left=2cm,right=2cm,top=2cm,bottom=2cm,includeheadfoot]{geometry} 
\usepackage{enumerate}
\usepackage{hyperref}
\usepackage{amssymb}
\usepackage{caption}

\usepackage{scrpage2}\pagestyle{scrheadings}
\ohead{\today}
\begin{document}
\chapter{Einführung}
\begin{abstract}
\section*{Zusammenfassung}
Aufmerksamkeit ist eine Grundvoraussetzungen für erfolgreiches Lernen in der Schule. Eine objektive Quantifizierung der Aufmerksamkeit eines Schülers oder einer ganzen Klasse könnte dabei helfen Lern- und Lehrprozesse besser zu verstehen und zu optimieren.\\
Aufmerksamkeit zu messen ist allerdings nur indirekt möglich, z.B. per Eye-Tracking und Beobachtung des Blickverhaltens. Mit dem momentanen Stand der Technik muss für jeden Schüler ein dezidiertes Gerät einsetzt werden. Dieser Prozess ist extrem teuer, schränkt die Probanden in ihrer Bewegungsfreiheit ein und ist in der Auswertung aufwendig.\\
Diese Arbeit untersucht die technische Machbarkeit eines effizienten Aufbaus zur Aufmerksamkeitsanalyse einer Menschengruppe. Dazu werden die Grenzen und erreichbaren Genauigkeiten einer Gesichtsanalyse basierend auf Bildmaterial einer einzelnen Kamera ausgelotet. Durch diesen Aufbau ergibt sich die Problematik dass Personen sich in sehr unterschiedlicher Distanz zur Kamera aufhalten können und, daraus resultierend, unterschiedliche Abbildungseigenschaften, wie z.B. die Anzahl der Pixel die das Gesicht einer Person im Bild darstellen.\\
Um alle Probanden in einem Kamerabild hinsichtlich Aufmerksamkeitszuwendung bewerten zu können, werden zuerst die einzelnen Gesichter im Bild detektiert und eindeutig einem Probanden zugeordnet. Im Folgeschritt wird der so gewonnene Bildausschnitt aufbereitet. Eine nachfolgende Analyse des abgebildeten Gesichts erlaubt die Bestimmung von Position und Orientierung im Raum. Die Augenregion ist für die gerichtete Aufmerksamkeit besonders aussagekräftig und wird deshalb gesondert behandelt, um genauere Ergebnisse bei der Bestimmung der Blickrichtung zu erhalten.\\
Diese Arbeit stellt mehrere Referenzmessungen vor, bei denen mithilfe bekannter Positionen und Blickwinkeln im Raum die Güte der algorithmischen Bestimmung von gerichteter Aufmerksamkeit quantifiziert werden kann. Alle dabei angewandten Szenarien beziehen sich auf die Dimensionen und Anwendungsszenarien in einem typischen Klassenzimmer.\\
Die bestimmten Genauigkeitswerte haben ergeben, dass die Bestimmung der Kopforientierung mit durchschnittlich $5^\circ$ Grad, die der Blickrichtung mit $10^\circ$ Grad möglich ist.\\
Für die Analyse kann meist nur auf der Kopforientierung gearbeitet werden, da für die Bestimmung der Blickrichtung in den - mit der Entfernung von der Kamera schnell kleiner werdenden - abgebildeten Kopfregionen im Bild zu wenige Informationen vorhanden sind. Abgeleitet aus den Ergebnissen der einzelnen Verfahren sollte eine Auswertung der Augen bis zu einer Distanz von maximal $4m$ möglich sein. Dieser Wert konnte im Test unter Realbedingungen allerdings nicht erreicht werden.
Die verwendeten Verfahren zur Gesichtsanalyse (Landmarkenbestimmung und Positionserrechnung) sind außerdem auf einen Winkel von $45^\circ$ relativ zur Kamera beschränkt.\\
Die in dieser Arbeit gewonnenen Erkenntnisse bedeuten, dass für eine aussagekräftige Auswertung entweder nur Kopforientierungen herangezogen werden können oder hochauflösendere Aufnahmen angefertigt werden müssen, wobei eine Kamera einen Bereich von 5 Metern Tiefe und $45^\circ$ abdecken kann.
\end{abstract}

\label{intension}
Die Grundlage für erfolgreiches Lernen ist die Aufmerksamkeit der Schüler. Sie ist ein wichtiger Indikator für die Qualität des Unterrichtes. Das Verhalten eines Schülers kann stark vereinfacht eingeteilt werden in \textit{ON-Task} (aufmerksam bei der Sache) und \textit{OFF-Task} (unaufmerksam). Allerdings ist das Erfassen einer Aufgabe zugewandten Aufmerksamkeit technisch schwierig, da es sich um einen kognitiven Prozess handelt der nur indirekt beobachtet werden kann. Entsprechend existieren verschiedene Erfassungsmethoden; Ein Vorschlag von Ehrhardt, Findeisen, Marinello und Reinhartz-Wenzel (1981) umfasst beispielsweise die Beurteilung von Blickrichtung, Körperhaltung und Tätigkeit.\\
Zur Erfassung werden z.B. Fragebögen eingesetzt, die Schüler und/oder Lehrer selbst ausfüllen oder ein Beobachter bewertet die Aufmerksamkeit einzelner Schüler anhand festgelegter Kriterien.\\
Die Zuwendung von Aufmerksamkeit kann indirekt z.B. durch eine Blickzuwendung gemessen werden (auch wenn nicht mit jeder Blickzuwendung zwangsweise eine Aufmerksamkeitszuwendung einhergehen muss, ist es eine hinreichende Annäherung). Während eine Blickrichtungsbestimmung erstrebenswert wäre, kann auch bereits die Bestimmung der Kopforientierung als Richtungsindikator verwendet werden.\\
Im Rahmen dieser Arbeit soll untersucht werden, wie weit es technisch möglich ist Filmmaterial einer einer Unterrichtsstunde im Bezug auf Blickrichtungen auszuwerten und mit welchen Einschränkungen und Genauigkeiten zu rechnen sind ist. Daraus lassen sich Anhaltspunkte sowohl über die Auswertbarkeit existierender Daten und als auch über einen optimalen Versuchsaufbau ableiten.\\
Gängige Methoden zur Bestimmung der Blickrichtung sind für diesen Zweck nur eingeschränkt geeignet, wie beispielsweise Eye-Tracking Brillen. Zum einen ist die Anschaffung einer großen Stückzahl dieser Geräte teuer und wurde bisher nur in wenigen speziell eingerichteten Laboratorien durchgeführt wie z.B. dem TüDiLab \cite{TueDiLab}. Zum anderen sind die Geräte entweder intrusiv und haben damit ein Ablenkungspotential (Brillen) oder schränken den Aktionsradius ein (Remote Tracker mit ihrer Head-Box von üblicherweise weniger als 30x30 cm).\\
Die hier bestimmten Grenzen der momentan zur Verfügung stehenden Algorithmen ergeben Anhaltspunkte, wie das optimale (also das voll abdeckende und trotzdem einfachste) Setup (Anzahl und Position der Kameras und deren Auflösung) für ein größeres Experiment aussehen muss, um die Aufmerksamkeit einer ganzen Klasse zu erfassen. Wäre man in der Lage, solch eine qualitativ hochwertige Auswertung mit nur wenigen Kamera durchführen zu können, so ist der Aufbau und die Aufnahmen der Daten auch für technische Laien durchführbar.\\
Werden viele Kameras verwendet ergeben sich verscheiden Problematiken: Alle Kameras müssen Synchronisiert werden im Bezug auf die Zeit und der Ausrichtung zueinander um die Ergebnisse basierend auf den einzelnen Videos miteinander abgleichen zu können. Diese Synchronisation ist bei wenigen Kameras deutlich einfacher. Außerdem müssen alle Aufzeichnungen in Echtzeit stattfinden, womit die Limitierung der Bandbreite bei Vernetzungen ebenfalls berücksichtigt werden muss und somit die Anzahl begrenzen kann.\\
Die Interpretation der Ergebnisse dieser Arbeit orientiert sich an Originalaufnahmen eines Englischunterrichtes, diese zeigen die gesamte Klasse aus Sichtrichtung der Tafel.\\
Da für diese Aufnahmen keine Ground-Truth Daten (exakte Position der Schüler/Kamera usw.) bekannt sind, wird eine Reihe von Versuchen durchgeführt, um die einzelne Aspekte und Problemstellungen der Datenanalyse dieser Videos genauer zu untersuchen.\\
In den ersten Versuchen wurden verschiedene Aufnahmen verwendet, um die Auswirkung von Position und Zielpunkt auf die Auswertung zu testen. Für den anschließenden Versuch wurde ein bewegliches Blickziel erstellt, um eine kontinuierliche Messwerterfassung zu testen.
\section{Problemstellung}
\label{Problemstellung}
Im Rahmen dieser Arbeit sollen die Grenzen aufgezeigt werden, wie weit es technisch möglich ist Filmmaterial einer einzigen Kamera im Bezug auf Blickrichtungen bzw. Ausrichtung des Gesichts Auszuwerten und mit welchen Einschränkungen und Genauigkeiten zu rechnen sind, wenn im Bild eine gesamte Klasse dargestellt ist.\\
Eine automatisierte Auswertung der Blickrichtung wäre erstrebenswert, da dies einer der wichtigsten Indikatoren für gerichtete Aufmerksamkeit ist. Ist dies nicht möglich, reicht eine Bestimmung der Kopforientierung aus, da diese in etwa der Blickrichtung entspricht.\\
Die Messung soll den Unterricht möglichst wenig beeinträchtigen, wodurch hierfür üblicherweise verwendete Geräte, wie z.B. Eye-Tracking Brillen, nicht verwendet werden können. Zum einen ist die Anschaffung einer großen Stückzahl dieser Geräte teuer und wurde bisher nur in wenigen speziell eingerichteten Laboratorien durchgeführt wie z.B. TüDiLab \cite{TueDiLab}. Zum anderen sind die Geräte entweder Ablenkend (Brillen) oder schränken den Aktionsradius ein (Remote Tracker).\\
Die hier bestimmten Grenzen ergeben Anhaltspunkte, wie das Setup (Anzahl und Position der Kameras und deren Auflösung) für ein größeres Experiment aussehen muss, um die Aufmerksamkeit einer ganzen Klasse zu erfassen. Wären man in der Lage, solch eine qualitativ hochwertige Auswertung mit nur wenigen Kamera durchführen zu können, so ist der Aufbau und die Aufnahmen der Daten auch für technische Laien durchführbar.\\
Eine Möglichkeit für das automatische Erfassen der Aufmerksamkeit wird in \glqq Real time detection of driver attention\grqq\cite{driverAttention} vorgestellt. Bei diesem Verfahren ist eine Kamera direkt von vorn auf den Fahrer gerichtet und anhand der Kopf und Augenposition bewertet, ob dieser aktiv auf den Verkehr achtet.\\
Ein weiteres dazu passendes Verfahren wird in \glqq AggreGaze\grqq \cite{AggreGaze} präsentiert, dabei wird eine einzige Kamera fest auf einem Bildschirm montiere, um die Blickrichtung der Passanten auf den Bildschirm zu bestimmen, dieses Verfahren arbeitete allerdings nur auf einem recht begrenzen Bereich in dem sich die Probanden aufhalten dürfen und das Ziel der Blicke ist sehr nahe an der Kamera.\\
Um die Machbarkeit der Analyse zu untersuchen, wurden verschiedene Videoaufnahmen ausgewertet. Unter anderem zwei Originalaufnahmen eines Englischunterrichtes, diese zeigen die gesamte Klasse aus Richtung der Tafel und liefern Eindrücke über die verschiedenen Probleme. Allerdings besitzen sie nur eine sehr geringe Auflösung ($640\times 480$ Pixel).\\
Für die Vorversuche wurde eine Actioncam verwendet um erste Eindrücke, bezüglich der Auswirkung von Position und Zielpunkt auf die Auswertung zu erhalten.\\
Um mehr Messwerte für unterschiedlichen Zielpunkte zu erhalten wurde ein weiterer Video-Datensatz mit der Logitech-Webcam erstellt, bei der die Probanden ein bewegtes Ziel beobachten sollten. Damit besser bewertet werden kann, wie mit dem Verfahren das Ziel der Aufmerksamkeit bestimmt werden kann.\\
Außerdem soll die Anwendung auch auf schon vorhanden Aufnahmen des Englischunterrichtes arbeiten.

\chapter{Grundlagen}
\section{Hardware}
\label{hardware}
Als Messinstrument für die Versuche wurden verschiedenen Farbkameras eingesetzt.\\
Das Videomaterial der Schulklasse wurde mit einer unbekannten Videokamera aufgezeichnet, daher sind nur die Parameter des Filmes $(640 \times 480$ Pixel mit $25Fps)$ bekannt.\\
Für die Messungen im Versuch wurde die Explorer 4K Action Camera verwendet, sie besitzt ein $170^\circ$ Weitwinkel-Linse mit großer Schärfentiefe. Mit ihrer 2.7K Einstellung wird ein $2688 \times 1520$ Video mit 30FPS aufgezeichnet.\\
Außerdem die Logitech c920 HD Pro Webcam, diese liefert ein $15Fps$ Video mit einer Auflösung von $1600\times 896$ Pixel. Die Kamera besitzt einen horizontalen Blickwinkel von etwa $70^\circ$.
\section{Software}
Für die Umsetzung werden folgende Software-Elemente aus fremder Quelle eingesetzt.
\subsection{ElSe}
Ellipse Selection for Robust Pupil Detection (ElSe), ein Algorithmus zur Bestimmung der Pupille in einem hochauflösenden Bild des Auges. Der Ursprüngliche ElSe-Algorithmus ist für Graubilder mit Infrarotbeleuchtung ausgelegt und wurde für diese Anwendung angepasst um Farbbilder verarbeiten zu können.\\
\cite{ElSe}
\subsection{MTCNN Face Detection}
Multi-task Cascaded Convolutional Networks ist ein Algorithmus zur Detektion von Gesichtern und Bestimmung von 5 Gesichts-Landmarks in Farbbilder. Dabei werden drei CNN auf eine Bildpyramide angewendet um so zuverlässig Gesichter verschiedenster Größe im Bild zu erkennen.\\
\cite{MTCCN}
\subsection{OpenCV}
Open Source Computer Vision, ist eine C/C++ Bibliothek von Algorithmen zur Bildverarbeitung in Echtzeit, veröffentlicht unter der BSD Lizenz (Berkeley
Software Distribution)\\
\cite{wiki_Wha_is_OPenCV}\cite{OpenCv_What_Is}
\subsection{OpenFace}
Ein Open-Source Echtzeitverfahren auf Basis von CLNF zur Bestimmung und Analyse von Gesichtsmerkmalen in Grau-Bildern und Videos. Dabei werden 68 signifikante Punkte im Gesicht bestimmt und auf Basis jener Position und Orientierung ermittelt.\\
\cite{OpenFace}

\subsection{Gesetzte Ziele}
Da die Anwendung auf Aufnehmen eines Unterrichtes arbeiten soll, ergeben sich folgende Bedingungen:
\begin{itemize}
\item Normale Brillen, Kontaktlinsen und ähnliches sind erlaubt.
\item Bewegung/Sprechen usw. der Schüler ist erlaubt.
\item Es soll gleichzeitig auf Distanzen zwischen $2.5 - 8m$ zur Kamera funktionieren, ohne das sich diese Bewegt oder gezoomt werden kann.
\item Möglichst alle Blickrichtungen der Schüler sollen möglichst genau erfasst werden.
\end{itemize}
Ein deutsches Klassenzimmer hat $55-65m^2$, da noch Abstand zur Tafel usw. beachtet werden muss ergibt sich, wenn sich die Kamera an der Tafel befindet, einen Abstand zu den Schülern von $2.5 - 8m$ zur Kamera auf einer Breiten von $6m$. Somit muss der Linsenwinkel mindestens $100^\circ$ betragen.

\subsection{Randbedingungen}
Des weiteren werden folgende Annahmen gemacht:
\begin{itemize}
\item Die Szene ist Innerhalb eines Gebäudes stattfinden, mit ausreichend gleichmäßiger Beleuchtung.
\item Die Überführung zwischen Welt- und Kamerakoordinatensystem bekannt.
\item Die Kamera befindet sich vor der Klasse, so das die Blickrichtung nach vorn von den Schülern möglichst zur Kamera ist.
\item Die Gesichter der der Schüler sind komplett sichtbar.
\end{itemize}
Natürlich sind auch alle inneren Parameter der Kamera bekannt.
\section{Gesichtserkennung}
Die Gesichtserkennung ist Teil der Bildverarbeitung und wird ständig weiterentwickelt. Darunter fallen neben der Detektion des Gesichtes auch seine Analyse wie Orientierung oder das erkennen von Mimik und Übereinstimmungen. 
\subsection{Künstliches neuronales Netz}
Ein künstliches neuronales Netz besteht aus miteinander verknüpften künstlichen Neuronen. Jedes Neuron erhält Eingangswerte, diese erhalten eine individuelle Gewichtung, mittels einer Übertragungsfunktion zusammengefasst und durch eine Schwellenwertfunktion das Ergebnis bestimmt.\\
Um die Parameter der Neuronen zu bestimmen, werden sie zufällig initialisiert un dann so angepasst,dass sie zu einer gegebenen Eingabe das Gewünschte Ergebnis anzeigt und der Fehler über dem gesamten Trainingsdatensatzes minimal ist.
\subsection{Convolutional Neural Network (CNN)}
Diese ist eine Weiterentwicklung der neuronalen Netzen und werden zur Klassifizierung verwendet unter anderem Im Bereich Bild- und Spracherkennung. Dies wird durch eine gewichtete Faltung erreicht und sind  state of the art bei vielen Anwendungen.\\
So wird die Information aus den umliegenden Punkten eines Bereiches zusammengefasst und komprimiert an die nächste Schicht weitergegeben, um in der untersten Schicht alle vorhanden Informationen zusammenzuführen. 
Der Faltungskern kann ja nach Anwendung beliebig gestaltet sein, so ist eine Glättung durch einen Gauß-Kernel oder Kantendetektion durch einen Kirsch-Operator möglich.\\
Ein CNN kann in zwei Bereiche aufgeteilt werden, der Feature Extraktion in welcher durch verschiedene Kernel und Komprimierung die Eingabeinformationen zur Klassifizierung, dem zweiten Bereich, aufbereitet.
Gelernt werden können die Kernel an sich und die jeweiligen Bewertungen.
\subsection{Constrained Local Model (CLM)}
Ist ein Verfahren um mehrere Punkte eines Objektes zu lokalisieren. Dabei wird eine Wahrscheinlichkeitskarte für jeden einzelnen erstellt, wo er sich aufhalten kann auf Basis eines Trainingsdatensatzes. Nun wird versucht für das Bild, auf welchem gerechnet werden soll, für jeden Punkt den maximalen Wert zu erreichen zwischen passendem Farbverlauf und Wahrscheinlichkeit.\\
Dieser Art der Bestimmung von Punkten mit Positionsabhängigkeiten ist ziemlich zuverlässig und dennoch dynamisch genug um auch mit kleinen Veränderungen klar zu kommen.\\
Dies ist Wichtig, bei der Detektion von verschiedenen Gesichtern in einem Video und zuverlässiger als Active Appearance Model (AAM). 
\subsection{PDM \& GAVAM}
Mit Point Distribution Model (PDM) können verformbare Objekte recht gut dargestellt werden. Dabei wird die durchschnittliche Form $\overline{X}$ bestimmt und eine Matrix $P$ von Eigenvektoren um die möglichen Deformierungen darzustellen.
\begin{align*}
X &= \overline{X}+P\cdot b
\end{align*}
Somit kann durch einen Skalierungsvektor $b$ alle möglichen Formen $X$ des Objektes dargestellt werden. Zur Vereinfachung reicht es die signifikantesten Eigenvektoren in $P$ auf zu nehmen und dennoch $X$ ausreichend genau zu beschreiben.\\
Ist bekannt welche Art der Verformung durch den Eingenvektor dargestellt ist, z.B. eine bestimmte Orientierung, so kann anhand des Skallierungsvektors die Rotation des berechneten Objektes bestimmt werden, siehe Generalized Adaptive View-based Appearance Model (GAVAM). Eine Problematik bei dieser Art der Bestimmung der Rotation entsteht, wenn Neben der Verschiebung der Landmarks durch die Rotation, auch eine Deformierung stattgefunden hat und somit niemals eine eindeutige perfekte Lösung gefunden werden kann. Dies ist vor allem de Problematik wenn auf Gesichtern gerechnet werden soll, da immer eine Veränderung der Mundwinkel oder Augenlider vorhanden sind.\\
\cite{wiki_PDM}\cite{pdf_PDM}\cite{pdf_GAVAM}
\subsection{Non-maximum suppression  (NMS)}
Ein Verfahren um ein lokales Maximum zu bestimmen und kann z.B. in einem Bild eingesetzt werden um Kanten exakter zu bestimmen. Als Eingabe für das Verfahren im Beispiel, wird das Ergebnis eines Kantendetektor z.B. Kirsch-Operator verwendet. Dabei gibt die Stärke der Farbe eines Pixels an, wie nahe es an einer Kante im Originalbild liegen. Bei der Verarbeitung wird nun der Farbwert jedes einzelnen Pixels des Eingabebildes mit seinen umliegenden verglichen und sollte es nicht maximal sein auf Null gesetzt.\\
Auf diese Weise bleibt nur noch ein Kantenpixel übrig. Wird das Verfahren auf die Bestimmung von Boxen eingesetzt, so wird jene Fläche bestimmt die von allen am ehesten beschreiben wird.
\cite{wiki_Canny}\cite{NMS}
\section{MTCNN Face Detection}
\label{MTCNN}
Bei Multi-task Cascaded Convolutional Neuronal Network (MTCNN) handelt es sich um ein Verfahren dass bei der Detektion von Gesichtern auch deren Ausrichtung berücksichtigt, um bessere Ergebnis zu erzielen.
\subsection{Constrained Local Neural Fields (CLNF)}
Dabei handelt es sich um einen Gesichtsdetektor. Für die Detektion wird für jedes Merkmal ein eigener Detektor auf einem Bildbereich angewendet und eine eigene Wahrscheinlichkeitskarte erstellt.\\
Als nächster Schritt wird das Ergebnissen der Detektoren mit eine Karte der Position aller Landmarks, mit ihrer Abweichung, kombiniert um somit die beste Position der Landmarks zu erhalten auf Bezug des Farbverlaufes und relativ zu den anderen Landmarks.
\cite{CLNF}
\subsection{Patch Experts}
Eine Bewertung, wie wahrscheinlich ein Landmark an einer bestimmten Position im Bild dargestellt ist. Dazu wird ein Bereich um die Position ausgewertet.
\cite{CLNF}
\subsection{Anforderungen}
Sein Einsatzgebiet ist die Vorverarbeitung eines Frames für die spätere Auswertung. Somit soll dieser Schritt von einem möglichst robusten Verfahren zur Detektion von Gesichtern durchgeführt werden.\\
Dabei kann auf einem hochauflösendem Bild mit verhältnismäßig kleinen, verschieden großen und weit verteilten Gesichtern gearbeitet werden.
\subsection{Die 3 Stufen der Verarbeitung}
Für die gute Detektionsqualität sorgt die dreistufige Verarbeitung mit verschiedenen CNN auf einer Bildpyramide. Bei der Bildpyramide handelt es sich um ein in verschiedenen Größen skaliertes Bild, damit der gesuchte Inhalt in der gewünschten Auflösung abgebildet ist, ohne dass etwas über den Inhalt zu wissen.\\
Dies ist von Vorteil, damit das CNN auf eine feste Größe von Gesichtern optimiert werden kann, um das Lernen, neben dem möglichen Farbverläufen, durch die Skalierung nicht zusätzlich zu erschweren und die CNN können auch auf ihre jeweilige Aufgabe besser optimiert werden können.
\begin{figure}
	\centering
	\includegraphics[width=0.5\linewidth]{img/MTCNN_Step}
	\caption{Darstellung des Funktionsablaufes von MTCNN\cite{MTCCN}}
	\label{img_MTCNN_Step}
\end{figure}
\subsubsection{Stufe 1}
Beim ersten Verarbeitungsschritt werden alle Bereiche eines Bilds gesucht, in denen möglicherweise ein Gesicht zu erkennen ist. Dazu wird für die Detektion ein CNN, dem sogenannten Proposal Network (P-Net), eingesetzt um alle möglichen Bounding-Boxen zu ermitteln in denen ein Gesicht zu sehen sein könnte. Diese Bounding-Boxen werden anschließend mit einem NMS ausgedünnt, um die am stärksten überlappenden Boxen zusammen zu fassen.
\subsubsection{Stufe 2}
Anschließend werden die möglichen Bereiche mittels eines weiten CNN analysiert, damit alle Nicht-Gesichtsbereiche erkannt und entfernt werden können.\\
Dies wird von dem Refine Network (R-Net) übernommen und anschließend die möglichen Bounding-Boxen mittels NMS weiter reduziert.
\subsubsection{Stufe 3}
Der letzte Schritt wird von einem deutlich genaueren CNN übernommen, um ein Gesicht zu detektieren, dem sogenannten Output Network (O-Net). Womit die resultierenden exakten Boxen und mit ihren jeweiligen 5 Landmarks ermittelt werden.
\subsection{Qualität}
MTCNN Face Detection ist bei der Zuverlässigkeit im Verglich zu anderen bekannten Verfahren überlegen, siehe \autoref{img_MTCNN_quality} und zudem Echtzeit fähig. Im Test-Datensatz sind auch Gesichtern mit einer Größe von $20\times 20$ Pixel enthalten und wurden erfolgreich erkannt.\\
Somit sind alle Anforderungen erfüllt um mit diesem Verfahren den vorhanden Frame für die nachfolgenden Berechnungen vorzubereiten.
\begin{figure}
	\centering
	\includegraphics[width=0.5\linewidth]{img/MTCNN_quality}
	\caption{normale blaue Linie\cite{MTCCN}}
	\label{img_MTCNN_quality}
\end{figure}
%Joint Face Detection and Alignment using Multi-task Cascaded Convolutional Networks
\section{Skalieren von Bildern}
\label{scale_Algos}
Da die Berechnungen meist auf recht kleinen Bildausschnitten ausgeführt wird, müssen diese für weitere Rechenschritte Hochskaliert damit es von OpenFace besser verarbeitet werden kann.\\
Dabei ist es wichtig, dass die Gesichtsmerkmale möglichst gut rekonstruiert werden, um die entsprechenden Landmarks zu bestimmen.
\subsection{Nearest-Neighbor}
Dieses Verfahren verwendet als neuer Farbwert, den gleichen Wert wie das nächstgelegene Pixel. Dadurch werden nur die ehemaligen Pixel größer und das Gesicht wirkt sehr Kantig, da keine neuen Farbwerte bestimmt werden.
\begin{figure}
	\centering
	\includegraphics[width=0.2\linewidth]{img/lena100_NN}
	\includegraphics[width=0.2\linewidth]{img/lena100_NN_differenz}
	\includegraphics[width=0.2\linewidth]{img/Schachbrett_NN}
	\caption{Die ursprüngliche Abbildung von Lena betrug 100 Pixel Kantenlänge und beim Schachbrett 48 Pixel, beide wurden mittels Nearest-Neighbor auf 512 Pixel skaliert und bei Lena die Differenz bestimmt}
	\label{img_NN}
\end{figure}
\subsection{Linear}
Dabei wird zwischen den nächst gelegenen umliegenden Pixel linear Interpoliert, wodurch weitere Farbwerte entstehen. Das Ergebnis ist gleichmäßiger als Neares Neighbor, aber immer noch ein recht einfaches Verfahren. Die Kanten werden allerdings unscharf.
\begin{figure}
	\centering
	\includegraphics[width=0.2\linewidth]{img/lena100_LINEAR}
	\includegraphics[width=0.2\linewidth]{img/lena100_LINEAR_differenz}
	\includegraphics[width=0.2\linewidth]{img/Schachbrett_LINEAR}
	\caption{Die ursprüngliche Abbildung von Lena betrug 100 Pixel Kantenlänge und beim Schachbrett 48 Pixel, beide wurden mittels linearer Interpolation auf 512 Pixel skaliert und bei Lena die Differenz bestimmt}
	\label{img_Linear}
\end{figure}
\subsection{Bicubic}
Um den Farbwert zu ermitteln, werden die umliegenden $4\times 4$ Pixelwerte betrachtet um den Farbverlauf als eine Funktion 3. Grades zu bestimmen. Somit werden feinere Details besser dargestellt als beim linearen Verfahren und Kanten bleiben stärker erhalten. Allerdings kann es durch den bestimmten Verlauf auch zum Überschwingen kommen, wodurch Fehlfarben entstehen können.
\cite{wiki_Bicubic}
\begin{figure}
	\centering
	\includegraphics[width=0.2\linewidth]{img/lena100_CUBIC}
	\includegraphics[width=0.2\linewidth]{img/lena100_CUBIC_differenz}
	\includegraphics[width=0.2\linewidth]{img/Schachbrett_CUBIC}
	\caption{Die ursprüngliche Abbildung von Lena betrug 100 Pixel Kantenlänge und beim Schachbrett 48 Pixel, beide wurden mittels bikubischem Verfahren auf 512 Pixel skaliert und bei Lena die Differenz bestimmt}
	\label{img_Bicubic}
\end{figure}
\subsection{Lanczos}
Dieser Filter besteht aus einer Sinc-Funktion über einen Bereich, um so eine Bewertung der benachbarten Pixelwerte zu erhalten. Somit ergibt sich der neue Farbwert aus den bewerteten umliegenden Pixeln.\\
Die Funktion kann und wird für die Anwendung auf einen $8\times 8$ Bereich begrenzt. \cite{wiki_Lanczos}
\[ L(x)= \left\{ \begin{array}{ll}
\frac{\sin(\pi x)}{\pi x} \cdot \frac{\sin(\pi \frac{x}{a})}{\pi \frac{x}{a}} & \textrm{wenn } -a < x \ge<a, a\leq 0\\
1 & \textrm{wenn } x = 0\\
0 & \textrm{sonst}
\end{array}\right. \]
\begin{figure}
	\centering
	\includegraphics[width=0.2\linewidth]{img/lena100_LANCZOS4}
	\includegraphics[width=0.2\linewidth]{img/lena100_LANCZOS4_differenz}
	\includegraphics[width=0.2\linewidth]{img/Schachbrett_LANCZOS4}
	\caption{Die ursprüngliche Abbildung von Lena betrug 100 Pixel Kantenlänge und beim Schachbrett 48 Pixel, beide wurden mittels Lanczus-Verfahren auf 512 Pixel skaliert und bei Lena die Differenz bestimmt}
	\label{img_Lanczos}
\end{figure}
%% http://docs.opencv.org/2.4/modules/imgproc/doc/geometric_transformations.html#resize
\section{Graukonvertierung: Farbbild nach Graubild}
\label{Graubild}
Da die Berechnungen von ElSe auf Graubildern arbeitet und das Eingabebild in Farbe ist, muss es in ein Graubild umgewandelt werden.\\
Die Wahl des Verfahrens beruht auf der Anforderung von ElSe, dass vor allem der Farbunterschied zwischen Pupille und der Umgebung maximal sein soll, die Pupille möglichst dunkel und das restliche Auge hell. Die Farbe der Iris erschwert die Differenzierung zusätzlich, wenn diese recht dunkel ausfällt ist auch der Unterschied zur Pupille entsprechend gering in den Grauwerten. Außerdem ist das Erkennen der Pupille bei sehr kleinen Bildern schwierig bis unmöglich wodurch auf der Iris gerechnet werden muss, und daher diese weiterhin erhalten bleiben sollte.\\
Nach der Umwandlung wird für die Anwendung das Graubild noch normiert, damit mindestens ein schwarzes und ein weißes Pixel vorhanden ist.\\
Die Wahl von Gleam basiert auf den Ergebnissen von \glqq Color-to-Grayscale: Does the Method Matter in Image Recognition?\grqq \cite{rgb_to_Gray} und New-Gleam als eine Umsetzung des dort veröffentlichtem Ausblicks. Luminance als Standart, Quadrat als gegenstück zu Gleam und Min/Max beruht auf der Idee die farbige Iris zu nutzen.\\
Das Eingabebild der Beispiele der einzelnen Graukonvertierungen ist in \autoref{img_Gray_Einagbe} dargestellt. Eine Farbpalette, das Bildverarbeitungsbeispiel Lena sowie ein Augenbereich aus dem Augendatensatz \cite{database_Eye}. 
\begin{figure}
	\centering
	\includegraphics[width=0.19\linewidth]{img/Farbtafel2}
	\includegraphics[width=0.19\linewidth]{img/lena}
	\includegraphics[width=0.19\linewidth]{img/Auge}
	\caption{Dies sind die Eingabebilder der verschiedenen Konverter von Farbe nach Grau. Links eine Farbpalette, Mitte Lena und Rechts ein Augenausschnitt aus dem Augendatensatz \cite{database_Eye}}
	\label{img_Gray_Einagbe}
\end{figure}
\subsection{Gleam-Verfahren}
\label{gray_Gleam}
Bei dem Gleam-Verfahren wird jede Farbe (Rot,Gelb und Grün) gleich stark bewertet allerdings wird jeder Farbwert mittels einer Gamma-Korrektur verändert und das Bild wirkt heller als bei dem Luminance-Verfahren.\\
Durch die Gamma-Korrektur wird vor allem der helle Bereich weiter erhöht, somit wird der Farbunterschied zwischen Iris und Auge vermindert, wodurch die Pupille der einzige dunkle Bereich wird, siehe \autoref{img_Gleam}.\\
Allerdings wird auch dieser Farbwert erhöht und sollte die Pupille nicht schwarz sein, wird sie eher ins Graue überführt.\cite{rgb_to_Gray}\\
\[G_{Gleam}=\dfrac{R^{\frac{1}{2,2}} + G^{\frac{1}{2,2}} + B^{\frac{1}{2,2}}}{3}\]
\subsection{Gleam-New-Verfahren}
\label{gray_New}
Dies ist eine Variante von Gleam bei dem zuerst das gesamte Bild analysiert wird um die Parameter für die jeweilige Gamma-Korrektur zu ermitteln. Dies ist etwas aufwendiger, aber für die kleinen Bilder hinnehmbar.\\
Durch die individuelle Veränderung der Farbkanäle, werden Farbunterschiede minimiert und somit alle stark farbigen Bereiche ebenfalls dunkel dargestellt. Der Kontrast zwischen der farbigen Iris und dem weißen Auge wird verbessert, siehe \autoref{img_NewGeam}.\\
Da allerdings alle Farben dunkel werden, entstehen weitere dunkle Bereiche die die Detektion der Pupille beeinträchtigen können.
\[G_{Gleam-New}=\dfrac{R^{r} + G^{g} + B^{b}}{3}\]
Wobei gilt $\{r,g,b\} = \frac{\log(V_{\max})}{\log(\{R,G,B\}_{\max})}$ mit $V_{\max}$ als maximal möglicher Farbwert und $R_{\max}$ als maximal Vorhandener Rot-Wert, $G_{\max}$ und $B_{\max}$ äquivalent.
\subsection{Luminance-Verfahren}
\label{gray_Luminance}
Dies ist ein lineares Verfahren, das der menschlichen Farbwahrnehmung entspricht und oft den Standard bei der Umwandlung von Farbbild nach Graubild darstellt. Somit entsteht ein natürlicher Farbverlauf, bei dem der Farbunterschied zwischen Pupille, Iris und Auge auf einem mittleren Niveau bleibt, siehe \autoref{img_Luminance}.\\
Eine Gamma-Korrektur wird bei der Umwandlung nicht verwendet.\cite{rgb_to_Gray}
\[G_{Luminance} = 0,299 R + 0,587 G + 0,114 B\]
\subsection{Min-Max-Verfahren}
\label{gray_MinMax}
Dabei handelt es sich eigentlich um zwei verschiedene Varianten, allerdings funktionieren beide nach dem selben Prinzip. Als Grauwert wird der jeweilige Extremwert aus den einzelnen Farbkanälen des Pixels gewählt.\\
Durch Verwendung der Extremwerte, ist nur noch der Wert von Relevanz, nicht die eigentliche Farbe, wodurch das gesamte Bild deutlich heller bzw. dunkler wird.\\
Bei dem Max-Verfahren werden alle farbigen und helle Bereiche hell dargestellt und nur gleichmäßig dunkel Bereiche bleiben dunkel wie es bei schwarz der Fall ist. Wenn der Minimalwert anstelle verwendet wird, bleiben nur gleichmäßig helle Bereiche hell, alle anderen werden abgedunkelt.
\begin{align*}
G_{Max} &= \max(R,G,B)\\
G_{Min} &= \min(R,G,B)
\end{align*}
\subsection{Quadrat-Verfahren}
\label{gray_Quadrat}
Dies ist ein Verfahren, dass das Eingabebild verdunkelt und vom Aufbau dem Inversen von Gleam entspricht. Somit ist das gesamte Bild dunkler als bei dem Luminance-Verfahren, siehe \autoref{img_Quadrat}.\\
Durch die Abdunklung werden kleine Farbänderungen in den dunklen Bereichen reduziert, wodurch die Pupille sehr dunkel und der Farbunterschied zur Iris geringer ausfällt.
\[G_{Quadrat}=\dfrac{R^2+G^2+B^2}{3}\]
\subsection{Normalisierung der Graubilder}
Um ein Graubild zu erhalten, das das volle Spektrum der möglichen Grauwerte erfüllt, wird das Eingabebild normalisiert. Dazu wird der Maximale $G_{max}$ und Minimale $G_{min}$ Grauwert im Bild gesucht. Anschließend wird der neue Grau-Wert $G_{new}$ wie folgt bestimmt:
\[G_{new} = (G-G_{min})\cdot \dfrac{V_{max}}{G_{max}-G_{min}}\]
Dabei ist $V_{max}$ der maximale mögliche Wert in der Ausgabe und $G$ der aktuelle Grauwert im Bild.\\
Da für die Anwendung ein schwarzer Bereich gegen einen hellen Hintergrund gesucht wird, wird für die Bestimmung der Extremwerte nicht das originale Bild verwendet, sonder ein Gauß-gefiltertes.\\
Dies hat den Vorteil, dass einzelne lokal auftretende Werte, z.B. Reflektionen, nicht als Extremwert verwendet werden, wodurch die Pupille gleichmäßiger dunkler und das gesamte Bild stärker aufgehellt wird.\\
Der Gauß-Filters ist ein Tiefpassfilter und wird verwendet um das Eingangssignal zu glätten. Dies hat in der Bildverarbeitung den Effekt, dass Details im Bild verschwimmen und das Bild unscharf wirkt.\\
Die einzelnen Werte werden ihrer Umgebung angepasst, wodurch lokal auftretende Extremar verschwinden bzw. abgeschwächt werden und ähnliche Farbwerte zu ihrer Umgebung erhalten bleiben.
\begin{figure}
	\centering
	\includegraphics[width=0.19\linewidth]{img/Farbkarte2}
	\includegraphics[width=0.19\linewidth]{img/Lena2}
	\includegraphics[width=0.19\linewidth]{img/Auge_2Gray}
	\caption{Ergebnis der Umwandlung von Farb- nach Grauwert mittels Gleam-Verfahren}
	\label{img_Gleam}
\end{figure}
\begin{figure}[p]
	\centering
	\includegraphics[width=0.19\linewidth]{img/Farbkarte1}
	\includegraphics[width=0.19\linewidth]{img/Lena1}
	\includegraphics[width=0.19\linewidth]{img/Auge_1Gray}
	\caption{Ergebnis der Umwandlung von Farb- nach Grauwert mittels Gleam-New-Verfahren}
	\label{img_NewGeam}
\end{figure}
\begin{figure}[p]
	\centering
	\includegraphics[width=0.19\linewidth]{img/Farbkarte0}
	\includegraphics[width=0.19\linewidth]{img/Lena0}
	\includegraphics[width=0.19\linewidth]{img/Auge_0Gray}
	\caption{Ergebnis der Umwandlung von Farb- nach Grauwert mittels Luminance-Verfahren}
	\label{img_Luminance}
\end{figure}
\begin{figure}[p]
	\centering
	\includegraphics[width=0.19\linewidth]{img/Farbkarte5}
	\includegraphics[width=0.19\linewidth]{img/Lena5}
	\includegraphics[width=0.19\linewidth]{img/Auge_5Gray}\\
	\includegraphics[width=0.19\linewidth]{img/Farbkarte4}
	\includegraphics[width=0.19\linewidth]{img/Lena4}
	\includegraphics[width=0.19\linewidth]{img/Auge_4Gray}
	\caption{Ergebnis der Umwandlung von Farb- nach Grauwert mittels Extremwert-Verfahren. Oben: Max-Verfahren, Unten: Min-Verfahren}
	\label{img_MinMax}
\end{figure}
\begin{figure}[p]
	\centering
	\includegraphics[width=0.19\linewidth]{img/Farbkarte3}
	\includegraphics[width=0.19\linewidth]{img/Lena3}
	\includegraphics[width=0.19\linewidth]{img/Auge_3Gray}
	\caption{Ergebnis der Umwandlung von Farb- nach Grauwert mittels Quadrat-Verfahren}
	\label{img_Quadrat}
\end{figure}
\section{OpenFace}
\label{OpenFace}
Ein Open-Source Echtzeitverfahren auf Basis von CLNF zur Bestimmung und Analyse von Gesichtsmerkmalen in Grau-Bildern und Videos. Dabei stehen für diese Anwendung nur die Kameraparameter zur Verfügung und keinerlei Zusätze wie ein Tiefenbild (kann mitverwendet werden wenn vorhanden) oder Infrarotbeleuchtung der Szene.\\
OpenFace kann 68 Landmarks ermitteln die das Gesicht beschreiben, und mit deren Hilfe Position, Blickrichtung und Gesichtsmerkmale bestimmen. Sollte ein Video als Quelle fungieren, kann OpenFace auch lernen, wodurch eine zuverlässigere Verarbeitung erzielt werden kann.\\
Als Ergebnis ist die Kopfposition (Translation und Orientierung) sowie Blickrichtung von Interesse, da mit ihnen zurückrechnet werden kann wohin die Person schaut.\\
Der Rechenaufwand zur Verarbeitung des Eingabebildes ist so ausgelegt, das ein Webcam-Video in Echtzeit ausgewertet werden kann, dies ist im aktuellen Fall nicht notwendig, da es sich um eine nachträgliche Auswertung handelt, bei der es vor allem um Genauigkeit geht.
\subsection{Bestimmung der Landmarks}
\label{bestimmung_Landmarks}
Für die Bestimmung der Landmarks wird OpenFace auf den Bildausschnitten eingesetzt. Dies bietet mehrere Vorteile, so wird nur auf Bildbereichen gearbeitet, in denen ein Gesicht zu sehen ist und unnötige Suche vermieden. Außerdem kann für jede Person die passende Initialisierung des CLNF basierend auf dem letzten Ergebnis dieser Person gewählt werden, auch für jene die nur selten dargestellt sind. Auf diese Weise kann der Bildausschnitt möglichst exakt und gleichzeitig mit den anderen ausgewertet werden.\\
Für die eigentliche Bestimmung der Landmarks bietet OpenFace zwei verschiedene Methoden, die Berechnung auf Bildern und Videos. Der Hauptunterschied ist das Lernen, dass bei der Videoauswertung verwendet wird, wodurch sich der Arbeitsbereich deutlich erhöht und bessere Ergebnisse liefert werden. Dies liegt an der Anpassung des Modells und dem möglichen Tracking der Landmarks.\\
Dies ist Interessant für die spätere Anwendung, da somit auch Einzelbilder verwendet werden können, die eine deutlich höhere Auflösung besitzen als ein Video. Allerdings sinkt bei der Verwendung von Einzelbilder der maximale Winkel relativ zur Kamera beträchtlich. Außerdem hat sich gezeigt das bei Verwendung eines Videos das Gesicht deutlich kleiner dargestellt sein kann bis keine Auswertung mehr möglich ist und sollte ein Gesicht im aktuellen Farame erfolgreichen detektiert werden auch die nachfolgenden Frames durch das lernen ausgewertet werden können.\\
Dennoch kann es passieren, dass trotz allem ein Gesicht falsch detektiert wird, wie z.B. das Erkennen eines sehr kleinen Gesichtes innerhalb einer Ohrmuschel. In solch einem Fall muss das CLNF zurückgesetzt werden, damit sich der Fehler nicht fortpflanzt.
\subsubsection{Gesichts-Landmarks: Detektion und Verfolgung}
Für die Bestimmung und Tracking der Landmarks wird ein Conditional Local Neural Fields (CLNF) eingesetzt. Dabei Handels es sich im Grunde um ein Constrained Local Model (CLM) nur mit verbesserter Patch Experts und Optimierungsfunktionen.\\
Die beiden Hauptkomponenten des CLNF von OpenFace ist das Point Distribution Model (PDM) zur Erfassung der Anordnung der Landmarks und Patch Experts zum Erfassen der Variante der einzelnen Landmarks.\\
Zu Beginn werden verschiedene initiale Hypothesen aus der dlib-Bibliothek verwendet und die Passende zur Eingabe ausgewählt. Bei den unterschiedlichen initial Hypothesen handelt es sich um die Darstellung verschiedener Gesichtsorientierungen auf denen unterschiedliche Netze trainiert wurden. Dies Herangehensweise ist langsam, aber auch exakter als eine einfache Hypothese. Wird ein Tracing, das Verfolgen der Landmarks über mehrere Frames, durchgeführt wird als initiale Hypothese das Ergebnis aus dem letzten Frame verwendet. Sollte das Tracing scheitern, wird das CNN reseted um Neu zu beginnen.\\
Auf diese Weise werden 68 Gesichts-Landmarks und  weitere 28 pro Auge erfasst. Zur Berechnung auf den Gesichtern sollten diese laut Paper \cite{OpenFace} eine Optimalgröße von 100 Pixeln für eine zuverlässige Detektion aufweisen.
\subsubsection{Detection der Gesichtsmerkmale}
Dieser Schritt kann von OpenFace ausgeführt werden, ist aber im aktuellen Fall nicht von Relevanz, da die Blickrichtung von Interesse ist und nicht die Mimik der Probanden.
\subsection{Veröffentlichte Genauigkeit}
Um die Qualität der Berechnung auf dem Kopf zu bewerten wurde der \glqq Biwi Kinect head pose\grqq \cite{BIWI_database},\glqq ICT-3DHP\grqq \cite{ICT_database} und \glqq BU Datensatz\grqq \cite{BU_database} ausgewertet. Dabei handelt es sich um Portrait-Fotos von Probanden, deren Körper in Richtung Kamera ausgerichtet ist und ihren Kopf in eine beliebige Richtung drehen. Für die Genauigkeit der Kopfposition haben sich folgend Werte ergeben in Grad:\\
\begin{tabular}{|l|c|c|c||c|c|}
	\hline
	&Yaw&Pitch&Roll&Mean&Median\\\hline
	Biwi Kinect&7.9&5.6&4.5&6.0&2.6\\\hline
	BU dataset&2.8&3.3&2.3&2.8&2.0\\\hline
	ICT-3DHP&3.6&3.6&3.6&3.6&-\\\hline
\end{tabular}\\\\
Für die Qualität wurde der Augendatensatz \glqq Appearancebased
gaze estimation in the wild\grqq \cite{database_Eye_old} zur Bestimmung der Blickrichtung verwendet und es ergab sich ein durchschnittlichen Fehler von 9.96 Grad.
\subsection{Auswirkung der Größe}
Durch die Aufgabenstellung muss das Verfahren zuverlässig bezüglich der Distanzen bzw. Darstellungsgröße sein. Zur Messung wurde der Datensatz von Labeled Faces in the Wild \cite{database_Face} verwendet. In diesem Datensatz ergibt sich im Originalbild eine durchschnittliche Kopfbreite von 94 Pixel. Bei Random Forests for Real Time 3D Face Analysis \cite{database_Face_Ori} ist die durchschnittliche Breite 78 Pixel.\\
Zur Durchführung wurden die Größe der Bilder mit dem Skalierungsfaktor multipliziert und linear verkleinert um so kleinere, weiter entferntere Gesichter zu erhalten und anschließend mit dem Image-Detector von OpenFace zu verarbeiten. Das Ergebnis ist in \autoref{img_lineareverkleinerung} dargestellt.\\
Es ist zu erkennen, dass die Wahrscheinlichkeit auf eine erfolgreiche Detektion ab $0.5$, also Gesichert mit etwa 47 Pixel Breite, rapide abnimmt. Bei der in \autoref{hardware} beschriebenen Kamera entspricht dies einer Distanz von etwa $4.5m$.\\
Bei der maximalen Distanz auf der gearbeitet werden soll $(8.5m)$ ergibt sich eine Gesichtsgröße von etwa 22 Pixel, das einer Skalierung von 0.25 entspricht. Bei dieser Bildgröße ist in der Standardanwendung ohne Skalierung keine Detektion möglich, siehe \autoref{img_lineareverkleinerung}.
\begin{figure}
	\centering
	\includegraphics[width=0.45\linewidth]{img/lineare_Verkleinerung}
	\includegraphics[width=0.45\linewidth]{img/lineare_Verkleinerung2}
	\caption{Die Bilder aus Labeled Faces in the Wild \cite{database_Face} (links) und Random Forests \cite{database_Face_Ori} wurden mit den Faktor auf der X-Achse linear verkleinert und die Erkennungsrate Y-Achse abgebildet}
	\label{img_lineareverkleinerung}
\end{figure}
\subsection{Auswirkung der verschiedenen Skalierungesverfahren auf Detektion}
\label{OpenFace_skal}
Um auf den gewünschten Distanzen arbeiten zu können, wird der jeweilige Bereich hochskaliert. Dazu wird das Ursprüngliche Bild $(250\times 250)$ linear um den angegebene Faktor verkleinert und anschließend mit den angegebenen Verfahren auf $300\times 300$ wieder vergrößert, damit die abgebildeten Gesichter in etwa 100 Pixel groß sind. Die Wahrscheinlichkeit auf eine Detektion ist in \autoref{img_hochskalliern} dargestellt.\\
Es ist zu erkennen das durch die Vergrößerung, jene Gesichter in Bereichen die normal nicht erkennbar sind, ausgewertet werden können.\\
Als das ungeeignetste Verfahren hat sich Nearest-Neighbor herausgestellt, siehe blaue Linie \autoref{img_hochskalliern}, da die Detektionsrate deutlich früher abfällt als bei den anderen. Die anderen haben sehr ähnliche Ergebnisse, nur das Lineare Verfahren ist etwas schlechter. Dennoch werden die Anforderungen, eine Detektion auf Gesichtern mit 22 Pixel (Skalierung 0.25), von allen erfüllt.\\
Ausgehend vom Skalierungsfaktor des Linearen-, Bicubic- und Lanczos-Verfahren wären mit der verwendeten Kamera auch Distanzen bis zu $14m$ möglich. Allerdings ist das Bild durch die Verkleinerung  deutlich besser als Originalaufnahmen, da Pixelrauschen und Ähnliches nicht vorhanden ist.
\begin{figure}
	\centering
	\includegraphics[width=0.5\linewidth]{img/Hochskalliern}
	\caption{Die Bilder aus Labeled Faces in the Wild \cite{database_Face} wurden mit den Faktor auf der X-Achse linear verkleinert und mit den verschiedenen Verfahren wieder vergrößert \autoref{scale_Algos}. Aufgetragen gegen die Detektionswahrscheinlichkeit.
		Nearest-Neighbor (blau), Linear (rot), Bicubic (braun), Lanczos (grün)}
	\label{img_hochskalliern}
\end{figure}
\subsection{Auswirkung von Pixelrauschen auf Detektion}
Mit diesem Test soll geprüft werden, welches der Verfahren auch stabil gegen Rauschen ist.\\
Um Pixelrauchen zu simulieren, wurden die Bilder aus Labeled Faces in the Wild \cite{database_Face} entsprechend verkleinert, mit Rauschen versehen um sie anschließend mit den unterschiedlichen Verfahren zu vergrößern.\\
Das Rauschen wird für jedes Pixel simuliert, indem eine Wahrscheinlichkeit von $50\%$ besteht, eine gleich verteilte Abweichung von $\pm 10\%$ des Originalen Farbwertes. Dieser Vorgag wurde für jedes Bild viermal wiederholt um Zufälligkeiten bei der Rauschsimulation zu vermeiden.\\
Wie zu erwarten ist Nearest-Neighbor am schlechtesten, aber auch zwischen den anderen Verfahren sind nun unterscheiden zu erkennen, siehe \autoref{img_hochskalliern_nois}. Die gesamte Erkrankungsrate ist signifikant kleiner als ohne Rauschen, wobei die Position $(0.15)$, ab welcher die Erkennungsrate rapide abfällt, beibehalten wird.
\begin{figure}
	\centering
	\includegraphics[width=0.7\linewidth]{img/Hochskalliern_Nois}
	\caption{Bilder aus Labeled Faces in the Wild \cite{database_Face}, mit dem X-Faktor verkleinert, um jedes Pixel mit $50\%$ Wahrscheinlichkeit auf $\pm 10\%$ Gleichverteilung der Abweichung}
	\label{img_hochskalliern_nois}
\end{figure}
\subsection{Arbeitsbereich bezüglich Rotation}
Von Interesse sind auch die Winkel, bei den Gesichter in verschiedenen Skalierungen noch erkannt werden, siehe \autoref{img_Rot_Value}.\\
Hier ist zu erkennend das der Wertebereich ab 0.7 abnimmt und ab 0.5 recht schnell. Dieser Bereich ist von Interesse, da selbst wenn ein Gesicht in dieser Größe allerdings außerhalb des Wertebereiches vorhanden sein sollte, dieses nicht erkannt wird.\\
Der Wertebereich auf den einzelnen Achsen ist ausreichend für die Anwendung sollte das Ziel der Aufmerksamkeit sich in der näher der Kamera befinden. Auch wenn die Rotation parallel zur Horizontalen etwas größer sein könnte.
\begin{figure}
	\centering
	\includegraphics[width=0.3\linewidth]{tabelle/X_Rot}
	\includegraphics[width=0.3\linewidth]{tabelle/Y_Rot}
	\includegraphics[width=0.3\linewidth]{tabelle/Z_Rot}
	\caption{Darstellung der noch detektierten Wertebereiche in Bogenmaß.}
	\label{img_Rot_Value}
\end{figure}
\section{Berechnung der Position}
\label{calc_Position}
Zur Bestimmung der Position $P=(X_{avg};Y_{avg};Z_{avg})$ des Gesichtes im Kamerakoordinaten wird die Größe, ein Skalierungsfaktor $S$, des Kopfes im Bild verwendet.\\
Da bei der Abbildung von den Koordinaten ins Bild gilt $x=f\cdot \frac{X}{Z}$ und $ y=f\cdot \frac{Y}{Z}$, somit kann die Tiefe wie folgt abgeschätzt werden.\\
Sei $P_1 = (X_1;Y_1;Z_1), P_2=(X_2;Y_2;Z_2)$ die Beschreibung der Größe $G$ eines Kopfes mit.\\
\begin{align*}
a &= \frac{\sqrt{(X_1-X_2)^2+(Y_1+Y_2)^2}}{\frac{Z_1-Z_2}{2}} =\frac{G}{Z_{avg}}\\
S &= \frac{S_G}{G}\\
\Rightarrow a\cdot f &= f\cdot\frac{G}{Z_{avg}} = S_G\\
Z_{avg} &= \frac{f}{S_G}\cdot G = \frac{f}{S}\\
X_{avg} &= \frac{x \cdot Z_{avg}}{f}\\
Y_{avg} &= \frac{y \cdot Z_{avg}}{f}\\
\end{align*}
Dies beschreibt allerdings nur eine Annäherung an die Tatsächliche Position, da mit einem Durchschnittlichen Kopfgröße gerechnet wird und je weiter der Kopf vom Zentrum entfernt ist dies mit beachtet werden muss.
\subsection{To Do}
\begin{itemize}
\item Bestimmung der Ziele - genauer\\
Einbeziehen der Pixelrichtung
\end{itemize}
\section{Bestimmung der Position \& Orientierung}
\label{bestimmung_Pos}
Für die Bestimmung der Position und Orientierung des Gesichtes wird wie in \ref{calc_Position} Beschreiben ausgeführt. Dies kann Wiederrum von OpenFace übernommen werden, dazu muss nur das Zentrum des Bildes und Brennweite $f_x,f_y$ bekannt sein. Außerdem werden noch erweiterte Verfahren angeboten, bei dem die Position im Bild besser mit einbezogen werden, um die Winkel der Kameraabbildung zu berücksichtigen.\\
Der signifikanteste Parameter für die Position ist die Brennweite $f_x$, da mit ihm die Tiefe geschätzt wird und sollte entsprechend exakt bestimmt sein. Von Interesse ist vor allem der Punkt auf den der Blick bzw. das Gesicht ausgerichtet ist, dadurch muss neben der Position im Kamerakoordinatensystem auch die Orientierung bekannt sein.\\
Da nur de Position des Kopfes und seine Orientierung bestimmt werden kann, ergibt sich das Problem, den konkreten Blickpunkt zu ermitteln, da ein ganzer Kegel, wenn eine Fehlertoleranz berücksichtigt wird, die als mögliche Lösungen in Frage kommen.\\
Außerdem liegt der Blickpunkt meist außerhalb des Bereiches der Kamera und muss entsprechend von einer Anwendung interpretiert werden.
\section{ElSe}
\label{ElSe}
Ellipse Selection for Robust Pupil Detection (ElSe), ist ein Algorithmus zur Bestimmung der Pupille in einem hochauflösenden Aufnahme des Auges unter realen Bedingungen.
\subsection{Aufbereitung der Bildinformation in der Augenregion}
\label{verbesserung_ElSe}
Zur Bestimmung der Blickrichtung ist die Augenregion natürlich von besonderer Bedeutung. Aus diesem Grund werden die Landmarks der Augenregion nochmals gesondert betrachtet. Aufgrund der besonderen Bedeutung existiert eine große Anzahl an Algorithmen, die speziell auf eine hochgenaue Bestimmung von Augenmerkmalen optimiert sind, wie Beispielsweise ElSe \cite{ElSe}, Goutam \cite{Eye_FastCorner}, Starburst \cite{Starburst}, Swirski \cite{Swirski2012}.\\
Daher bestimmt OpenFace zusätzlich zu den 64 Landmarks, die das Gesicht beschreiben, weitere 28 Landmarks pro Auge, aus denen die Blickrichtung ermittelt wird. Dazu kommt ein weiteres CLNF zum Einsatz, das auf Augen Trainiert wurde. Dabei zeigten die Vorabtests, dass die Detektionsgenauigkeit bei den getesteten kleinen Gesichtern unzureichend ausfällt.\\
Um die Position der Landmarks zu verbessern, kann auf dem Bildausschnitt der Augen der ElSe-Algotithmus eingesetzt werden. Dieser Algorithmus arbeitet auf einem Farbbild um so die Umrisse der Pupille zu berechnen. Dieses Verfahren wurde gewählt, da es im Test \cite{ElSe} am besten abgeschnitten hat und direkt das Zentrum der Pupille liefert.\\
Für die Bestimmung der Blickrichtung ist vor allem das Zentrum der Pupille und Iris sowie deren Umrisse ausschlaggebend, daher müssen diese aus dem Ergebnis von ElSe abgeleitet werden.\\
Der ElSe Algorithmus wurde für Eye-Tracking Brillen entwickelt, die die Augenregion hochauflösend abbilden. Entsprechend nimmt die Detektionsleistung bei niedriger auflösenden Bildern rasch ab und da diese Berechnung unabhängig der Landmarks ausgeführt wird, empfiehlt sich das Ergebnis zu überprüfen, damit die bestimmten Landmarks auch innerhalb der Augenhöhle liegen.\\
Bei der Berechnung wird jedes Auge unabhängig vom anderen ausgeführt. Durch die Messungenauigkeit und bei nahe an der Person befindlichen Blickzielen können die Blickrichtungen beider Augen verschieden sein. Wird ein weiter entfernterer Punkt von beiden Augen fokussiert, so kann die Blickrichtung beider Augen als parallel angenommen werden, da der unterschied zwischen Beiden minimal ausfällt. Um den Fehler zu minimieren wird als Ergebnis die durchschnittliche Blickrichtung beider Augen verwendet.
\subsection{Beschreibung}
Bei realen Aufnahmen sind Bildfehler unvermeidlich, so können Reflektionen (Brille, Kontaktlinse usw.), Make-Up und körperliche Eigenschaften wie Augenfarbe die Detektion erschweren.\\
Der Ursprüngliche ElSe-Algorithmus ist für Graubilder einer Eye-Tracking-Brille ausgelegt und optimiert. Dies betrifft vor allem die Qualität der Aufnahme im Bezug auf die Auflösung und die Infrarotbeleuchtung des Bildes, zudem ist es auf diesen Bildern zu einer Echtzeitauswertung in der Lage. Die Infrarotbeleuchtung wird verwendet, damit das Auge ausreichend beleuchtet ist ohne den Probanden zu blenden.\\
Für die Anwendung wurde ElSe angepasst um auf Farbbilder die nach Grau konvertiert wurden, arbeiten zu können. Ziel ist es die Blickrichtung möglichst exakt zu bestimmen, wofür die Landmarks der Pupille ausschlaggebend sind.\\
Als Ergebnis liefert ElSe eine Ellipse, die den Umriss der Pupille im Bild beschreibt, aus der die Landmark abgeleitet werden können.
\subsubsection{Pupille bestimmen mit Kantendetektion}
Da die Pupille als schwarzen Fleck im Bild dargestellt ist und die Iris einen helleren Farbton aufweist, wird ein Kantendetektor verwendet, der alle Pixel markiert, bei denen eine starke Farbänderung auftritt. Bei ElSe wird ein Morphologischen Ansatz eingesetzt, von Relevanz sind nur zusammenhängende Kantenpixel um die Kante zwischen Pupille und Iris zu finden, alle anderen können ignoriert werden. Wobei jedes Kantenpixel als Startpunkt der Berechnung dienen kann.\\
Um jene Kantenpixel zu erhalten, die die Pupille beschreiben, wird versucht fortlaufende Kanten zu finden, die eine Ellipse bilden. Jene die nicht diesen Anforderung entsprechen, können recht schnell ignoriert werden. Anschließend können auch alle offenen Ellipsenverläufe und jene die am meisten vom bestimmten Verlauf abweichen, verworfen werden.\\
Das beste Ergebnis aller bestimmten Ellipsen wird als Lösung verwendet.
\subsubsection{Grobe Bestimmung der Pupille}
\label{ElSe_Grob}
Sollte die Bestimmung der Ellipse, wie im letzten Kapitel beschreiben, scheitern, so wird das Zentrum des dunkelsten Kreises ermittelt. So ein Punkt kann immer gefunden werden, ist aber nicht zwingend die Pupille.\\
Auf einem verkleinertem Bild \autoref{img_else} (1) wird ein kreisförmiger Mean-Filter eingesetzt mit Ergebnis in \autoref{img_else} (3). Zur zweiten Faltung wird der Durchschnitt über ein Quadrat ohne inneren Kreis eingesetzt mit Ergebnis in \autoref{img_else} (2), wobei bei beiden Kreisen der selbe Radius verwendet wird.\\
Nun wird das Ergebnis des Quadratischen Mean-Filters invertiert \autoref{img_else} (4) und mittels Punkt-Multiplikation mit dem anderen Meanfilter zusammengebracht \autoref{img_else} (5). Im resultierendem Bild wird nun der höchste Wert gesucht, da dies das Zentrum des dunkelsten kreisförmigen Ortes im Bild ist.\\
Ergebnis des Beispiels ist als Kreuz in \autoref{img_else} (6) markiert. 
\begin{figure}
	\centering
	\includegraphics[width=0.8\linewidth]{img/ElSe}
	\caption{Ablauf der alternativen Berechnung zur Pupillen-Detektion von \cite{ElSe}}
	\label{img_else}
\end{figure}
\subsubsection{Ergebnisse}
Für den Test, wurden Bilder von $384\times 288$ Pixel Größe verwendet.
Im Vergleich zu den anderen Verfahren, ist ElSe in den meisten Fällen ihnen überlegen, mit einer Verbesserung der Erkennungsrate um $14.53\%$ auf dem verwendeten Datensatz \cite{ElSe}.\\
Ein Problem entsteht wenn der Farbunterschied zwischen Iris und Pupille recht gering ausfällt oder durch Reflektionen der Kantenverlauf gestört wird.\\
Für die Anwendung im aktuellen Fall, ist der Bereich der Augen sehr klein und eine eindeutige Detektion entsprechend schwierig, wodurch vor allem die grobe Bestimmung der Ellipse von Interesse ist.
\subsection{Versuchsaufbau für die Auswirkung der Graubild-Verfahren auf ElSe}
Um die einzelnen Verfahren besser Vergleichen zu können, wurden künstliche Augen aus dem Datensatz \cite{database_Eye} verwendet damit die exakte Position der Landmarks bekannt ist.\\
Ein gutes Verfahren muss stabil gegenüber der Skalierung sein, damit es auch auf kleinen Bereichen zuverlässig arbeitet. Da für die spätere Anwendung vor allem das Zentrum der Pupille von Interesse ist, wird der Abstand zum Zentrum als Qualitätsmaß verwendet.\\
Da ElSe für Eye-Tracking Brillen entwickelt wurde, also für ein Qualitativ hochwertiges Bild eines Auges, wurde der Bildbereich soweit verkleinert das noch alle Landmarks des Auges mit etwas Rand dargestellt wird, um diesen Anforderungen entsprechend nahe zu kommen.\\
Somit sind die Bildausschnitten im Datensatz auf denen gerechnet wird etwa 64 auf 29 Pixel groß und werden für die Verarbeitung auf eine Breite von 384 Pixeln vergrößert. Somit ergibt sie die Bildgröße für die ElSe entwickelt wurde, da durch die Skalierung allerdings keine zusätzlichen Informationen entstehen, ist vor allem die grobe Bestimmung der Ellipse, beschreiben in \autoref{ElSe_Grob}, von Interesse. Diese Auswahl des Bildbereiches kann auch in der späteren Anwendung eingesetzt werden, da der Augenbereich durch eigene Landmarks in der Gesichtsanalyse relativ genau bestimmt ist.\\
Um die Qualität der Berechnung bei verschiedenen Größen zu simulieren, wurde das Bild linear verkleinert.
\subsection{Auswirkung des Radius}
Ein wichtiger Parameter des ElSe-Verfahrens ist der Radius des Filters. Um den besten Parameter zu bestimmen wurde der Augen-Datensatz \cite{database_Eye} verwendet und die Augenpartie ausgeschnitten. Im Datensatz besitzen die abgebildeten Augen durchschnittlich 15 Pixel Breite Pupille und eine Iris von 34 Pixel Durchmesser.\\
In \autoref{ElSe_Gray_Iris} und \autoref{ElSe_Gray_Zentrum} ist zu erkennen, dass der Radius signifikant für die Qualität der Berechnung ist. Da für die spätere Anwendung vor allem das Zentrum der Pupille von Interesse ist, vgl. \autoref{OpenFace_Blickrichtung}, muss ElSe in diesem Aspekt zuverlässig Ergebnisse liefern.\\
Im Versuch hat sich ein Radius von etwa einem Zwölftel des zu erwartetem Durchmesser der Iris bzw. Pupille als sinnvoll erwiesen, um deren Ausmaße möglichst exakt zu bestimmen. Im Versuch entspricht dies 8 und 18 Pixel. Um die Position des Zentrums der Iris und der Pupille möglichst gut zu bestimmen, erwies sich ein Radius von 10 am besten, siehe \autoref{ElSe_Gray_Zentrum}, wobei dieser Fehler nicht so sehr steigt bei Veränderung des Radius, als bei der Größenbestimmung von Pupille und Iris.
\subsection{Auswirkung der verschiedenen Graubild-Verfahren}
Es zeigt sich das die Verfahren, um den Farbwert in einen Grauwert zu überführen, durchaus Auswirkungen auf die Qualität der Berechnung hat.\\
Das beste Ergebnis liefert das Gleab-Verfahren (Beschreiben in \autoref{gray_Gleam}) mit einer Abweichung von 5.89 Pixeln, siehe \autoref{ElSe_Gray_Zentrum}, da die Abweichung vom Zentrum minimal ist. Ein mittleres Ergebnis liefert das Luminance-Verfahren, beschreiben in \autoref{gray_Luminance}, mit welchem eine Abweichung auf dem Augen-Trainingsdatensatz von 6.42 Pixel erreicht wird.\\
Im Vergleich liefert das Quadratische-Verfahren, beschreiben in \autoref{gray_Quadrat}, die schlechtesten Ergebnis, da die durchschnittliche Abweichung bei 7.23 Pixel liegt.\\
Bei der Berechnung auf verschieden groß skalierten Bildern ist die Abweichung von ElSe bei Verwendung von Gleam konstant bei etwa 5.9 Pixel und arbeitet somit stabil, siehe \autoref{ElSe_scall}.
\begin{figure}
	\centering
	\includegraphics[width=0.32\linewidth]{Eye_Img_Box/Norm_Radius_A}
	\includegraphics[width=0.32\linewidth]{Eye_Img_Box/Gleam_Radius_A}
	\includegraphics[width=0.32\linewidth]{Eye_Img_Box/New_Radius_A}
	\includegraphics[width=0.32\linewidth]{Eye_Img_Box/Qua_Radius_A}
	\includegraphics[width=0.32\linewidth]{Eye_Img_Box/Min_Radius_A}
	\includegraphics[width=0.32\linewidth]{Eye_Img_Box/Max_Radius_A}
	%\includegraphics[width=0.49\linewidth]{Eye_Img/Normal_Abstand_P}
	%\includegraphics[width=0.49\linewidth]{Eye_Img/Gleam_Abstand_P}
	%\includegraphics[width=0.49\linewidth]{Eye_Img/New_Abstand_P}
	%\includegraphics[width=0.49\linewidth]{Eye_Img/Quadrat_Abstand_P}
	%\includegraphics[width=0.49\linewidth]{Eye_Img/Min_Abstand_P}
	%\includegraphics[width=0.49\linewidth]{Eye_Img/Max_Abstand_P}
	\caption{Abstand des Zentrums der Landmark-Pupille und der berechneten Ellipse in [Pixel/Skalierung]\\Oben-Links: Luminance, Oben-Mitte: Gleam, Oben-Rechts: Gleam New, Unten-Links: Quadrat, Unten-Mitte: Min-Wert, Unten-Rechts: Max-Wert}
	\label{ElSe_Gray_Zentrum}
\end{figure}
\begin{figure}
	\centering
	\includegraphics[width=0.32\linewidth]{Eye_Img_Box/Norm_Radius_P}
	\includegraphics[width=0.32\linewidth]{Eye_Img_Box/Gleam_Radius_P}
	\includegraphics[width=0.32\linewidth]{Eye_Img_Box/New_Radius_P}
	\includegraphics[width=0.32\linewidth]{Eye_Img_Box/Qua_Radius_P}
	\includegraphics[width=0.32\linewidth]{Eye_Img_Box/Min_Radius_P}
	\includegraphics[width=0.32\linewidth]{Eye_Img_Box/Max_Radius_P}
	%\includegraphics[width=0.49\linewidth]{Eye_Img/Normal_Width_P}
	%\includegraphics[width=0.49\linewidth]{Eye_Img/Gleam_Width_P}
	%\includegraphics[width=0.49\linewidth]{Eye_Img/New_Width_P}
	%\includegraphics[width=0.49\linewidth]{Eye_Img/Quadrat_Width_P}
	%\includegraphics[width=0.49\linewidth]{Eye_Img/Min_Width_P}
	%\includegraphics[width=0.49\linewidth]{Eye_Img/Max_Width_P}
	\caption{Unterschied Zwischen den Radien der Landmark-Pupille und der Berechneten Ellipse in [Pixel/Skalierung]\\Oben-Links: Luminance, Oben-Mitte: Gleam, Oben-Rechts: Gleam New, Unten-Links: Quadrat, Unten-Mitte: Min-Wert, Unten-Rechts: Max-Wert}
	\label{ElSe_Gray_Pupille}
\end{figure}
\begin{figure}
	\centering
	\includegraphics[width=0.32\linewidth]{Eye_Img_Box/Norm_Radius_I}
	\includegraphics[width=0.32\linewidth]{Eye_Img_Box/Gleam_Radius_I}
	\includegraphics[width=0.32\linewidth]{Eye_Img_Box/New_Radius_I}
	\includegraphics[width=0.32\linewidth]{Eye_Img_Box/Qua_Radius_I}
	\includegraphics[width=0.32\linewidth]{Eye_Img_Box/Min_Radius_I}
	\includegraphics[width=0.32\linewidth]{Eye_Img_Box/Max_Radius_I}
	%\includegraphics[width=0.49\linewidth]{Eye_Img/Normal_Width_I}
	%\includegraphics[width=0.49\linewidth]{Eye_Img/Gleam_Width_I}
	%\includegraphics[width=0.49\linewidth]{Eye_Img/New_Width_I}
	%\includegraphics[width=0.49\linewidth]{Eye_Img/Quadrat_Width_I}
	%\includegraphics[width=0.49\linewidth]{Eye_Img/Min_Width_I}
	%\includegraphics[width=0.49\linewidth]{Eye_Img/Max_Width_I}
	\caption{Unterschied Zwischen den Radien der Landmark-Iris und der Berechneten Ellipse in [Pixel/Skalierung] gegen die Radius-Größe.\\ Oben-Links: Luminance, Oben-Mitte: Gleam, Oben-Rechts: Gleam New, Unten-Links: Quadrat, Unten-Mitte: Min-Wert, Unten-Rechts: Max-Wert}
	\label{ElSe_Gray_Iris}
\end{figure}
\begin{figure}
	\centering
	\includegraphics[width=0.32\linewidth]{Eye_Img_Box/Norm_Radius_P_8}
	\includegraphics[width=0.32\linewidth]{Eye_Img_Box/Norm_Radius_A_10}
	\includegraphics[width=0.32\linewidth]{Eye_Img_Box/Norm_Radius_I_18}\\
	\includegraphics[width=0.32\linewidth]{Eye_Img_Box/Gleam_Radius_P_8}
	\includegraphics[width=0.32\linewidth]{Eye_Img_Box/Gleam_Radius_A_10}
	\includegraphics[width=0.32\linewidth]{Eye_Img_Box/Gleam_Radius_I_18}
	%\includegraphics[width=0.32\linewidth]{Eye_Img/Normal_Width_P_8}
	%\includegraphics[width=0.32\linewidth]{Eye_Img/Normal_Abstand_P_11}
	%\includegraphics[width=0.32\linewidth]{Eye_Img/Normal_Width_I_18}\\
	%\includegraphics[width=0.32\linewidth]{Eye_Img/Gleam_Width_P_8}
	%\includegraphics[width=0.32\linewidth]{Eye_Img/Gleam_Abstand_P_11}
	%\includegraphics[width=0.32\linewidth]{Eye_Img/Gleam_Width_I_18}
	\caption{Auswirkung von der Bildgröße auf die Qualität der Berechnung.\\ Oben: Luminance, Unten Gleam}
	\label{ElSe_scall}
\end{figure}
\subsection{Vergleich zu OpenFace}
Als Referenz wird das Ergebnis von OpenFace, für die zusätzlich bestimmten Landmarks der Augen, verwendet. Dies wurde auch auf dem Augendatensatz \cite{database_Eye} angewendet um vergleichbare Ergebnisse zu erhalten.\\
In \autoref{OpenFace_Eye} ist zu erkennen dass dieses Verfahren im Schnitt oft schlechtere Ergebnisse liefert als das Ergebnis von ElSe, allerdings ohne das begehen von großen Fehlern und auch öfters genauere.\\
Da die hohe Qualität von ElSe nur erreicht werden kann, wenn es auf passenden Bildausschnitt angewendet wird, ist auch die Detektion des Auge von Interesse.\\
Nach \autoref{OpenFace_Eye_Box} ist zu entnehmen, das der Bereich des Auges zwar nicht so exakt bestimmt wird, allerdings überdeckt er den relevanten Bereich ausreichend genau. Dargestellt sind Koordinaten, X- und Y-Position in Pixel sowie die Ausdehnung der Box (Width und Hight) ebenfalls in Pixel relativ zur umschließenden Box der Landmarks. Somit liegen die Landmarks der Augen im Bildausschnitt, wodurch diese Ausschnitt verwendet werden kann als Eingabe von ElSe.
\begin{figure}
	\centering
	\includegraphics[width=0.45\linewidth]{Eye_Img_Box/Openface_PC}
	\includegraphics[width=0.45\linewidth]{Eye_Img_Box/Openface_PW}
	%\includegraphics[width=0.45\linewidth]{Eye_Img/OpenFace_Zentrum_P}
	%\includegraphics[width=0.45\linewidth]{Eye_Img/OpenFace_Width_P}
	\caption{Auswirkung von Skalierung auf die Qualität der Augendetektion von ObenFace}
	\label{OpenFace_Eye}
\end{figure}
\begin{figure}
	\centering
	\includegraphics[width=0.245\linewidth]{Eye_Img_Box/Openface_BoxX}
	\includegraphics[width=0.245\linewidth]{Eye_Img_Box/Openface_BoxY}
	\includegraphics[width=0.245\linewidth]{Eye_Img_Box/Openface_BoxW}
	\includegraphics[width=0.245\linewidth]{Eye_Img_Box/Openface_BoxH}
	%\includegraphics[width=0.245\linewidth]{Eye_Img/Box_X}
	%\includegraphics[width=0.245\linewidth]{Eye_Img/Box_Y}
	%\includegraphics[width=0.245\linewidth]{Eye_Img/Box_W}
	%\includegraphics[width=0.245\linewidth]{Eye_Img/Box_H}
	\caption{Bestimmung der Box ums Auge}
	\label{OpenFace_Eye_Box}
\end{figure}
\subsection{Ergebnis}
So ist im Test der Durchschnitt bei allen Skalierungen ElSe den Ergebnisse von OpenFace überlegen, durch die Verteilung ist allerdings eine Kombination beider Verfahren sinnvoll, so kann das Ergebnis von OpenFace bei Bilder in denen die Iris größer als 21 Pixel ist direkt als Lösung verwendet werden, da der mögliche Fehler von OpenFace geringer ist als von ElSe.\\
Im Bereich zwischen 21 und 15 Pixel können beide Ergebnisse Kombiniert werden, da sie ungefähr gleich gute Ergebnisse liefern.\\
Sollte die Iris im Originalbild noch kleiner sein, so ist ElSe deutlich genauer, da es noch bis zu einer Irisgröße von 3 Pixel noch stabil funktioniert.

\section{Bestimmung einer Position auf der die Aufmerksamkeit liegt}
Ist bekannt wohin die Personen alle Blicken, kann dies aus ihrer Blickrichtung bestimmt werden.
\subsection{Schnittpunkt berechnen}
\label{calc_Center}
Verwende Blickrichtung mit Linie $L_i = s \cdot n_i+ p_i$ mit $s\in \mathbb{R}$ und $n_i,p_i \in \mathbb{R}^3$
\begin{align*}
c&=(\sum_{i} I -n_in_i^T)^{-1}
(\sum_{i} (I -n_in_i^T)p_i)
\end{align*}
\subsection{Mittelwert}
Ermittle aus den bestimmten Winkel die Blickrichtung mit $O= R_{a,b,c}\cdot (0,0,-1)^T$ und deren Durchschnitt $O_{avg}$ und die Durchschnittliche Position $P_{avg}$ der Personen. Bei der Bestimmung der Tiefe muss nun geschätzt werden. Wenn das Ziel die Tafel ist und die Kamera an der Tafel platziert wurde, kann die Tiefe im Bild verwendet werden um somit die Position der Tafel zu ermitteln.\\
Dies muss angewandt werden, wenn die Blickrichtungen zu sehr parallel verlaufen oder so verrauscht sind um ein sinnvolles Ergebnis mit der Bestimmung des Schnittpunkt \autoref{calc_Center} zu erhalten.

\chapter{Implementierung}
\section{Ablauf der Implementierung}
\begin{center}
\begin {tikzpicture}
	\node[circle,draw,align=center] (C) at(0,0) {Kamera};
	\node[draw,align=center] (F) at(0,-2.5)  {Gesichtserkennung\\\autoref{MTCNN}};
	\node[draw,align=center] (S) at(0,-4.5)  {Skalierung\\\autoref{scale_Algos}};
	\node[draw,align=center] (G) at(6,-6.5)  {Graubild\\\autoref{Graubild}};
	\node[draw,align=center] (A) at(0,-6.5)  {Gesichtsanalyse\\\autoref{OpenFace}};
	\node[draw,align=center] (E) at(6,-8.5)  {ElSe\\\autoref{ElSe}};
	
	\node (outA) at(0,-11.3)  {};
	\node (outB) at(6,-11.3)  {};
	
	\draw[->] (E)to node[right,align=center]{Blickrichtung\\\autoref{OpenFace_Blickrichtung} }(outB);
	\draw[->] (A)to[out=-45,in=90] node[right]{}(outB);
	
	\draw[->] (C)to node[right]{Frame}(F);
	\draw[->] (F)to node[right]{Bildausschnitt}(S);
	\draw[->] (S)to node[right]{Eingabebild}(A);
	\draw[->] (A)to node[left,align=center]{Gesichtsorientierung\\\autoref{OpenFace_Pos_Ori} }(outA);
	
	\draw[->] (A)to node[above]{Augenbereich}(G);
	\draw[->] (C)to[out=-45,in=90] node[left]{}(G);
	\draw[->] (G)to node[right]{Eingabebild}(E);
	
\end{tikzpicture}
\end{center}
Da nur eine einzige fest montierte Kamera ohne Zoom eingesetzt wird, muss diese eine entsprechend hohe Auflösung besitzen, damit alle Personen zu erkennen sind. Zur Bestimmung der Blickrichtung sowie Kopfposition und Orientierung wird ein mehrstufiges Verfahren eingesetzt um alle Teilprobleme zu lösen.\\
Am Anfang müssen alle Gesichter, die im aktuellen Frame vorhanden sind, detektiert werden da nur auf diesen eine Berechnung ausgeführt wird.
Dabei machen die relevanten Bereiche nur einen sehr geringen Anteil des gesamten Bildes aus. Dazu wird die MTCNN Face Detection eingesetzt, siehe \autoref{MTCNN}. Dieses Verfahren machte im Vorabtests auf Probebildern einen sehr guten Eindruck und konnte die meisten Gesichtern mit verschieden Größen und Blickrichtungen finden.\\
Für die weiteren Berechnungen muss bekannt sein, welcher Bereich von einem Gesicht im Frame eingenommen wird, um die relevanten Bildausschnitte aufzubereiten. Dabei muss das gesamte Gesicht in der Box sein, weitere Besonderheiten gibt es nicht, da OpenFace einen eigenen Facedetector besitzt. Je nach verwendetem Trainingsdatensatz und darin enthaltener Annotation werden z.B. Kinn und Haaransatz noch als Gesichtsbereich oder schon als außerhalb betrachtet. So geben beiden Methoden (OpenFace und MTCNN-Face) Boxen aus, diese sind in ihren Ausmaßen allerdings nicht identisch. Da die folgende Verarbeitung eine OpenFace-skalierte Box erwartet, hat sich eine Vergrößerung der MTCNN-Face Box um $30\%$ als sinnvoll erwiesen, um Ungenauigkeiten bezüglich der Position und Dimension des Kopfes im Bild entgegen zu wirken.\\
Sind mehrere Gesichter in mehreren Frames des Videos abgebildet, so muss auch eine Identitätszuordnung vorgenommen werden, damit bekannt ist welches Gesicht in Bild 1 welchem in Bild 2 entspricht. Für die Zuordnung reicht es meist aus, jene Box zu wählen, die am ehesten den selben Bildausschnitt repräsentiert wie im vorigen Frame, da sich die Gesichter meist weder groß Bewegen noch sich die einzelnen Boxen der Probanden überlappen.\\
Damit sicher auf allen Gesichter gerechnet werden kann, ist eine semiautomatische Korrektur erforderlich um Falsch-Detektionen zu entfernen und fehlende Boxen der Gesichtern ergänzen zu können. Die gefundenen 5 Landmarks von MTCNN-Face Detection sind für die nachfolgende Berechnung nicht relevant, da sie gerade bei kleinen Gesichtern zu ungenau sind um sie zu verwenden. Daher können alle bisher unternommenen Schritte auch von anderen Verfahren übernommen werden, da es sich hierbei nur um ein Vorverarbeitungsschritt handelt und zur Beschleunigung sowie Stabilität des späteren Berechnung beitragen soll.\\
Damit das Verfahren im nächsten Schritt zuverlässig arbeiten kann, werden alle zu kleinen Bildbereiche hochskaliert, um die Gesichter auf eine Mindestgröße zu bringen, siehe \autoref{scale_Algos}\\
Diese Bildbereiche werden nun von OpenFace weiterverarbeitet um die Landmarks, die signifikanten Punkte eines Gesichtes, zu bestimmen. Durch die vorige Zuordnung der Gesichert kann das Verfahren gezielt auf den einzelnen Person arbeiten und ein entsprechend eingestelltes CLNF verwenden, um bessere Ergebnisse zu erzielen, siehe \autoref{OpenFace}. Außerdem könne alle gefundenen Personen gleichzeitig (parallel) ausgewertet werden. Für dem im nächsten Schritt verwendetem ElSe Algorithmus, muss der Bildausschnitt des Auges in ein Graubild umgewandelt werden, siehe \autoref{Graubild}.\\
Um die Position der Pupille noch exakter zu ermitteln wird ElSe verwendet, da durch eine exakte Bestimmung der Pupillenposition, auch eine genaue Blickrichtungsbestimmung möglich ist, siehe \autoref{ElSe}.\\
Nun wird auf Basis der Landmarks und Kameraparameter die Position und Orientierung der Gesichter sowie die Blickrichtung bestimmt, siehe \autoref{calc_Position}. Diese Ergebnisse können dann von weiteren Anwendungen verwendet werden.
\section{Gesichtserkennung}
\label{detection_Gesicht}
Da nur eine einzige fest montierte Kamera ohne Zoom eingesetzt wird, muss diese eine entsprechend hohe Auflösung besitzen damit alle Personen zu erkennen sind. Allerdings machen die eigentlichen Bereiche der Gesichter nur einen sehr geringen Anteil des gesamten Bildes aus und diese relevanten Bildausschnitte müssen für die spätere Anwendung noch aufbereitet werden, siehe \autoref{skalierung}\\
Für die automatische Detektion wird MTCNN-Face eingesetzt, da dieses Verfahren im Vorabtests auf Probebildern einen sehr guten Eindruck gemacht hat und die meisten Gesichtern mit verschieden Größen und Blickrichtungen finden konnte. Sogar recht kleine mit $20\times 20$ Pixeln soll laut Beschreibung des Verfahrens \autoref{MTCNN} möglich sein. Bei diesem Schritt müssen alle Gesichert gefunden werden, auf denen die Berechnung stattfinden soll. Dabei muss das gesamte Gesicht in der Box sein, weitere Besonderheiten gibt es nicht, da OpenFace einen eigenen Facedetector besitzt.\\
Die von den beiden Methoden (OpenFace und MTCNN-Face) ausgegebenen Boxen sind allerdings in ihren Ausmaßen nicht identisch. Je nach verwendetem Trainingsdatensatz und darin enthaltener Annotation werden z.B. Kinn und Haaransatz noch als Gesichtsbereich oder schon als außerhalb betrachtet. Da die folgende Verarbeitung eine OpenFace-skalierte Box erwartet, hat sich eine Vergrößerung der Box um $30\%$ als sinnvoll erwiesen bei Verwendung des MTCNN-Face Detektors.\\
Ebenfalls in diesem Schritt werden die einzelnen Boxen den Personen zugeordnet, damit im späteren Verlauf das korrekte CLNF für die Person verwendet werden kann. Für die Zuordnung reicht es meist aus, jene Box zu wählen die am ehesten den selben Bereich wie im vorigen Frame einnimmt. Dabei wird einfach für jede Box im neuen Frame die Box Im vorigen Frame gesucht die den selben Bildausschnitt repräsentiert. Dies ist ausreichend, da die Gesichter sich meist weder groß Bewegen noch sich die einzelnen Boxen der anderen überlappen.\\
Damit sicher auf allen Gesichter gerechnet werden kann, ist eine semiautomatische Korrektur erforderlich um Falsch-Detektionen zu entfernen und fehlende Boxen der Gesichtern ergänzen zu können.\\
Die gefundenen 5 Landmarks von MTCNN-Face Detection sind für die nachfolgende Berechnung nicht relevant, da sie gerade bei kleinen Gesichtern zu ungenau sind. Daher kann dieser Bereich auch von anderen Verfahren übernommen werden, da es sich hierbei nur um ein Vorverarbeitungsschritt handelt und zur Beschleunigung sowie Stabilität des späteren Berechnung beitragen soll.

\section{Skalierung auf Mindestgröße}
\label{skalierung}
Da OpenFace optimiert ist auf Gesichtern von mindestens 100 Pixel zu arbeiten, werden die Bildbereiche auf diese Größe hochskaliert. \autoref{scale_Algos}\\
Dies erhöht den Informationsgehalt der Bilder nicht, sie sind nur besser nutzbar, da sie dem Trainingsdatensatz stärker ähneln.
Die von MTCNN gelieferten und vergrößerten Boxen werden nun auf $130 \times 180$ Pixel gebracht um Ungenauigkeiten bezüglich der Position und Dimension des Kopfes im Bild entgegen zu wirken. Neben der Skalierung des Bildausschnittes, muss bekannt sein, wie Punkte im skalierten Bildausschnitt in das Frame überführt werden kann, damit dies bei späteren Berechnungen berücksichtigt wird.\\
Die Skalierung ist für jeden Bildausschnitt individuell und kann sich durchaus über die Zeit ändern, wenn sich z.B. die Distanz zwischen Person und Kamera verändert.\\
Von einer zu starken Vergrößerung ist abzuraten, da sich der Rechenaufwand pro Gesicht erhöht und die Zuverlässigkeit der Berechnungen von OpenFace sinkt, z.B. durch Falschdetektion.
\section{Bestimmung der Landmarks}
\label{bestimmung_Landmarks}
Für die Bestimmung der Landmarks wird OpenFace eingesetzt. Dabei wird jeder Bildausschnitt unabhängig der anderen Betrachtet und da bekannt ist, um welche Person es ich im Bild handelt, kann direkt mit dem jeweiligen CNN gearbeitet werden, das auf diese Person optimiert wurde.\\
Durch die vorige Selektion wird nur auf jenen Bildausschnitten gerechnet auf denen auch die Person zu sehen ist, wodurch nicht unnötig gesucht werden muss und auch ein Lernen auf Personen stattfinden kann die nur selten zu sehen sind, da sie nur resettet werden, wenn sie eigentlich zu sehen sein müssten aber nicht detektiert wurden.\\
Für die eigentliche Bestimmung der Landmarks bietet OpenFace zwei verschiedene Methoden, die Berechnung auf Bildern und Videos. Der Hauptunterschied ist das Lernen, dass bei der Videoauswertung verwendet wird, wodurch sich die Bereiche, auf denen Ergebnisse geliefert werden, deutlich erhöht.\\
Dies ist interessant für die spätere Anwendung, da somit auch Einzelbilder verwendet werden können, die eine deutlich höhere Auflösung haben als ein Video. Allerdings sinkt dann der maximale Winkel relativ zur Kamera beträchtlich, zu Gunsten der maximalen Distanz. Außerdem können schon kleinste Farbänderungen im Bild beim Hochskallieren ausschlaggebend sein, ob ein Gesicht erkannt werden kann, wodurch bei gleicher Bildqualität Gesichter im Video besser erkannt werden.\\
Da die gesamte Berechnung auf Grau-Bildern basiert ist auch eine Farbkorrektur, wie Verbesserung des Kontrast, Farbverlauf usw. möglich, um etwaige Einflüsse bei der Aufnahme zu korrigieren.\\
Dennoch kann es passieren, dass trotz allem  ein Gesicht falsch detektiert wird, wie z.B. das erkennen eines Gesichtes in der Ohrmuschel, diese müssen entsprechend behandelt werden, da ansonsten das Lernen auf diese Bereiche stattfindend und im nächsten Frame erneut nach diesen Merkmalen gesucht wird.

\begin{itemize}
\item Verbesserung durch Farbkorrektur
\end{itemize}
\section{Verbesserung der Augen}
\label{verbesserung_ElSe}
Zusätzlich zu den 64 Landmarks, die ein Gesicht beschreiben, kann von OpenFace weitere 28 Landmarks für ein Auge bestimmt werden, aus denen dann die Blickrichtung ermittelt wird.\\
Um die Position der Landmarks zu verbessern, kann auf dem Bildausschnitt der Augen der ElSe-Algotithmus eingesetzt werden. Dieser Algorithmus arbeitet auf einem Farbbild um so die Umrisse der Pupille zu berechnen.\\
Da unter den 28 Landmarks die Umrisse von Pupille und Iris beschreiben wird, müssen diese aus dem Ergebnis von ElSe abgeleitet werden. Dabei hat sich eine Veränderung des Radius mit ?? für Pupille und ?? für die Iris bewährt.\\
Allerdings muss das Auge für die Berechnung aus entsprechend vielen Pixeln bestehen, wodurch es im Originalbild mindestens mit 10 Pixeln dargestellt wird, um sinnvolle Ergebnisse zu erhalten. Da diese Berechnung unabhängig der Landmarks ausgeführt wird, empfiehlt sich das Ergebnis zu überprüfen, damit die bestimmte Ellipse auch innerhalb der Augenhöhle liegt.\\
Dabei wird jedes Auge unabhängig vom anderen betrachtet, wodurch sich verschiedene Blickrichtung ergeben. Ab einer Distanz von mehr als ??cm kann die Blickrichtung beider Augen als parallel angesehen und kann entsprechend behandelt werden. Eine Verbesserung ergibt sich, wenn beide Augen anhängig von einander bestimmt werden, damit sich der Fehler minimiert.
\subsection{Farb- nach Grau-Bild - To Do}
Bilder einfügen für jedes Verfahren\\\\
Da die Berechnungen von ElSe auf Grau-Bildern arbeitet, das Eingabebild in Farbe ist, muss es in ein Grau-Bild umgewandelt werden. Dabei soll vor allem der Farbunterschied zwischen Pupille und der Umgebung maximal sein.\\
Um die Auswirkung der verschiedenen Farb- nach Grau-Bilder Konverter zu  ermitteln wurden einige Verfahren verwendet um ihre Auswirkung auf die Detektion zu ermitteln. Nach der Umwandlung von Farb- nach Grauwert wird für die Anwendung das Graubild noch normiert, damit Mindestens ein schwarzes und ein weißes Pixel vorhanden ist.
\subsubsection{Luminance}
Dies ist ein lineares Verfahren, das der menschlichen Farbwahrnehmung entspricht. Eine Gammakorrektur wird bei der Umwandlung nicht verwendet.
\[G_{Luminance} = 0.299 \cdot R + 0.587 \cdot G + 0.114 \cdot B\]
\subsubsection{Gleam}
Bei dem Gleam Verfahren wird jede Farbe (Rot,Gelb und Grün) gleich stark bewertet allerdings wird jeder Farbwert mittels einer Gamma-Korrektur verbessert.
\[G_{Gleam}=\dfrac{R^{\frac{1}{2.2}} + G^{\frac{1}{2.2}} + B^{\frac{1}{2.2}}}{3}\]
\subsection{Gleam New}
Dies ist eine verbesserte Variante von Gleam bei dem zuerst das gesamte Bild Analysiert wird um die Parameter für die jeweilige Gamme-Korrektur zu ermitteln.
\[G_{Gleam New}=\dfrac{R^{r} + G^{g} + B^{bx}}{3}\]
Wobei gilt $\{r,g,b\} = \frac{\log(V_{\max})}{\log(\{R,G,B\}_{\max})}$ mit $V_{\max}$ als maximal möglicher Farbwert und $R_{\max}$ als maximal Vorhandener Rot-Farbwert, $G_{\max}$ und $B_{\max}$ äquivalent.
\subsection{Auswirkung der verschiedenen Verfahren - To Do}
Grafiken neu machen\\\\
Um die einzelnen Verfahren besser Vergleichen zu können wurden Künstliche Augen aus dem Datensatz \cite{database_Eye} verwendet, da die Exakte Position der einzelnen Landmarks bekannt sind.
Da auch in der späteren Anwendung der Augenbereich genauer bestimmt ist, bevor ElSe zum Einsatz kommt wurde, nur der Bildbereich in dem alle Landmarks der Augenlider liegen, somit sind die Bilder etwa 64 auf 29 Pixel groß. Um die Qualität der Berechnung bei verschiedenen Größen zu simulieren, wurde das Bild um den angegebenen Faktor verkleinert.\\
\begin{figure}
	\centering
	\includegraphics[width=0.45\linewidth]{ElSe_Img/ElSe_22G_Pupile_Zentrum}
	\includegraphics[width=0.45\linewidth]{ElSe_Img/ElSe_Gray_15_Pupile_Zentrum}
	\includegraphics[width=0.45\linewidth]{ElSe_Img/ElSe_Norm_15_Pupile_Zentrum}
	\caption{Abstand des Zentrums der Landmark-Pupille und der Berechneten Ellipse in [Pixel/Skalierung] Oben-Links: Gleam, Oben-Rechts: Gleam New, Unten-Links: Luminance}
	\label{ElSe_Gray_Zentrum}
\end{figure}
\begin{figure}
	\centering
	\includegraphics[width=0.45\linewidth]{ElSe_Img/ElSe_22G_Iris_Width}
	\includegraphics[width=0.45\linewidth]{ElSe_Img/ElSe_Gray_15_Iris_Width}
	\includegraphics[width=0.45\linewidth]{ElSe_Img/ElSe_Norm_15_Iris_Width}
	\caption{Unterschied Zwischen den Radien der Landmark-Iris und der Berechneten Ellipse in [Pixel/Skalierung] Oben-Links: Gleam, Oben-Rechts: Gleam New, Unten-Links: Luminance}
	\label{ElSe_Gray_Width}
\end{figure}
Es Zeigt sich, dass das Verfahren um den Farbwert in einen Grauwert zu überführen keinen signifikanten Unterschied existiert. Die Differenz zwischen dem besten und schlechtesten Verfahren liegen bei hundertstel Pixel und bei seiner Verkleinerung auf 0.02 auf 1 Pixel unterschied, dies ist so gering, dass das Verfahren Offensichtlich keine Auswirkung auf die eigentliche Berechnung hat.
\subsection{ElSe - Auswirkung des Radius - To Do}
Neu Berechnen ohne Vergrößerung\\\\
Ein weiter wichtiger Parameter des ElSe-Verfahrens ist der Radius des Filters. Wiederrum wurde der Augen-Datensatz \cite{database_Eye} verwendet und diesmal auf der Originalgröße gearbeitet.\\
Im Datensatz besitzen die abgebildeten Augen eine Durchschnittlich Pupille von 15 Pixel und eine Iris von 34 Pixel, wiederum auf dem Bildausschnitt gerechnet.
\begin{figure}
	\centering
	\includegraphics[width=0.45\linewidth]{ElSe_Img/Gray_Alt_Abst}
	\includegraphics[width=0.45\linewidth]{ElSe_Img/Gray_Alt_Wid}
	\caption{Auswirkung bei der Veränderung des Radius des ElSe-Algorithmus}
	\label{ElSE_Radius}
\end{figure}
Es ist zu erkennen, dass die Wahl des Radius signifikant ist für die Qualität der Berechnung. Im Versuch hat sich ein Radius von etwa einem Zehntel des zu erwartetem Durchmesser der Iris als sinnvoll erwiesen, um möglichst genaue Ergebnisse zu erhalten. Dabei ist zu erwähnen das eine Differenzierung zwischen Iris und Pupille meist nicht möglich ist, da der Grauwert meist recht gering ist und deutlich weniger als der Grauwert vom Rest des Auges.\\
Die Korrekte Größe der Pupille bzw. Iris wird nur mäßig gut ermittelt, da meist eine Zwischenlösung beider berechnet wird und danke der Ruflektionen und Augenfarben nicht klar getrennt werden. Betrachtet man den ermittelten Abstand, siehe \autoref{ElSe_Gray_Zentrum}, sieht man, dass dieser recht konstant bleibt, somit ist dieses Verfahren recht stabil gegenüber Skalierung und wird selbst bei verschiedenen Radien noch gut ermittelt.
\subsection{OpenFace - Auge To Do}
Als Referenz wir das Ergebnis von OpenFace für die zusätzlich bestimmten Landmarks der Augen verwendet. Dies wurde auch auf dem Augendatensatz \cite{database_Eye} angewendet um vergleichbare Ergebnisse zu erhalten.
\begin{figure}
	\centering
	\includegraphics[width=0.45\linewidth]{ElSe_Img/OpenFace_Pupile_Abstand}
	\includegraphics[width=0.45\linewidth]{ElSe_Img/OpenFace_Pupile_Width}
	\caption{}
	\label{OpenFace_Eye}
\end{figure}
Es ist zu erkennen dass dieses Verfahren oft schlechtere Ergebnisse liefert als das Ergebnis von ElSe, allerdings ohne das begehen von großen Fehlern und auch öfters genauere Ergebnisse.\\
\begin{itemize}
	\item Grafik neu
	\item Vergleich zu ElSe
\end{itemize}


\chapter{Ergebnisse}
\section{Erreichte Werte}
\subsection{Auswirkung der Größe}
Durch den Aufbau, muss das Verfahren zuverlässig bezüglich der Größe sein, zur Messung wurde der Datensatz von Labeled Faces in the Wild \cite{database_Face} verwendet. In diesem Datensatz ergibt sich im Originalbild eine durchschnittliche Kopfbreite von 94 Pixel. Bei Random Forests for Real Time 3D Face Analysis \cite{database_Face_Ori} ist die durchschnittliche Breite 78 Pixel. Zur Beschleunigung wurde OpenFace zu erst auf das gesamte Bild eingesetzt um die möglichen Gesichter zu finden, in jeder Skalierungsstufe wurde nur der Gesichtsbereich, mit Toleranz, betrachtet\\
Zur Durchführung wurden die Größe der Bilder mit dem Faktor multipliziert um so kleinere Gesichter zu erhalten und anschließend mit dem Image-Detector von OpenFace zu detektieren, siehe \autoref{img_lineareverkleinerung}.\\
\begin{figure}
	\centering
	\includegraphics[width=0.45\linewidth]{img/lineare_Verkleinerung}
	\includegraphics[width=0.45\linewidth]{img/lineare_Verkleinerung2}
	\caption{Die Bilder aus Labeled Faces in the Wild \cite{database_Face} (links) und Random Forests \cite{database_Face_Ori} wurden mit den Faktor auf der X-Achse linear verkleinert und die Erkennungsrate Y-Achse abgebildet}
	\label{img_lineareverkleinerung}
\end{figure}
Es ist zu erkennen, dass die Wahrscheinlichkeit auf eine erfolgreiche Detektion ab $0.5$, also etwa Gesichert mit 47 Pixel Breite, rapide abnimmt. Bei der verwendeten Kamera \autoref{hardware} entspricht dies einer Distanz von etwa $4.5m$.\\
Bei der maximalen Distanz auf der gearbeitet werden soll $(8.5m)$ ergibt sich eine Gesichtsgröße von etwa 22 Pixel, das einer Skalierung von 0.25 entspricht. Bei dieser Bildgröße ist keine Detektion möglich, siehe \autoref{img_lineareverkleinerung}.
\subsection{verschiedenen Skalierungesverfahren}
Um auf den gewünschten Distanzen arbeiten zu können, wird der jeweilige Bereich Hochskaliert. Dazu wird das Ursprüngliche Bild $(250\times 250)$ linear um den angegebene Faktor verkleinert und anschließend mit den angegebenen Verfahren auf $300\times 300$ wieder vergrößert. Die Wahrscheinlichkeit auf eine Detektion ist in \autoref{img_hochskalliern} abgebildet.\\
\begin{figure}
	\centering
	\includegraphics[width=0.5\linewidth]{img/Hochskalliern}
	\caption{Die Bilder aus Labeled Faces in the Wild \cite{database_Face} wurden mit den Faktor auf der X-Achse linear verkleinert und mit den verschiedenen Verfahren wieder vergrößert \autoref{scale_Algos}. Aufgetragen gegen die Detektionswahrscheinlichkeit.
	Nearest-Neighbor (blau), Linear (rot), Bicubic (braun), Lanczos (grün)}
	\label{img_hochskalliern}
\end{figure}
Es ist zu erkennen das durch die Vergrößerung, Gesichter in Bereichen die normal nicht erkennbar sind, bestimmbar werden. Als das ungeeignetste Verfahren hat sich Nearest-Neighbor herausgestellt, siehe blaue Linie \autoref{img_hochskalliern}. Die anderen haben sehr ähnliche Ergebnisse, nur das Lineare Verfahren ist etwas schlechter. Dennoch werden die Anforderungen, einer Detektion auf Gesichtern von 22 Pixel (Skalierung 0.25) von allen erfüllt.\\
Ausgehend vom Skalierungsfaktor des Linearen-, Bicubic- und Lanczos-Verfahren wären mit der verwendeten Kamera auch Distanzen bis zu $14m$ möglich. Allerdings ist das Bild durch die Verkeilung  deutlich besser als Originalaufnahmen, da Pixelrauschen nicht vorhanden ist.
\subsection{Pixelrauschen bei den Skalierungesverfahren}
Um Pixelrauchen zu simulieren, wurden die Bilder aus Labeled Faces in the Wild \cite{database_Face} entsprechend verkleinert und dann mit Rauschen versehen um sie anschließend mit den verschiedenen Verfahren zu vergrößern.\\
Somit soll geprüft werden, welches der Verfahren auch stabil gegen Rauschen ist.
\begin{figure}
	\centering
	\includegraphics[width=0.7\linewidth]{img/Hochskalliern_Nois}
	\caption{Bilder aus Labeled Faces in the Wild \cite{database_Face}, mit dem X-Faktor verkleinert, um jedes Pixel mit $50\%$ Wahrscheinlichkeit auf $\pm 10\%$ Gleichverteilung der Abweichung}
	\label{img_hochskalliern_nois}
\end{figure}
Das Rauschen wird für jedes Pixel mit einer Wahrscheinlichkeit von $50\%$ auf eine gleich verteilte Abweichung von $\pm 10\%$ des Originalen Farbwertes simuliert. Anschließend wird das verrausche Bild mit den verschiedenen Verfahren vergrößert. Dieser Vorgag wurde für jedes Bild vier mal wiederholt um Zufälligkeiten zu vermeiden.\\
Wie zu erwarten ist Nearest-Neighbor am schlechtesten, aber auch zwischen den anderen Verfahren sind nun unterscheiden zu erkennen, die gesamte Erkrankungsrate ist signifikant kleiner als ohne Rauschen, wobei die Position $(0.15)$ ab der die Erkennungsrate rapide abfällt beibehalten wird.
\subsection{Auswirkung von Pixelrauschen}
Durch Aufnahme eines Schwarzbildes der Actioncam zeigt sich, dass das Pixelrauschen recht hoch ist, siehe \autoref{img_noishight}. Das Rauschen hat keine Normalverteilung, sondern es besteht aus kleinen Bereiche, die den selben fehlerhaften Farbwert besitzen.
\begin{figure}
	\centering
	\fbox{\includegraphics[width=1\linewidth]{img/NoisHight}}
	\caption{Aufnahme eines Schwarz-Bildes $(2688\times 1520)$ der Actioncam um den Faktor 7 verstärkt und invertiert.}
	\label{img_noishight}
\end{figure}
\subsection{Größe und Genauigkeit}
Um die Qualität auf verschiedenen Distanzen zu ermitteln, wurde der Datensatz Forests for Real Time 3D Face Analysis \cite{database_Face_Ori} verwendet, da für jedes Gesicht sein Position und Orientierung bekannt ist. Um die verschiedenen Ditanzen würden die Bilder mit dem angegebene Faktor (X-Achse) verkleinert und mit dem Original verglichen.\\
Da verschiedene Verfahren angeboten werden zur Bestimmung der Position und Orientierung, werden diese miteinander verglichen, siehe \autoref{img_X_Pos}. Zur Bestimmung wurde nur das RGB-Bild verwendet und nich zusätzlich die Tiefeinaufnahme, da dies in der Anwendung auch nicht vorhanden sind.\\
Es zeigt sich, dass Pose World, also die einfache Bestimmung der Position mittels Skalierungsfaktor und zusätzlicher Korrektur der Wikel die besten Ergebnisse liefert.\\
Die Bestimmung mittels der Überführung von 3D zu 2D Punkten ist nicht notwendig,da ein schlechteres Ergebnis erzieht wurde.
\subsubsection{Position}
Zur Bestimmung der Position gibt es zwei Verfahren, die direkte mittels Brennweite und Skalierung oder die Überführungsmatrix von den 3D und 2D Landmarks.\\
\begin{figure}
	\centering
	\includegraphics[width=0.45\linewidth]{tabelle/X_Pos_PC}
	\includegraphics[width=0.45\linewidth]{tabelle/X_Pos_PW}
	\includegraphics[width=0.45\linewidth]{tabelle/X_Pos_CPC}
	\includegraphics[width=0.45\linewidth]{tabelle/X_Pos_CPW}
	\caption{Pose World (links oben), Pose World (rechts oben), Correct Pose Camera (links unten) und Coorect Pose World, der Abstand (Y-Achse) ist in Millimeter.}
	\label{img_X_Pos}
\end{figure}
Die Funktionen Pose Camera und Pose World (Obere in \autoref{img_X_Pos}) verwenden die einfache Bestimmung mittels Skalierung. Dargestellt ist nur die X-Werte, da die Y-Werte eine recht ähnliche Verteilung liefern.\\
Bei den Z-Werten ergibt sich ein etwas anderer Verlauf, bei dem allerdings sie Fehlerquote bei kleinen Bildern gut sichtbar wird, siehe \autoref{img_Z_Pos}.\\
\begin{figure}
	\centering
	\includegraphics[width=0.45\linewidth]{tabelle/Z_Pos_PC}
	\includegraphics[width=0.45\linewidth]{tabelle/Z_Pos_PW}
	\includegraphics[width=0.45\linewidth]{tabelle/Z_Pos_CPC}
	\includegraphics[width=0.45\linewidth]{tabelle/Z_Pos_CPW}
	\caption{Pose World (links oben), Pose World (rechts oben), Correct Pose Camera (links unten) und Coorect Pose World, der Abstand (Y-Achse) ist in Millimeter.}
	\label{img_Z_Pos}
\end{figure}
Zur Bewertung, die Durchschnittliche Distanz zwischen Kamera und Kopf beträgt ca $70cm$ bei einer Kopfbreite von 78 Pixel. Der schnelle Abfall der Genauigkeit ist an der selben stelle (0.5) an der auch die Detektionsrate stark absinkt.
\subsubsection{Orientierung}
Auch bei der Orientierung werden die verscheiden Methoden miteinander verglichen. Die Analyse hat gezeigt, dass die Qualität der Verfahren von den einzelnen Rotationen abhängt.\\
\begin{figure}
	\centering
	\includegraphics[width=0.45\linewidth]{tabelle/X_Rot_PC}
	\includegraphics[width=0.45\linewidth]{tabelle/X_Rot_PW}
	\includegraphics[width=0.45\linewidth]{tabelle/X_Rot_CPC}
	\includegraphics[width=0.45\linewidth]{tabelle/X_Rot_CPW}
	\caption{Pose World (links oben), Pose World (rechts oben), Correct Pose Camera (links unten) und Coorect Pose World, der Abstand (Y-Achse) ist im Bogenmaß.}
	\label{img_X_Pot}
\end{figure}
Bei der X-Rotation, dargestellt in \autoref{img_X_Pot} können die rechten Verfahrenen (Pos World und Correct Pose World) überzeugen. Vor alle, Pose World hat selbst bei kleinen Abbildungen nur eine mittlere Abweichung von $8.5^\circ$\\
\begin{figure}
	\centering
	\includegraphics[width=0.45\linewidth]{tabelle/Y_Rot_PC}
	\includegraphics[width=0.45\linewidth]{tabelle/Y_Rot_PW}
	\includegraphics[width=0.45\linewidth]{tabelle/Y_Rot_CPC}
	\includegraphics[width=0.45\linewidth]{tabelle/Y_Rot_CPW}
	\caption{Pose World (links oben), Pose World (rechts oben), Correct Pose Camera (links unten) und Coorect Pose World, der Abstand (Y-Achse) ist m Bogenmaß.}
	\label{img_Y_Pot}
\end{figure}
Um die Y-Rotation zu ermitteln ist nun allerdings die linken (Pose Came und Correct Pose Came) den rechten (Posw Worls und Correcht Pose World) deutlich überlegen, siehe \autoref{img_Y_Pot}. Auch hier liegt der mittlere Fehler über lange Zeit bei etwa $9^\circ$\\
\begin{figure}
	\centering
	\includegraphics[width=0.45\linewidth]{tabelle/Z_Rot_PC}
	\includegraphics[width=0.45\linewidth]{tabelle/Z_Rot_PW}
	\includegraphics[width=0.45\linewidth]{tabelle/Z_Rot_CPC}
	\includegraphics[width=0.45\linewidth]{tabelle/Z_Rot_CPW}
	\caption{Pose World (links oben), Pose World (rechts oben), Correct Pose Camera (links unten) und Coorect Pose World, der Abstand (Y-Achse) ist im Bogenmaß.}
	\label{img_Z_Pot}
\end{figure}
Bei der Bestimmung von der Z-Rotation sind die Correct Pose Came und Pose Came nahe zu gleich gut, Correkt Pose World allerding schlechter und Pose World besser, siehe \autoref{img_Z_Pot}. Wobei auffällt, dass Pose World bei Werten unter 0.4 plötzlich sehr schlecht wird.
\subsubsection{Wertebereich Rotation}
Von Interesse sind auch die Winkel, bei den Gesichter in verschiedenen Skalierungen noch erkannt werden, siehe \autoref{img_Rot_Value}.\\
Hier ist zu erkennend das der Wertebereich ab 0.7 abnimmt und ab 0.5 recht schnell. Dies ist wichtig zu wissen, da wenn kein gesicht in diesen Bereichen nicht erkannt werden kann auch die späteren Verfahren nicht bestimmt werden können.\\
Der Wertebereich auf den einzelnen Achsen sollte ist ausreichend sein für die Anwendung, auch wenn die Rotation pralle zur Horizontalen etwas größer sein könnte.
\begin{figure}
	\centering
	\includegraphics[width=0.3\linewidth]{tabelle/X_Rot}
	\includegraphics[width=0.3\linewidth]{tabelle/Y_Rot}
	\includegraphics[width=0.3\linewidth]{tabelle/Z_Rot}
	\caption{Darstellung der noch detektierten Wertebereiche in Bogenmaß.}
	\label{img_Rot_Value}
\end{figure}
\subsection{Qualität der Skalierung}
Nun wird der Zusammenhang zwischen den verscheiden Skalierungsverfahren und der Qualität der Ergebnisse gesucht.\\
Es zeigt sich, dass bei der Bestimmung der Parameter ist das Nearest-Neighbor Verfahren am genauesten, allerdings sit der Wertebereich deutlich eingeschränkt, die Mindestgröße des Gesichts im Orginal und den geringer Wertebereich bei den Rotationen ist dieses Verfahren eher ungeeignet.\\
Bei dem Linearen Verfahren ist die Abweichung bei den Rotationen am größten, auch wenn es sich nur um etwa ein halbes Grad handelt. Zwischen dem Bicubic- und Lanczos-Verfahren gibt es in den relevanten Bereichen keinen signifikanten Unterschied, wobei das Lanczos in den kleineren Bereichen gleichmäßigere Ergebnisse, kann aber vom Rechenaufwand abhängig gemacht werden welches Verfahren gewählt werden soll. 
\subsubsection{Position}
Als erstes wird die berechnete Distanz miteinander verglichen.
\begin{figure}
	\centering
	\includegraphics[width=0.45\linewidth]{tabelle2/X_Pos_Cubic}
	\includegraphics[width=0.45\linewidth]{tabelle2/X_Pos_Lanc}
	\includegraphics[width=0.45\linewidth]{tabelle2/X_Pos_Linear}
	\includegraphics[width=0.45\linewidth]{tabelle2/X_Pos_NN}
	\caption{Zusammenhang zwischen der Skalierung (X-Achse) und der Abweichung in X-Richtung (Y-Achse) in Millimeter. 
	 Bicubic (oben links), Lanczos (oben rechts), Linear (unten links), Nearest-Neighbor (unten rechts)}
	\label{img_X_Pos_Skal}
\end{figure}
 In \autoref{img_X_Pos_Skal} ist die Abweichung entlang der X-Achse dargestellt. Nearest-Neighbor liefert die besten Ergebnise, auch wenn durch die schlechtere Detektionsrate dieses Verfahren früher ausfällt als die anderen drei.\\
\begin{figure}
	\centering
	\includegraphics[width=0.45\linewidth]{tabelle2/Y_Pos_Cubic}
	\includegraphics[width=0.45\linewidth]{tabelle2/Y_Pos_Lanc}
	\includegraphics[width=0.45\linewidth]{tabelle2/Y_Pos_Linear}
	\includegraphics[width=0.45\linewidth]{tabelle2/Y_Pos_NN}
	\caption{Zusammenhang zwischen der Skalierung (X-Achse) und der Abweichung in Y-Richtung (Y-Achse) in Millimeter. 
		Bicubic (oben links), Lanczos (oben rechts), Linear (unten links), Nearest-Neighbor (unten rechts)}
	\label{img_Y_Pos_Skal}
\end{figure}
Auf der Y-Achse ist das Lineare-Verfahren etwas besser als die Andren, das Nearest-Neighbor ist hierbei überraschend das Schlechteste, siehe \autoref{img_Y_Pos_Skal}.\\
\begin{figure}
	\centering
	\includegraphics[width=0.45\linewidth]{tabelle2/Z_Pos_Cubic}
	\includegraphics[width=0.45\linewidth]{tabelle2/Z_Pos_Lanc}
	\includegraphics[width=0.45\linewidth]{tabelle2/Z_Pos_Linear}
	\includegraphics[width=0.45\linewidth]{tabelle2/Z_Pos_NN}
	\caption{Zusammenhang zwischen der Skalierung (X-Achse) und der Abweichung in Z-Richtung (Y-Achse) in Millimeter. 
		Bicubic (oben links), Lanczos (oben rechts), Linear (unten links), Nearest-Neighbor (unten rechts)}
	\label{img_Z_Pos_Skal}
\end{figure}
Nur schwer zu erkennen, da der Unterschied nur minimal ist, ist auch hier das  Nearest-Neighbor  genauer, siehe \autoref{img_Y_Pos_Skal}. Die anderen drei sind nahezu identisch. Bei sehr kleinen Skalierungen existieren durchaus auch sehr große Fehler, diese wurden allerdings bei der Darstellung abgeschnitten, da bei dieser Größe die Detektionsrate so klein ist, dass sie nahezu irrelevant werden.
\subsubsection{Orientierung}
Als weitere Bestimmung wird die berechneten Winkel um die jeweilige Achse.
\begin{figure}
	\centering
	\includegraphics[width=0.45\linewidth]{tabelle2/X_Rot_Cubic}
	\includegraphics[width=0.45\linewidth]{tabelle2/X_Rot_Lanc}
	\includegraphics[width=0.45\linewidth]{tabelle2/X_Rot_Linear}
	\includegraphics[width=0.45\linewidth]{tabelle2/X_Rot_NN}
	\caption{Zusammenhang zwischen der Skalierung (X-Achse) und der Abweichung des Winkels in X-Richtung, Angabe in Bogenmaß. 
		Bicubic (oben links), Lanczos (oben rechts), Linear (unten links), Nearest-Neighbor (unten rechts)}
	\label{img_X_Rot_Skal}
\end{figure}
\begin{figure}
	\centering
	\includegraphics[width=0.45\linewidth]{tabelle2/X_Rot_Val_Cubic}
	\includegraphics[width=0.45\linewidth]{tabelle2/X_Rot_Val_Lanc}
	\includegraphics[width=0.45\linewidth]{tabelle2/X_Rot_Val_Linear}
	\includegraphics[width=0.45\linewidth]{tabelle2/X_Rot_Val_NN}
	\caption{Zusammenhang zwischen der Skalierung (X-Achse) und der Abweichung des Winkels in X-Richtung, Angabe in Bogenmaß. 
		Bicubic (oben links), Lanczos (oben rechts), Linear (unten links), Nearest-Neighbor (unten rechts)}
	\label{img_X_Rot_Val_Skal}
\end{figure}
Geringste Abweichung bei der bestimung der X-Rotation bei Nearest-Neighbor, siehe \autoref{img_X_Rot_Skal}. Auffällig ist außerdem der kleinere Wertebereich des Linearen-Verfahrens.\\
\begin{figure}
	\centering
	\includegraphics[width=0.45\linewidth]{tabelle2/Y_Rot_Cubic}
	\includegraphics[width=0.45\linewidth]{tabelle2/Y_Rot_Lanc}
	\includegraphics[width=0.45\linewidth]{tabelle2/Y_Rot_Linear}
	\includegraphics[width=0.45\linewidth]{tabelle2/Y_Rot_NN}
	\caption{Zusammenhang zwischen der Skalierung (X-Achse) und der Abweichung des Winkels in Y-Richtung, Angabe in Bogenmaß.
		Bicubic (oben links), Lanczos (oben rechts), Linear (unten links), Nearest-Neighbor (unten rechts)}
	\label{img_Y_Rot_Skal}
\end{figure}
\begin{figure}
	\centering
	\includegraphics[width=0.45\linewidth]{tabelle2/Y_Rot_Val_Cubic}
	\includegraphics[width=0.45\linewidth]{tabelle2/Y_Rot_Val_Lanc}
	\includegraphics[width=0.45\linewidth]{tabelle2/Y_Rot_Val_Linear}
	\includegraphics[width=0.45\linewidth]{tabelle2/Y_Rot_Val_NN}
	\caption{Zusammenhang zwischen der Skalierung (X-Achse) und der Wertebereich des Winkels in Y-Richtung, Angabe in Bogenmaß.
		Bicubic (oben links), Lanczos (oben rechts), Linear (unten links), Nearest-Neighbor (unten rechts)}
	\label{img_Y_Rot_Val_Skal}
\end{figure}
Auch bei der Y-Rotation schneidet Nearest-Neighbor am besten ab, siehe \autoref{img_Y_Rot_Skal}, allerdings sind die unterscheide minimal.\\
\begin{figure}
	\centering
	\includegraphics[width=0.45\linewidth]{tabelle2/Z_Rot_Cubic}
	\includegraphics[width=0.45\linewidth]{tabelle2/Z_Rot_Lanc}
	\includegraphics[width=0.45\linewidth]{tabelle2/Z_Rot_Linear}
	\includegraphics[width=0.45\linewidth]{tabelle2/Z_Rot_NN}
	\caption{Zusammenhang zwischen der Skalierung (X-Achse) und der Abweichung des Winkels in Z-Richtung, Angabe in Bogenmaß.
		Bicubic (oben links), Lanczos (oben rechts), Linear (unten links), Nearest-Neighbor (unten rechts)}
	\label{img_Z_Rot_Skal}
\end{figure}
\begin{figure}
	\centering
	\includegraphics[width=0.45\linewidth]{tabelle2/Z_Rot_Val_Cubic}
	\includegraphics[width=0.45\linewidth]{tabelle2/Z_Rot_Val_Lanc}
	\includegraphics[width=0.45\linewidth]{tabelle2/Z_Rot_Val_Linear}
	\includegraphics[width=0.45\linewidth]{tabelle2/Z_Rot_Val_NN}
	\caption{Zusammenhang zwischen der Skalierung (X-Achse) und der Wertebereiche des Winkels in Z-Richtung, Angabe in Bogenmaß.
		Bicubic (oben links), Lanczos (oben rechts), Linear (unten links), Nearest-Neighbor (unten rechts)}
	\label{img_Z_Rot_Val_Skal}
\end{figure}
Kein erkennbarer Unterschied zwischen den einzelnen Verfahren. Wobei bei Nearest-Neighbor deutlich früher der Wertebereich sinkt.
\subsection{To Do}
\begin{itemize}
	\item Patch Experts und Optimierungsfunktionen CLM
	\item Auswirkung von Pixelrauschen
	\begin{itemize}
		\item Rauschen der Actioncam bestimmen\\
		Done
		\item Simulation des Rauschens\\
		Add Gaußverteilung auf Image 
	\end{itemize}
\item Angabe des Koordiantansystems
\item Auswerten der Messung
\item Wann ELSE
\item Mittlung Ergebnis / Landmarks
\item Zuverlässigkeit mit Farbkorrektur
\end{itemize}
\section{OpenFace auf Video - To Do}
Durch das Lernen von OpenFace muss auch die Qualität auf einem Video betrachtet werden. Dazu wurde ein eigener Datensatz erstellt und ausgewertet.\\
Für den Versuch wurde ein Video verwendet, Welches ein Bewegtes Kreuz zeigt. Dieses Kreuz sollten die Probanden normal mit dem Blick folgen damit für jeden Zeitpunkt die Blickrichtung bekannt ist.
\subsection{Versuchsaufbau}
Die Anordnung der Eckpunkte sind in \autoref{img_targets} Dargestellt und wurden mittels eines Projektors auf eine Breite von $2.88m$ und eine Höhe von $1.49m$.\\
Das Ziel das Betrachtet werden soll (Target), beginnt immer in der Mitte und bleibt dort $1s$ Stehen, bewegt sich innerhalb von 4 Sekunden einen der Randpunkte, dargestellt in \autoref{img_targets}, verweilt dort für eine Sekunde und begibt sich in $4s$ zu einem nächstgelegenen Randpunkt, bleibt dort $1s$ und geht zurück zum Zentrum, dies wiederholt.\\
Die Versuchspersonen stellten sich etwa $1.5m$ von der vor der Leinwand entfernt auf, die Kamera befand sich $24cm$ unterhalb und $12.5cm$ vor dem Zentralen Punkt der Targets mit Blickrichtung von den Targets weg.
\begin{figure}
	\centering
	\fbox{\includegraphics[width=0.7\linewidth]{img/Targets}}
	\caption{Eckpositionen des Bewegten Zieles bei der Videoaufnahme}
	\label{img_targets}
\end{figure}
\subsection{Versuchs - Durchführung}
Um die ungefähre Position des Kopfes zu ermitteln, wurde die Distanz zwischen dem Nasenrücken und den 4 Eckpunkten mittels eines Laserdistanzmessers bestimmt um die Position relativ zur Leinwand und Kamera zu ermitteln.\\
Während der Aufnahme wurde auf weitere Messung der exakten Position verzichtet.
Die 6 Probanden (5 Männlich, 1 Weiblich, 3 Brille, 5 Ohne) verfolgten das Ziel $2m$ und $1s$ auf natürlicher weise.\\
Um die Bewegung des Punktes mit der Aufgezeichneten Kopfbewegung zu Synchronisieren, war im Kamerabild der duplizierte Bildschirm zum Projektor zusehen.\\
Die Aufnahme wurde mit $15Fps$ in Farbe mit einer Auflösung von $1600\times 896$ Pixel aufgezeichnet. Die Kamera besitzt einen horizontalen Blickwinkel von etwa $70^\circ$.
\subsection{Ergebnis - To Do}
Dargestellt sind alle Auftreffpunkte der Blickrichtung auf die Leinwand währen der gesamten Aufnahme.
\begin{itemize}
	\item Graphik mit den Blickverfolgung
	\item Plot Winkel gegen Winkel und Abweichung
	\item 
\end{itemize}
\subsection{Fehleranalyse}
Eine Betrachtung der Fehlerquellen die Bei der Messung entstanden sind bzw. die durch den Aufbau Entstehen. Außerdem eine weitere bei der Berechnung.
\subsubsection{Messung}
Die erste Ungenauigkeit liegt bei der Distanz zur Leinwand, diese wurde nur zu beginn, vor der eigentlichen Aufnahme bestimmt. Somit ist entsteht eine Abweichung da Kopf in Bewegung ist, auch währen der Aufnahme.\\
Die eigentliche Messung der Distanz ist ebenfalls ungenau, da sie eine Abweichung von etwa $1cm$ in alle Richtungen aufweist. Außerdem liegt der Ursprung der Rechnung etwas Tiefer und weiter Hinten als der Messpunkt.
Die Parameter für der Überführungsmatrix von Welt- nach Kamerakoordinaten sowie die Brennweite wurden zwar sorgsam bestimmt, sind aber dennoch nicht perfekt.\\
Durch den Bedingten Aufbau, musste die Kamera in Richtung des Projektors ausgerichtet werden, wodurch diese wiederum von dem direkten Licht geschützt werden musste. Somit konnte sich die Kamera nicht im Zentrum der Messpunkte befinden.\\
Da die Kamera und die Leinwand fest Montiert sind, ergibt sich auch die Problematik das der Kopf der Probanden ebenfalls nicht im Zentrum des Kamerabildes Befinden und somit immer ein Blickwinkel von unten auf das Gesicht entsteht.\\
Da die Probanden ebenfalls zwischen der Leinwand und dem Projektor standen, verdeckten diese das Bild, wodurch es manchmal passierte das der Zielpunkt im Schatten verschwand.
\subsubsection{Umgebung}
Bei der Aufzeichnung hat sich vor allem das Problem mit der ungleichmäßigen Beleuchtung bzw. dem Gegenlicht ergeben. Diesem musste entgegengewirkt werden, damit das Gesicht gut erkennbar ist. Ein Problem das auch in der realen Anwendung auftreten wird.\\
Ein weiteres allgemeines Problematik zeigt sich auch wieder bei der Auflösung des Gesichtes, somit ist eine Berechnung auf dem Gesicht zwar möglich, auf den Augen allerdings nicht.\\
Somit ergibt sich ein weiteres Problem, da im allgemeinen eine Exkursionen, der Winkelbereich der Augenbewegungen, bis etwa  $20^\circ$ stattfindet und diese nicht erfasst werden können.\\
Ein Weiterer nicht zu verachtendes Problem ist die Ruflektion vor allem auf den Brillen, von den starken Lichtquellen wie Fenster, Projektor- und dessen Bild sowie der Lampen. Auch Schatten gerade bei den Augenhöhlen erschweren die Auswertung. 
\begin{itemize}
	\item Bild für den Versuchsaufbau
	\item Typ der Kamera
\end{itemize}

\section{Fehleranalyse}
Mit entsprechend hochauflösenden Kameras können auch bessere Resultate auf größerer Distanz erreicht werden. Gerade die Bestimmung der Blickrichtung ist meist nicht möglich, da die Augenpartie viel zu klein für eine Berechnung ist. So bleibt meist nur die Gesichtsorientierung mit ihr natürlichen Ungenauigkeit.\\
Da Bewegung erlaubt ist, passiert es immer wieder, dass Teile des Gesichtes verdeckt werden, durch Hände beim Melde, andere Schüler oder dem Lehrer, der vor der Kamera steht oder sich der Kopf zu weit wegdreht und das Tracking nicht mehr möglich ist.. Aber auch die Frisuren spielen eine Rolle, da dadurch diese einige Landmarks verdeckt werden und so das Gesicht nicht erkannt wird wie z.B. die Augenbrauen.\\
Eine Lösungsansatz währen Landmarks in Profilbildern zu detektieren und sowie das verwenden von weiteren Kameras aus anderen Perspektiven.\\
\section{Verbesserungen}
\begin{itemize}
	\item Mehrere Kameras für 3D und weniger verdecken und wegdrehen
\end{itemize}

%%%%%%%%%%%%%%%%%%%%%%%%%%%%%%%%%
% Das Literaturverzeichnis      %
%%%%%%%%%%%%%%%%%%%%%%%%%%%%%%%%%

\addcontentsline{toc}{chapter}{Literaturverzeichnis}
% Dadurch werden die Zitate mit abgekürzten Namen der Autoren generiert.
\bibliographystyle{alpha}
\bibliography{Quellenverzeichnis}
\end{document}