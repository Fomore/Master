\section{Software}
Für die Umsetzung werden folgende Software-Elemente aus fremder Quelle eingesetzt.
\subsection{ElSe}
Ellipse Selection for Robust Pupil Detection in Real-World Environments, ein Algorithmus zur Bestimmung der Pupille im Bild. Der Ursprüngliche ElSe-Algorithmus ist für Graubilder mit Infrarotbeleuchtung ausgelegt, wurde für diese Anwendung auf Farbbilder modifiziert.\\
Entwickelt von der Uni Tübingen. \cite{ElSe}
\subsection{MTCNN Face Detection}
Multi-task Cascaded Convolutional Networks, ein Algorithmus zur Detektion von Gesichtern und Bestimmung von 5 Gesichts-Landmarks in Farbbilder. Dabei werden drei CNN auf einer Bildpyramide angewendet um so zuverlässig Gesichter mit verschiedenster Größe zu erkennen.\\
\cite{MTCCN}
\subsection{OpenCV}
Open Source Computer Vision, ist eine C/C++ Bibliothek von Algorithmen zur Bildverarbeitung in Echtzeit unter der BSD Lizenz (Berkeley
Software Distribution)\\
\cite{wiki_Wha_is_OPenCV}\cite{OpenCv_What_Is}
\subsection{OpenFace}
Ein Open-Source Echtzeitverfahren auf Basis von CLNF zur Bestimmung und Analyse von Gesichtsmerkmalen in Grau-Bildern und Videos. Dabei werden 68 signifikante Punkte im Gesicht bestimmt und auf Basis jener Position und Orientierung ermittelt.\\
Entwickelt von der University of Cambridge \cite{OpenFace}
