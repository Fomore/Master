\section{Skalierung auf Mindestgröße}
\label{skalierung}
Da OpenFace optimiert ist auf Gesichtern von mindestens 100 Pixel zu arbeiten, werden die Bildbereiche auf diese Größe hochskaliert. \autoref{scale_Algos}\\
Dies erhöht den Informationsgehalt der Bilder nicht, sie sind nur besser nutzbar, da sie dem Trainingsdatensatz stärker ähneln.
Die von MTCNN gelieferten und vergrößerten Boxen werden nun auf $130 \times 180$ Pixel gebracht um Ungenauigkeiten bezüglich der Position und Dimension des Kopfes im Bild entgegen zu wirken. Neben der Skalierung des Bildausschnittes, muss bekannt sein, wie Punkte im skalierten Bildausschnitt in das Frame überführt werden kann, damit dies bei späteren Berechnungen berücksichtigt wird.\\
Die Skalierung ist für jeden Bildausschnitt individuell und kann sich durchaus über die Zeit ändern, wenn sich z.B. die Distanz zwischen Person und Kamera verändert.\\
Von einer zu starken Vergrößerung ist abzuraten, da sich der Rechenaufwand pro Gesicht erhöht und die Zuverlässigkeit der Berechnungen von OpenFace sinkt, z.B. durch Falschdetektion.