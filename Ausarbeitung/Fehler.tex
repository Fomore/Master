\section{Fehleranalyse}
Mit entsprechend hochauflösenden Kameras können auch bessere Resultate auf größerer Distanz erreicht werden. Gerade die Bestimmung der Blickrichtung ist meist nicht möglich, da die Augenpartie viel zu klein für eine Berechnung ist. So bleibt meist nur die Gesichtsorientierung.\\
Da Bewegung erlaubt ist passiert es immer wieder, dass Teile des Gesichtes verdeckt werden durch Hände beim Melde, andere Schüler oder dem Lehrer, der vor der Kamera steht oder sich der Kopf zu weit wegdrehen. Aber auch die Frisuren spielen eine Rolle, da dadurch diese einige Landmarks verdeckt werden und so das Gesicht nicht erkannt wird wie z.B. die Augenbrauen.\\
Eine Lösungsansatz währen Landmarks in Profilbildern zu detektieren und das verwenden von weiteren Kameras aus anderen Perspektiven.\\
\section{Verbesserungen}
\begin{itemize}
	\item Mehrere Kameras für 3D und weniger verdecken und wegdrehen
\end{itemize}