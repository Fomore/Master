\section{Fehleranalyse}
Mit entsprechend hochauflösenden Kameras, können auch bessere Resultate auf größeren Distanzen erzielt werden. Gerade die Bestimmung der Blickrichtung auf großer Distanz ist meist nicht möglich, da die Augenpartie viel zu klein für eine Berechnung ist. So bleibt meist nur die Gesichtsorientierung mit ihr natürlichen Ungenauigkeit.\\
Da Bewegung erlaubt ist, passiert es immer wieder, dass Teile des Gesichtes verdeckt werden, durch Hände beim Melden, andere Schüler oder dem Lehrer selbst, der vor der Kamera steht oder sich der Kopf zu weit wegdreht und das Tracking scheitert. Aber auch die Frisuren spielen eine Rolle, da dadurch diese einige Landmarks verdeckt werden können, wie z.B. die Augenbrauen, und das Gesicht nicht erkannt wird .