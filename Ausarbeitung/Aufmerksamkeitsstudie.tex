\section{Übliche Erfassung von On/OFF-Task}
Für \glqq Das Münchener Aufmerksamkeitsinventar (MAI)\grqq \cite{MAI_Verhaltensbeobachtung} wurden beispielsweise die Kategorien \textit{\glqq ON-TASK, reaktiv/fremd-initiiert: der Schüler reagiert auf eine entsprechende Aufforderung oder Frage des Lehrers\grqq} oder \textit{\glqq OFF-TASK - aktiv, interagierend, störend: Der Schüler nimmt die Lerngelegenheit nicht nur nicht wahr, sondern ist erkennbar anderweitig engagiert\grqq}, festgelegt. Um das Verhalten eines Schülers zu bewerten wird dieser $5s$ lange beobachtet und einer Kategorie zugeordnet.\\
Bei der \glqq Videostudie zur Wirksamkeit des Unterrichtsprozesses \grqq \cite{aufmerksamkeit_Studie} wurden die Kriterien \textit{\glqq Blickkontakt zum legitimen Sprecher oder Objekt, Aktive Beteiligung an der Aufgabe, keine Ausübung anderer Tätigkeiten, keine Motorische Unruhe und keine themenferne Kommunikation\grqq}, festgelegt. Dann wurde der Schüler in einem Ein-Minuten-Intervall beobachtet und bewertet. Sind drei oder mehr Kriterien erfüllt, gilt der Schüler als on-Task (Aufmerksam).\\
Bei dieser Art der Auswertung gibt es allerdings Interpretationsfreiheiten, die von jedem Beobachter anders ausgelegt werden können. So kann das spielen mit dem Stiftes als motorische Unruhe oder nur als Zeichen von Nervosität bewertet werden. Außerdem ist diese Art der Auswertung sehr zeitintensiv, alleine eine einzige Beurteilung jedes einzelnen Schülers einer Klasse, etwa 30 Personen nach Vorgabe der Klassenbildung \cite{klassenteiler}, benötigt mindestens 30 Minuten. Somit kann eine Auswertung aller Schüler während einer Unterrichtsstunde schnell 15 und mehr Arbeitsstunden dauern. Um subjektive Bewertungen zu vermeiden, sollte außerdem ein beträchtlicher Teil der Daten von mindestens zwei Beobachtern parallel ausgewertet werden, um deren Übereinstimmung beurteilen zu können, was noch mehr Arbeit bedeutet.\\
Basiert die Auswertung auf wenigen Zeitintervalle um Arbeitszeit zu sparen, wird das gesamte Verhalten eines Schülers währen des Unterrichts mit nur wenigen beobachteten Minuten beschrieben und entsprechend ungenau. Somit können sowohl quantitativ genaue, als auch temporal hochauflösende Daten nicht sinnvoll erstellt werden.\\
So kann bei grob gewählten Auswertungsintervallen nur eine Aussage über den gesamten Unterricht gemacht werden und nicht beispielsweise über einzelne Übungen oder über einen einzelnen Schüler.