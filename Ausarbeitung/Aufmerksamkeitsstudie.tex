\section{Erfassung von ON/OFF-Task}
\label{on_off_Task}
\glqq Das Münchener Aufmerksamkeitsinventar (MAI)\grqq \cite{MAI_Verhaltensbeobachtung} ist ein Verfahren zur Bestimmung der Aufgabenbezogenen Aufmerksamkeit. Es unterscheidet unter anderem zwischen \textit{\glqq ON-TASK, reaktiv/fremd-initiiert: der Schüler reagiert auf eine entsprechende Aufforderung oder Frage des Lehrers\grqq} oder \textit{\glqq OFF-TASK - aktiv, interagierend, störend: Der Schüler nimmt die Lerngelegenheit nicht nur nicht wahr, sondern ist erkennbar anderweitig engagiert\grqq}. Um das Verhalten eines Schülers zu bewerten wird dieser $5s$ lang beobachtet und einer dieser binären Kategorien zugeordnet.\\
Bei der \glqq Videostudie zur Wirksamkeit des Unterrichtsprozesses \grqq \cite{aufmerksamkeit_Studie} wurden die Kriterien \textit{\glqq Blickkontakt zum legitimen Sprecher oder Objekt, Aktive Beteiligung an der Aufgabe, keine Ausübung anderer Tätigkeiten, keine Motorische Unruhe und keine themenferne Kommunikation\grqq}, unterschieden. Ein Schüler wird in einem Ein-Minuten-Intervall beobachtet und bewertet. Sind drei oder mehr Kriterien erfüllt, gilt das Intervall als ON-Task.\\
Bei dieser Art der Auswertung gibt es allerdings Interpretationsfreiheiten, die von jedem Beobachter anders ausgelegt werden können. So kann z.B. das Spielen mit einem Stift als motorische Unruhe oder als Zeichen von Nervosität bewertet werden. Außerdem ist diese Art der Auswertung sehr zeitintensiv, alleine die Beurteilung eines einzelnen Schülers einer Klasse benötigt mindestens 30 Minuten. Somit kann eine Auswertung aller Schüler (etwa 30 Personen nach Vorgabe der Klassenbildung \cite{klassenteiler}) für eine einzelne Unterrichtsstunde schnell 15 und mehr Arbeitsstunden dauern. Um subjektive Bewertungen zu vermeiden sollte außerdem ein beträchtlicher Teil der Daten von mindestens zwei Beobachtern parallel ausgewertet werden, um deren Übereinstimmung beurteilen zu können.\\
Basiert die Auswertung auf wenigen ausschnittsweisen Zeitintervalle um Arbeitszeit zu sparen, wird das gesamte Verhalten eines Schülers während des Unterrichts mit nur wenigen beobachteten Minuten beschrieben. Somit können sowohl quantitativ genaue, als auch temporal hochauflösende Daten nicht erstellt werden.\\
Bei grob gewählten Auswertungsintervallen kann also nur eine Aussage über den gesamten Unterricht gemacht werden und nicht beispielsweise über einzelne Übungen oder einen einzelnen Schüler.