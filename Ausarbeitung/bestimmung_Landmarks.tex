\section{Bestimmung der Landmarks}
\label{bestimmung_Landmarks}
Für die Bestimmung der Landmarks wird OpenFace auf den Bildausschnitten eingesetzt. Dies bietet mehrere Vorteile, so wird nur auf Bildbereichen gearbeitet, in denen ein Gesicht zu sehen ist und unnötige Suche vermeiden. Außerdem kann für jede Person die passende Initialisierung des CLNF basierend auf dem letzten Ergebnis dieser Person gewählt werden, auch auf für jene Personen die nur selten zu sehen sind. Auf diese Weise kann der Bildausschnitt möglichst exakt und allen gleichzeitig ausgewertet werden.\\
Für die eigentliche Bestimmung der Landmarks bietet OpenFace zwei verschiedene Methoden, die Berechnung auf Bildern und Videos. Der Hauptunterschied ist das Lernen, dass bei der Videoauswertung verwendet wird, wodurch sich der Arbeitsbereich auf dem Ergebnisse geliefert werden sich deutlich erhöht. Dies liegt an der Anpassung des Modells und dem möglichen Tracking der Landmarks.\\
Dies ist interessant für die spätere Anwendung, da somit auch Einzelbilder verwendet werden können, die eine deutlich höhere Auflösung besitzen als ein Video. Allerdings sinkt bei der Verwendung von Einzelbilder der maximale Winkel relativ zur Kamera beträchtlich, zu Gunsten der maximalen Distanz. Außerdem hat sich gezeigt das bei Verwendung eines Videos das Gesicht deutlich kleiner sein kann bis endgültig keine Auswertung mehr möglich ist, kann bei einer erfolgreichen Detektion auch die nachfolgenden Frames ausgewertet werden können.\\
Dennoch kann es passieren, dass trotz allem ein Gesicht falsch detektiert wird, wie z.B. das Erkennen eines sehr kleinen Gesichtes innerhalb einer Ohrmuschel. In solch einem Fall muss das CLNF zurückgesetzt werden, damit sich der Fehler nicht fortpflanzt.