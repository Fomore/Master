\section{Bestimmung der Landmarks}
\label{bestimmung_Landmarks}
Für die Bestimmung der Landmarks wird OpenFace eingesetzt. Dabei wird jeder Bildausschnitt unabhängig der anderen Betrachtet und da bekannt ist, um welche Person es ich im Bild handelt, kann direkt mit dem jeweiligen CNN gearbeitet werden, das auf diese Person optimiert wurde.\\
Durch die vorige Selektion wird nur auf jenen Bildausschnitten gerechnet auf denen auch die Person zu sehen ist, wodurch nicht unnötig gesucht werden muss und auch ein Lernen auf Personen stattfinden kann die nur selten zu sehen sind, da sie nur resettet werden, wenn sie eigentlich zu sehen sein müssten aber nicht detektiert wurden.\\
Für die eigentliche Bestimmung der Landmarks bietet OpenFace zwei verschiedene Methoden, die Berechnung auf Bildern und Videos. Der Hauptunterschied ist das Lernen, dass bei der Videoauswertung verwendet wird, wodurch sich die Bereiche, auf denen Ergebnisse geliefert werden, deutlich erhöht.\\
Dies ist interessant für die spätere Anwendung, da somit auch Einzelbilder verwendet werden können, die eine deutlich höhere Auflösung haben als ein Video. Allerdings sinkt dann der maximale Winkel relativ zur Kamera beträchtlich, zu Gunsten der maximalen Distanz. Außerdem können schon kleinste Farbänderungen im Bild beim Hochskallieren ausschlaggebend sein, ob ein Gesicht erkannt werden kann, wodurch bei gleicher Bildqualität Gesichter im Video besser erkannt werden.\\
Da die gesamte Berechnung auf Grau-Bildern basiert ist auch eine Farbkorrektur, wie Verbesserung des Kontrast, Farbverlauf usw. möglich, um etwaige Einflüsse bei der Aufnahme zu korrigieren.\\
Dennoch kann es passieren, dass trotz allem  ein Gesicht falsch detektiert wird, wie z.B. das erkennen eines Gesichtes in der Ohrmuschel, diese müssen entsprechend behandelt werden, da ansonsten das Lernen auf diese Bereiche stattfindend und im nächsten Frame erneut nach diesen Merkmalen gesucht wird.

\begin{itemize}
\item Verbesserung durch Farbkorrektur
\end{itemize}