\section{OpenFace auf Video}
Durch das Lernen von OpenFace muss auch die Qualität auf einem Video betrachtet werden. Dazu wurde ein eigener Datensatz erstellt und ausgewertet.\\
Für den Versuch wurde ein Video verwendet, welches ein bewegtes Kreuz zeigt. Dieses Kreuz sollten die Probanden mit dem Blick normal folgen damit für jeden Zeitpunkt das Ziel der Aufmerksamkeit bekannt ist.
\subsection{Versuchsaufbau}
Die Anordnung der Eckpunkte sind in \autoref{img_targets} dargestellt und wurden mittels eines Projektors auf eine Breite von $2.88m$ und eine Höhe von $1.49m$ gebracht.\\
Das Ziel welches betrachtet werden soll (Target) beginnt immer in der Mitte und bleibt dort $1s$ stehen, bewegt sich innerhalb von 4 Sekunden zu einen der Randpunkte, verweilt dort für eine Sekunde und begibt sich in $4s$ zu einem nächstgelegenen Randpunkt, bleibt dort $1s$ und geht zurück zum Zentrum, dies wiederholt sich für alle Eckpunkte. Ein gesamter Durchlauf dauert 2min und 1s.\\
Die Versuchspersonen stellten sich etwa $1.5m$ vor der Leinwand entfernt auf, die Kamera befand sich $24cm$ unterhalb und $12.5cm$ vor dem zentralen Punkt der Targets mit Blickrichtung zum Projektor und Personen.
\begin{figure}
	\centering
	\fbox{\includegraphics[width=0.7\linewidth]{img/Targets}}
	\caption{Eckpositionen des Bewegten Zieles bei der Videoaufnahme}
	\label{img_targets}
\end{figure}
\subsection{Versuchsdurchführung}
Um die ungefähre Position des Kopfes zu ermitteln, wurde die Distanz zwischen Stirn auf dem Nasenrücken und den 4 Eckpunkten mittels eines Laserdistanzmessers bestimmt um die Position relativ zur Leinwand und Kamera ermitteln zu können. Während der Aufnahme wurde auf weitere Messung der exakten Position verzichtet.\\
Die 6 Probanden (5 Männlich, 1 Weiblich, 3 Brille, 5 Ohne) verfolgten das Ziel natürliche Weise.\\
Um die Bewegung des Punktes mit der Aufgezeichneten Kopfbewegung zu Synchronisieren, war im Kamerabild der duplizierte Bildschirm zum Projektorbild zusehen.\\
Die Aufnahmen wurden mit der Logiteh-Webcam \autoref{hardware} erstellt.
\subsection{Ergebnis}
Die Auswertung des Versuches hat die Erwartungen und Problematiken bestätigt. Eine Verarbeitung des Videomaterials ist sogar bei sehr niedriger Auflösung noch möglich, wobei die Qualität besser sein könnte.
\subsubsection{Erster Eindruck}
Dargestellt in \autoref{img_videosumme} sind alle Auftreffpunkte der Blickrichtung auf die Leinwand währen der gesamten Aufnahme.\\
Es ist zu erkennen, dass die eigentlichen Kopfbewegungen sichtbar sind, es aber vor allem in den Randbereichen zu einer großen Differenz kommt.\\
Da nur der Unterschied zwischen Target und Auftreffpunkt der gemessenen Gesichtsorientierung aufgezeigt werden kann, kommt es zu verschiedenen Fehlern, vor allem wird das Target mit den Augen gefolgt wodurch zu beginn der Bewegung, dem Target nur mit den Augen gefolgt wird, bis sich der Kopf bewegt. Dies wird so lange fortgeführt, bis die Kopfdehnung unangenehm wird und der das ende absehbar ist, wodurch die letzten Bewegungen nurnoch von den Augen gemacht werden (Quelle).
\begin{figure}
	\centering
	\includegraphics[width=0.7\linewidth]{OpenFace_Img/VideoSumme}
	\caption{Dargestellt sind alle gemessene Auftreffpunkte der Gesichtsorientierung auf die Leinwand (Rosa) und des Targets (Schwarz)}
	\label{img_videosumme}
\end{figure}
\subsubsection{Qualität}
Durch die begrenzte Auflösung der Kamera und dem großen Distanzbereich auf dem gearbeitet werden muss, ist vor allem die Stabilität bei der Skalierung wichtig.\\
Bei der Bestimmung des horizontalen Winkels der Kopforientierung zeig sich das die bestimmten Werte im Schnitt etwas zu gering ausfallen, die Orientierung in Richtung Kamera kann zuverlässig bestimmt werden, je größer der zu messende Winkel wird, desto  stärker wird auch der Fehler.\\
Betrachtet man in der Originalgröße die jeweiligen Quartale, so sind die Grenzen etwa $5^\circ$ auseinander. Genug um einzelne Bereiche differenzieren zu können, jedoch zu ungenau für Berechnungen.\\
Bei der Bestimmung des vertikalen Winkels zeigt sich, das dieser Wert nur sehr ungenau bestimmt werden konnte, vor allem der Winkel nach Oben ist fast nicht messbar. Jener Richtung Boden wird besser erfasst, allerdings ist, bedingt durch den Versuchsausbau, der Wertebereich recht gering.\\
Die bestimmte Blickrichtung ist trotz Verbesserung durch ElSe und Mittlung beider Augen, schon in der Originalgröße nur begrenzt verwendbar. Die Mittelwerte liegen selbst bei den Maximal Werten sehr eng beieinander und die Bereiche überschneiden sich stark. Die Differenz der Mittelwerte von den Extremar sind nur etwa $20^\circ$ auseinander liegen, bei einer eigentlichen Differenz von etwa $90^\circ$.\\
Die Auswirkung der Skalierung ist annehmbar gering, allgemein steigt die Abweichung und der Bereich der Detektion sinkt. Bei einem Skalierungsfaktor von 0.01 können die einzelnen Bereiche noch gut getrennt werden, dies entspricht eine Distanz von etwa $14m$. Auf der horizontalen Achse liegt der Abstand der Quartale etwa $9^\circ$ weit auseinander.\\
Bei der Bestimmung des vertikalen Winkels ergibt sich ein ähnliches Verhalten, wobei vor allem der Wertebereich auf $30^\circ$ sinkt.\\
Das Ergebnis der Blickrichtung kann  bei dieser Skalierung nicht verwendet werden, da die Differenz zwischen dem Rechten und Linken Maximalwert nur $8^\circ$ beträgt und die Quartale sich fast vollständig überschneiden.\\
Überraschend ist das Ergebnis bei dem Skalierungsfaktor von 0.05 (ca $24m$). Die Ausrichtungen sind, zumindest horizontal, noch erkennbar und soweit differenzierbar um grobe Richtungsänderungen zu erkennen.
\begin{figure}
	\centering
	\includegraphics[width=0.3\linewidth]{OpenFace_Img/Head_x_S1}
	\includegraphics[width=0.3\linewidth]{OpenFace_Img/Head_y_S1}
	\includegraphics[width=0.3\linewidth]{OpenFace_Img/EyeAVG_x_S1}\\
	\includegraphics[width=0.3\linewidth]{OpenFace_Img/Head_x_S05}
	\includegraphics[width=0.3\linewidth]{OpenFace_Img/Head_y_S05}
	\includegraphics[width=0.3\linewidth]{OpenFace_Img/EyeAVG_x_S05}\\
	\includegraphics[width=0.3\linewidth]{OpenFace_Img/Head_x_S025}
	\includegraphics[width=0.3\linewidth]{OpenFace_Img/Head_y_S025}
	\includegraphics[width=0.3\linewidth]{OpenFace_Img/EyeAVG_x_S025}\\
	\includegraphics[width=0.3\linewidth]{OpenFace_Img/Head_x_S01}
	\includegraphics[width=0.3\linewidth]{OpenFace_Img/Head_y_S01}
	\includegraphics[width=0.3\linewidth]{OpenFace_Img/EyeAVG_x_S01}\\
	\includegraphics[width=0.3\linewidth]{OpenFace_Img/Head_x_S005}
	\includegraphics[width=0.3\linewidth]{OpenFace_Img/Head_y_S005}
	\includegraphics[width=0.3\linewidth]{OpenFace_Img/EyeAVG_x_S005}
	\caption{Dargestellt ist die Auswertung der Videoaufnahme mit der Kopfausrichtung Horizontal (Links), Kopforientierung Vertikal (Mitte) und die X-Ausrichtung der Augen (Rechts)\\Skalierungsfaktor von oben nach unten (1/0.5/0.25/0.1/0.05)}
	\label{graph_VideoSkalierung}
\end{figure}
\subsection{Fehleranalyse des Versuches}
Eine Betrachtung der Fehlerquellen die bei der Messung entstanden sind bzw. die durch den Aufbau Entstehen, sowie bei der Berechnung.
\subsubsection{Messung}
Die erste Ungenauigkeit liegt bei der Distanz zur Leinwand, diese wurde nur vor der eigentlichen Aufnahme bestimmt. Somit ist entsteht eine Abweichung da die Kopfbewegung während der Aufnahme nicht erfasst wird.\\
Die eigentliche Messung der Distanz  vom Kopf der Personen zur Leinwand ist ebenfalls ungenau, da sie eine Abweichung von etwa $1cm$ in alle Richtungen aufweist. Außerdem liegt der Ursprung des Kopfes in der Anwendung etwas Tiefer und weiter Hinten als der gemessene Nasenrücken ist.
Die Parameter für der Überführungsmatrix von Welt- nach Kamerakoordinaten sowie die Brennweite wurden zwar sorgsam bestimmt, sind aber dennoch nicht perfekt.\\
Bedingt durch den Aufbau und der verwendeten Hardware, musste die Kamera in Richtung des Projektors ausgerichtet werden, wodurch diese wiederum von dem direkten Licht geschützt werden musste. Somit konnte sich die Kamera nicht im Zentrum der Messpunkte befinden.\\
Da die Kamera und die Leinwand fest Montiert sind, ergibt sich auch die Problematik das der Kopf der Probanden ebenfalls nicht im Zentrum des Kamerabildes Befinden und somit hat die Kamera immer einen  Blickwinkel von unten auf das Gesicht.\\
Da die Probanden ebenfalls zwischen der Leinwand und dem Projektor standen, verdeckten diese das Bild, wodurch es manchmal passierte das der Zielpunkt im Schatten verschwand.
\subsubsection{Umgebung}
Bei der Aufzeichnung hat sich vor allem das Problem mit der ungleichmäßigen Beleuchtung bzw. dem Gegenlicht ergeben. Diesem musste durch abdunkeln der Fenster und Verwendung der Tafelbeleuchtung entgegengewirkt werden, damit das Gesicht gut erkennbar ist. Ein Problem das auch in der realen Anwendung auftreten wird.\\
Ein weiteres allgemeines Problematik zeigt sich auch wieder bei der Auflösung des Gesichtes, somit ist eine Berechnung auf dem Gesicht zwar möglich, auf den Augen allerdings nicht.\\
Somit ergibt sich ein weiteres Problem, da im allgemeinen eine Exkursionen, der Winkelbereich der üblichen Augenbewegungen, bis etwa  $20^\circ$ stattfindet und diese nicht erfasst werden können.\\
Ein weiteres nicht zu verachtendes Problem ist die Reflektion vor allem auf den Brillen, von den starken Lichtquellen wie z.B. Fenster, Projektor- und dessen Bild, sowie Lampen die Pupille verdecken. Auch Schatten gerade bei den Augenhöhlen erschweren die Auswertung. 
\begin{itemize}
	\item Bild für den Versuchsaufbau
\end{itemize}
