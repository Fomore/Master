\section{OpenFace auf Video - To Do}
Durch das Lernen von OpenFace muss auch die Qualität auf einem Video betrachtet werden. Dazu wurde ein eigener Datensatz erstellt und ausgewertet.\\
Für den Versuch wurde ein Video verwendet, Welches ein Bewegtes Kreuz zeigt. Dieses Kreuz sollten die Probanden normal mit dem Blick folgen damit für jeden Zeitpunkt die Blickrichtung bekannt ist.
\subsection{Versuchsaufbau}
Die Anordnung der Eckpunkte sind in \autoref{img_targets} Dargestellt und wurden mittels eines Projektors auf eine Breite von $2.88m$ und eine Höhe von $1.49m$.\\
Das Ziel das Betrachtet werden soll (Target), beginnt immer in der Mitte und bleibt dort $1s$ Stehen, bewegt sich innerhalb von 4 Sekunden einen der Randpunkte, dargestellt in \autoref{img_targets}, verweilt dort für eine Sekunde und begibt sich in $4s$ zu einem nächstgelegenen Randpunkt, bleibt dort $1s$ und geht zurück zum Zentrum, dies wiederholt.\\
Die Versuchspersonen stellten sich etwa $1.5m$ von der vor der Leinwand entfernt auf, die Kamera befand sich $24cm$ unterhalb und $12.5cm$ vor dem Zentralen Punkt der Targets mit Blickrichtung von den Targets weg.
\begin{figure}
	\centering
	\fbox{\includegraphics[width=0.7\linewidth]{img/Targets}}
	\caption{Eckpositionen des Bewegten Zieles bei der Videoaufnahme}
	\label{img_targets}
\end{figure}
\subsection{Versuchs - Durchführung}
Um die ungefähre Position des Kopfes zu ermitteln, wurde die Distanz zwischen dem Nasenrücken und den 4 Eckpunkten mittels eines Laserdistanzmessers bestimmt um die Position relativ zur Leinwand und Kamera zu ermitteln.\\
Während der Aufnahme wurde auf weitere Messung der exakten Position verzichtet.
Die 6 Probanden (5 Männlich, 1 Weiblich, 3 Brille, 5 Ohne) verfolgten das Ziel $2m$ und $1s$ auf natürlicher weise.\\
Um die Bewegung des Punktes mit der Aufgezeichneten Kopfbewegung zu Synchronisieren, war im Kamerabild der duplizierte Bildschirm zum Projektor zusehen.\\
Die Aufnahme wurde mit $15Fps$ in Farbe mit einer Auflösung von $1600\times 896$ Pixel aufgezeichnet. Die Kamera besitzt einen horizontalen Blickwinkel von etwa $70^\circ$.
\subsection{Ergebnis - To Do}
Dargestellt sind alle Auftreffpunkte der Blickrichtung auf die Leinwand währen der gesamten Aufnahme.
\begin{itemize}
	\item Graphik mit den Blickverfolgung
	\item Plot Winkel gegen Winkel und Abweichung
	\item 
\end{itemize}
\subsection{Fehleranalyse}
Eine Betrachtung der Fehlerquellen die Bei der Messung entstanden sind bzw. die durch den Aufbau Entstehen. Außerdem eine weitere bei der Berechnung.
\subsubsection{Messung}
Die erste Ungenauigkeit liegt bei der Distanz zur Leinwand, diese wurde nur zu beginn, vor der eigentlichen Aufnahme bestimmt. Somit ist entsteht eine Abweichung da Kopf in Bewegung ist, auch währen der Aufnahme.\\
Die eigentliche Messung der Distanz ist ebenfalls ungenau, da sie eine Abweichung von etwa $1cm$ in alle Richtungen aufweist. Außerdem liegt der Ursprung der Rechnung etwas Tiefer und weiter Hinten als der Messpunkt.
Die Parameter für der Überführungsmatrix von Welt- nach Kamerakoordinaten sowie die Brennweite wurden zwar sorgsam bestimmt, sind aber dennoch nicht perfekt.\\
Durch den Bedingten Aufbau, musste die Kamera in Richtung des Projektors ausgerichtet werden, wodurch diese wiederum von dem direkten Licht geschützt werden musste. Somit konnte sich die Kamera nicht im Zentrum der Messpunkte befinden.\\
Da die Kamera und die Leinwand fest Montiert sind, ergibt sich auch die Problematik das der Kopf der Probanden ebenfalls nicht im Zentrum des Kamerabildes Befinden und somit immer ein Blickwinkel von unten auf das Gesicht entsteht.\\
Da die Probanden ebenfalls zwischen der Leinwand und dem Projektor standen, verdeckten diese das Bild, wodurch es manchmal passierte das der Zielpunkt im Schatten verschwand.
\subsubsection{Umgebung}
Bei der Aufzeichnung hat sich vor allem das Problem mit der ungleichmäßigen Beleuchtung bzw. dem Gegenlicht ergeben. Diesem musste entgegengewirkt werden, damit das Gesicht gut erkennbar ist. Ein Problem das auch in der realen Anwendung auftreten wird.\\
Ein weiteres allgemeines Problematik zeigt sich auch wieder bei der Auflösung des Gesichtes, somit ist eine Berechnung auf dem Gesicht zwar möglich, auf den Augen allerdings nicht.\\
Somit ergibt sich ein weiteres Problem, da im allgemeinen eine Exkursionen, der Winkelbereich der Augenbewegungen, bis etwa  $20^\circ$ stattfindet und diese nicht erfasst werden können.\\
Ein Weiterer nicht zu verachtendes Problem ist die Ruflektion vor allem auf den Brillen, von den starken Lichtquellen wie Fenster, Projektor- und dessen Bild sowie der Lampen. Auch Schatten gerade bei den Augenhöhlen erschweren die Auswertung. 
\begin{itemize}
	\item Bild für den Versuchsaufbau
	\item Typ der Kamera
\end{itemize}
