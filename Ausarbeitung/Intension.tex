\section{Intension}
\label{intension}
Die Grundlage für erfolgreiches Lernen ist die Aufmerksamkeit der Schüler und  daher ausschlaggebend für die Qualität des Unterrichtes. Das Verhalten kann eingeteilt werden in on-Task (der Schüler ist aufmerksam bei der Sache) und off-Task (der Schüler ist unaufmerksam). Allerdings ist das erfassen der Aufgabe zugewandten Aufmerksamkeit recht schwierig und verschiedene Erfassungsmethoden versuchen dies zu erreichen. Ein Vorschlag von Ehrhardt, Findeisen, Marinello und Reinhartz-Wenzel (1981) umfasst die Parameter Blickrichtung, Körperhaltung und Tätigkeit.\\
Zur Erfassung werden z.B. Fragebögen eingesetzt, die Schüler und Lehrer selbst ausfüllen oder es gibt ein Beobachter der die Aufmerksamkeit einzelner Schüler bewertet.\\
Für das \glqq Das Münchener Aufmerksamkeitsinventar (MAI)\grqq \cite{MAI_Verhaltensbeobachtung} wird beispielsweise die Kategorien \glqq ON-TASK, reaktiv/fremd-initiiert: der Schüler reagiert auf eine entsprechende Aufforderung oder Frage des Lehrers\grqq oder \glqq OFF-TASK - aktiv, interagierend, störend: Der Schüler nimmt die Lerngelegenheit nicht nur nicht wahr, sondern ist erkennbar anderweitig engagiert\grqq festgelegt. Um das Verhalten eines Schülers zu bewerten wird dieser $5s$ lange beobachtet und eine Kategorie wird zuzuordnen.\\
Bei der \glqq Videostudie zur Wirksamkeit des Unterrichtsprozesses \grqq \cite{aufmerksamkeit_Studie} wurden die Kriterien \glqq Blickkontakt zum legitimen Sprecher oder Objekt, Aktive Beteiligung an der Aufgabe, keine Ausübung anderer Tätigkeiten, keine Motorische Unruhe und keine themenferne Kommunikation\grqq festgelegt. Dann wurde der Schüler in einem ein Minuten-Intervall beobachtet und bewertet. Sind drei oder mehr Punkte erfüllt, gilt die Aufmerksamkeit des Schüler als on-Task.\\
Bei dieser Art der Auswertung gibt es allerdings Interpretationsfreiheiten, die von jedem Beobachter anders ausgelegt werden können. Außerdem ist sie sehr zeitintensiv, alleine eine einzige Beurteilung jedes einzelnen Schülers einer Klasse, etwa 30 Personen nach Vorgabe der Klassenbildung \cite{klassenteiler}, benötigt mindestens 30 Minuten. Somit kann eine Auswertung aller Schüler während einer Unterrichtsstunde schnell 15 und mehr Arbeitsstunden dauern. Um eine subjektive Bewertungen zu vermeiden, sollte außerdem ein beträchtlicher Teil der Daten von mindestens zwei Beobachtern parallel ausgewertet werden, um deren Übereinstimmung beurteilen zu können.\\
Basiert die Auswertung auf wenigen Zeitintervalle um Arbeitszeit zu sparen, wird das gesamte Verhalten eines Schülers währen des Unterrichts mit nur wenigen beobachteten Minuten beschrieben und ist entsprechend ungenau. Somit können sowohl quantitativ genaue, als auch temporal hochauflösende Daten nicht erstellt werden.\\
So kann bei grob gewählten Auswertungsintervallen nur eine Aussage über den gesamten Unterricht gemacht werden und nicht beispielsweise über einzelne Übungen oder über einen einzelnen Schüler.