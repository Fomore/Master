\section{Intension}
\label{intension}
Zur Bewertung der Qualität des Unterrichtes, wird die Aufmerksamkeit der Schüler verwendet, da zwischen beiden ein Zusammenhang besteht. Allerdings ist der Parameter Aufmerksamkeit recht schwierig zu erfassen, wodurch verschiedene Verfahren verwendet werden. Unter anderem Fragebögen die ein Schüler selbst ausfüllen sollen oder die Auswertung durch einen Beobachter der bewertet ob ein einzelner Schüler Aufmerksam (on-Task) oder nicht (off-Task) ist.\\
Für die Bewertung ob on/off-Task werden Kriterien festgelegt, wie Blickrichtung, Körperhaltung und Tätigkeit, dann wird die Person beobachtet wie diese sich verhält.\\
Bei der Videostudie zur Wirksamkeit des Unterrichtsprozesses \cite{aufmerksamkeit_Studie} wurden die Kriterien Blickkontakt zum legitimen Sprecher oder Objekt, Aktive Beteiligung an der Aufgabe, keine Ausübung anderer Tätigkeiten, keine Motorische Unruhe und keine Themenferne Kommunikation festgelegt. Dann wird immer in einem 1min Intervalle der Schüler beobachtet und bewertet. Sollte drei oder mehr Punkte erfüllen, gilt der Schüler als on-Task.\\
Diese Art der Auswertung ist recht einfach, allerdings gibt es Interpretationsfreiheiten, gerade bei den Bewertungen der Tätigkeiten, die von jedem Beobachter anders gewertet werden. Außerdem ist es sehr zeitintensiv, so werden alleine zum anschauen des Videos für jeden Schüler bei einer Klassen (30 Personen) 30 min gebraucht und sollte jeder Schüler mehrfach beobachtet werden entsprechend mehr.\\
Sollten recht wenige Zyklen durchgeführt werden, so wird das gesamte Verhalten eines Schülers in der Unterrichtsstunde mit nur wenigen beobachteten Minuten beschrieben, die Auswertung benötigt allerdings entsprechend weniger Zeit.\\
Durch eine zu geringe Auswertungsrate kann nur eine Aussage über den gesamten Unterricht gemacht werden und nicht über einzelne Übungen oder ähnliches, auch die Bewertung eines einzelnen Schülers ist nur schwer möglich.
\cite{aufmerksamkeit_Studie}