\label{intension}
Die Grundlage für erfolgreiches Lernen ist die Aufmerksamkeit der Schüler. Sie ist ein wichtiger Indikator für die momentane Aufnahmefähigkeit des Schülers und die Qualität des Unterrichts. Das Verhalten eines Schülers kann stark vereinfacht eingeteilt werden in \textit{ON-Task} (aufmerksam bei der Sache) und \textit{OFF-Task} (unaufmerksam). Allerdings ist das Erfassen von \textit{einer Aufgabe zugewandten Aufmerksamkeit} technisch schwierig, da es sich um einen kognitiven Prozess handelt der nur indirekt beobachtet werden kann. Entsprechend existieren verschiedene Erfassungsmethoden; Ein Vorschlag von Ehrhardt, Findeisen, Marinello und Reinhartz-Wenzel (1981) umfasst beispielsweise die Beurteilung von Blickrichtung, Körperhaltung und Tätigkeit.\\
Zur Erfassung werden z.B. Fragebögen eingesetzt, die Schüler und/oder Lehrer selbst ausfüllen oder ein unabhängiger Beobachter bewertet die Aufmerksamkeit einzelner Schüler anhand festgelegter Kriterien.\\
Die Zuwendung von Aufmerksamkeit kann indirekt z.B. durch eine Blickzuwendung gemessen werden (auch wenn nicht mit jeder Blickzuwendung zwangsweise eine Aufmerksamkeitszuwendung einhergehen muss, ist dies eine oftmals hinreichende Annäherung). Während eine Blickrichtungsbestimmung erstrebenswert wäre, kann auch bereits die Bestimmung der Kopforientierung als Richtungsindikator verwendet werden.\\
Im Rahmen dieser Arbeit soll untersucht werden, wie weit es technisch möglich ist Filmmaterial einer Unterrichtsstunde im Bezug auf Blickrichtungen auszuwerten und mit welchen Einschränkungen und Genauigkeiten zu rechnen ist. Daraus lassen sich Anhaltspunkte sowohl über die Auswertbarkeit existierender Daten als auch über einen optimalen Versuchsaufbau ableiten.\\
Gängige Methoden zur Bestimmung der Blickrichtung, wie beispielsweise Eye-Tracking Brillen, sind für diesen Zweck nur eingeschränkt geeignet. Zum einen ist die Anschaffung einer großen Stückzahl dieser Geräte teuer und wurde bisher nur in wenigen speziell eingerichteten Laboratorien durchgeführt wie z.B. dem TüDiLab \cite{TueDiLab}. Zum anderen sind die Geräte entweder intrusiv und haben damit ein Ablenkungspotential (Brillen) oder schränken den Aktionsradius ein (Remote Tracker mit ihrer Head-Box von üblicherweise weniger als 30x30 cm).\\
In dieser Arbeit werden die Grenzen der momentan zur Verfügung stehenden Algorithmen bestimmt. Dies liefert Anhaltspunkte für ein optimales Setup für ein größeres Experiment, um die Aufmerksamkeit einer ganzen Klasse erfassen zu können.\\
Dies betrifft vor allem die Anzahl und Position der Kameras und deren Auflösung, damit der gesamte Bereich eines Klassenzimmers voll abgedeckt und trotzdem möglichst einfach ist.\\
Wäre man in der Lage, solch eine qualitativ hochwertige Auswertung mit nur wenigen Kameras durchführen zu können, so ist der Aufbau und die Aufnahmen der Daten auch für technische Laien durchführbar.\\
Müssen dagegen viele Kameras verwendet werden, so ergeben sich verschiedene Problematiken: Alle Kameras müssen synchronisiert werden im Bezug auf die Zeit und ihre Ausrichtung zueinander, um die Ergebnisse basierend auf den einzelnen Videos miteinander abgleichen zu können. Diese Synchronisation ist bei wenigen Kameras deutlich einfacher. Außerdem müssen alle Aufzeichnungen in Echtzeit stattfinden, womit die Limitierung der Bandbreite bei Vernetzungen ebenfalls berücksichtigt werden muss und somit die Anzahl begrenzen kann.\\
Die Interpretation der Ergebnisse dieser Arbeit orientiert sich an Originalaufnahmen eines Englischunterrichtes. Diese zeigen die gesamte Klasse aus Sichtrichtung der Tafel.\\
Da für diese Aufnahmen keine Ground-Truth Daten (exakte Position der Schüler/Kamera usw.) bekannt sind, wird eine Reihe von Versuchen durchgeführt, um die einzelne Aspekte und Problemstellungen der Datenanalyse dieser Videos genauer zu untersuchen.\\
In den ersten Versuchen wurden verschiedene Aufnahmen verwendet, um die Auswirkung von Position und Blickziel relativ zur Kamera zu testen. Für den anschließenden Versuch wurde ein bewegliches Blickziel erstellt, um eine kontinuierliche Messwerterfassung zu testen.