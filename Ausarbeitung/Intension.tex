\section{Intension}
\label{intension}
Die Grundlage für erfolgreiches Lernen ist die Aufmerksamkeit der Schüler und  daher ausschlaggebend für die Qualität des Unterrichtes. Das Verhalten kann eingeteilt werden in on-Task (der Schüler ist aufmerksam bei der Sache) und off-Task (der Schüler ist unaufmerksam). Allerdings ist das erfassen der Aufgaben zugewandte Aufmerksamkeit recht schwierig und unterschiedliche Erfassungsmethoden versuchen dies zu bewerten. So werden z.B. Fragebögen eingesetzt, die Schüler und Lehrer selbst ausfüllen oder es gibt ein Beobachter der die Aufmerksamkeit einzelner Schüler bewertet.\\
Für die Bewertung von on/off-Task werden bei einer Studie verschiedene Kriterien festgelegt, wie Blickrichtung, Körperhaltung und Tätigkeit um damit das tatsächlich beobachtete Verhalten der Schüler zu bewerten.\\
Bei der \glqq Videostudie zur Wirksamkeit des Unterrichtsprozesses \grqq \cite{aufmerksamkeit_Studie} wurden z.B. die Kriterien Blickkontakt zum legitimen Sprecher oder Objekt, Aktive Beteiligung an der Aufgabe, keine Ausübung anderer Tätigkeiten, keine Motorische Unruhe und keine themenferne Kommunikation festgelegt. Dann wurde im ein Minuten-Intervall der Schüler beobachtet und bewertet. Sind drei oder mehr Punkte erfüllt, gilt die Aufmerksamkeit des Schüler als on-Task.\\
Bei dieser Art der Auswertung gibt es allerdings Interpretationsfreiheiten die von jedem Beobachter anders ausgelegt werden können. Außerdem ist diese Art der Bewertung sehr zeitintensiv. Alleine eine einzige Beurteilung jedes einzelnen Schülers einer Klasse, etwa 30 Personen nach Vorgabe der Klassenbildung \cite{klassenteiler}, benötigt dies mindestens 30 Minuten. Somit kann eine Auswertung aller Schüler während einer Unterrichtsstunde schnell 15 und mehr Arbeitsstunden dauern. Um subjektive Bewertungen zu vermeiden sollte außerdem ein beträchtlicher Teil der Daten von mindestens zwei Beobachtern parallel ausgewertet werden, um deren Übereinstimmung beurteilen zu können.\\
Basiert die Auswertung auf wenigen Zeitintervalle um Zeit zu sparen, so wird das gesamte Verhalten eines Schülers währen des Unterrichts mit nur wenigen beobachteten Minuten beschrieben und ist entsprechend ungenau. Wodurch sowohl quantitativ genaue, als auch temporal hochauflösende Daten nicht erstellt werden können.\\
So kann bei zu grob gewählten Auswertungsintervallen nur eine Aussage über den gesamten Unterricht gemacht werden und nicht beispielsweise über einzelne Übungen oder über einen einzelnen Schüler.\\
\cite{aufmerksamkeit_Studie}