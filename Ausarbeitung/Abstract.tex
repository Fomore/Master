\begin{abstract}
\section*{Zusammenfassung}
Der übliche Aufbau zur Analyse von Gesichter ist, das verwenden eines Messgerät pro Person und Merkmal, wie Beispielsweise Eye-Tracking Brillen. Soll eine Auswertung auf mehreren Probanden gleichzeitig durchgeführt werden, so ist die Verwendung von weniger Geräten einfacher in der Handhabung.\\
Im Rahmen der Aufmerksamkeitsmessung im Unterricht, soll eine automatisierte Messung der Aufmerksamkeit einer ganzen Klassen durchgeführt werden. Dabei soll der Unterricht durch die Messinstrumente möglichst wenig beeinflusst werden.\\
Um Anhaltspunkte eines effizienten Aufbaus des eigentlichen Versuchs zu erhalten, sollen die Grenzen und Qualität bei der Gesichtsanalyse basierend auf Bildmaterial einer einzelnen Kamera aufgezeigt werden. Diese Kamera wird fest montiert um eine Frontalaufnahme aller Probanden gleichzeitig zu erhalten, wobei die gesamte Klasse im Fokus der Kamera liegt. Durch diesen Aufbau ergibt sich die Problematik von den sehr unterschiedlichen Distanzen zur Kamera und daraus resultierend die dargestellte Größe aller Probanden im Bild.\\
Um alle Probanden im Frame bewerten zu können, werden zuerst die einzelnen Gesichter im Bild gesucht, den Probanden zugeordnet und aufbereitet. Anschließend werden die Gesichter analysiert um ihre Position und Orientierung zu bestimmen. Die Augenregion wird zusätzlich behandelt, um genauere Ergebnisse bei der Bestimmung der Blickrichtung zu erhalten.\\
Die Versuche haben ergeben, das mit den heutigen HD Kameras eine gleichzeitige Analyse von mehreren Probanden im selben Frame möglich ist, die sich auf der Fläche eines üblichen Klassenzimmers verteilen. Für die Analyse kann meist nur auf den Gesichtern gearbeitet werden, da für die Bestimmung der Blickrichtung zu wenige Informationen in den kleinen Bildern vorhanden sind. Außerdem Können mit den verwendeten Verfahren nur Gesichter bis zu einem Winkel von $45^\circ$ relativ zur Kamera erfasst werden.
\end{abstract}
