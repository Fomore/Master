\begin{abstract}
\section*{Zusammenfassung}
Aufmerksamkeit ist eine Grundvoraussetzungen für erfolgreiches Lernen in der Schule. Eine objektive Quantifizierung der Aufmerksamkeit eines Schülers oder einer ganzen Klasse könnte dabei helfen Lern- und Lehrprozesse besser zu verstehen und zu optimieren.\\
Aufmerksamkeit zu messen ist allerdings nur indirekt möglich, z.B. per Eye-Tracking und Beobachtung des Blickverhaltens. Mit dem momentanen Stand der Technik muss für jeden Schüler ein dezidiertes Gerät einsetzt werden. Dieser Prozess ist extrem teuer, schränkt die Probanden in ihrer Bewegungsfreiheit ein und ist in der Auswertung aufwendig.\\
Diese Arbeit untersucht die technische Machbarkeit eines effizienten Aufbaus zur Aufmerksamkeitsanalyse einer Menschengruppe. Dazu werden die Grenzen und erreichbaren Genauigkeiten einer Gesichtsanalyse basierend auf Bildmaterial einer einzelnen Kamera ausgelotet. Durch diesen Aufbau ergibt sich die Problematik dass Personen sich in sehr unterschiedlicher Distanz zur Kamera aufhalten können und, daraus resultierend, unterschiedliche Abbildungseigenschaften, wie z.B. die Anzahl der Pixel die das Gesicht einer Person im Bild darstellen.\\
Um alle Probanden in einem Kamerabild hinsichtlich Aufmerksamkeitszuwendung bewerten zu können, werden zuerst die einzelnen Gesichter im Bild detektiert und eindeutig einem Probanden zugeordnet. Im Folgeschritt wird der so gewonnene Bildausschnitt aufbereitet. Eine nachfolgende Analyse des abgebildeten Gesichts erlaubt die Bestimmung von Position und Orientierung im Raum. Die Augenregion ist für die gerichtete Aufmerksamkeit besonders aussagekräftig und wird deshalb gesondert behandelt, um genauere Ergebnisse bei der Bestimmung der Blickrichtung zu erhalten.\\
Diese Arbeit stellt mehrere Referenzmessungen vor, bei denen mithilfe bekannter Positionen und Blickwinkeln im Raum die Güte der algorithmischen Bestimmung von gerichteter Aufmerksamkeit quantifiziert werden kann. Alle dabei angewandten Szenarien beziehen sich auf die Dimensionen und Anwendungsszenarien in einem typischen Klassenzimmer.\\
Die bestimmten Genauigkeitswerte haben ergeben, dass die Bestimmung der Kopforientierung mit durchschnittlich $5^\circ$ Grad, die der Blickrichtung mit $10^\circ$ Grad möglich ist.\\
Für die Analyse kann meist nur auf der Kopforientierung gearbeitet werden, da für die Bestimmung der Blickrichtung in den - mit der Entfernung von der Kamera schnell kleiner werdenden - abgebildeten Kopfregionen im Bild zu wenige Informationen vorhanden sind. Abgeleitet aus den Ergebnissen der einzelnen Verfahren sollte eine Auswertung der Augen bis zu einer Distanz von maximal $4m$ möglich sein. Dieser Wert konnte im Test unter Realbedingungen allerdings nicht erreicht werden.
Die verwendeten Verfahren zur Gesichtsanalyse (Landmarkenbestimmung und Positionserrechnung) sind außerdem auf einen Winkel von $45^\circ$ relativ zur Kamera beschränkt.\\
Die in dieser Arbeit gewonnenen Erkenntnisse bedeuten, dass für eine aussagekräftige Auswertung entweder nur Kopforientierungen herangezogen werden können oder hochauflösendere Aufnahmen angefertigt werden müssen, wobei eine Kamera einen Bereich von 5 Metern Tiefe und $45^\circ$ abdecken kann.
\end{abstract}
