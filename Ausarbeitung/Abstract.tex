\begin{abstract}
\section*{Zusammenfassung}
Aufmerksamkeit ist einer der Grundvoraussetzungen für erfolgreiches Lernen in der Schule. Eine objektive Quantifizierung der Aufmerksamkeit eines Schülers oder einer ganzen Klasse könnte dabei helfen Lern- und Lehrprozesse besser zu verstehen und zu verbessern.\\
Eine technische Messung der Aufmerksamkeit ist nur indirekt möglich, z.B. per Eye-Tracking Brille um des Blickverhaltens zu beobachten. Mit dem momentanen Stand der Technik muss für jeden Schüler ein einzelnes Gerät einsetzt werden. Dieser Prozess ist extrem teuer, störend für die Probanden und in der Auswertung aufwendig.\\
Diese Arbeit untersucht die technische Machbarkeit eines effizienten Aufbaus zur Aufmerksamkeitsanalyse einer Menschengruppe. Dazu werden die Grenzen und möglichen Genauigkeiten einer Gesichtsanalyse basierend auf Bildmaterial einer einzelnen Kamera ausgelotet.\\
Durch diesen Aufbau ergibt sich die Problematik sehr unterschiedlicher Distanzen zwischen den zu messender Person zur Kamera und, daraus resultieren, unterschiedliche Abbildungseigenschaften, wie z.B. die Anzahl an Pixel die das Gesicht der Person im Bild darstellen.\\
Um alle Probanden in einem Kamerabild bewerten zu können, werden zuerst die einzelnen Gesichter im Bild detektiert eindeutig einem, den Probanden zugeordnet und dann aufbereitet. Durch eine folgende Analyse des abgebildeten Gesichts lassen sich dessen Position und Orientierung im Raum bestimmen. Die Augenregion ist für die gerichtete Aufmerksamkeit besonders aussagekräftig und wird deshalb zusätzlich gesondert behandelt, um genauere Ergebnisse bei der Bestimmung der Blickrichtung zu erhalten.\\
Die Versuche haben ergeben, das mit den heutigen hochauflösenden Kameras eine gleichzeitige Analyse mehrerer Probanden möglich ist, wenn sie sich auf der Fläche eines üblichen Klassenzimmers verteilen.\\
Für die Analyse kann meist nur auf der Kopforientierung gearbeitet werden, da für die Bestimmung der Blickrichtung zu wenige Informationen in den kleinen Bildern vorhanden sind. Abgeleitet aus den Ergebnisse der einzelnen Verfahren sollte eine Auswertung der Augen auf einer Distanz von $4m$ möglich sein, konnte im Test unter Realbedingungen allerdings bei weitem nicht erreicht werden.
Die verwendeten Verfahren zur Gesichtsanalyse (Landmarkenbestimmung und Positionserrechnung) sind auf einen Winkel von $45^\circ$ relativ zur Kamera beschränkt.
\end{abstract}
