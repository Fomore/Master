\section{Abstarct}
Der übliche Aufbau zur Analyse von Gesichter ist, das verwenden von eines Messgerät pro Person und Merkmal, wie Beispielsweise Eye-Tracking Brillen. Soll eine Auswertung auf mehreren Probanden gleichzeitig durchgeführt werden, so ist die Verwendung von weniger Geräten einfacher.\\
Im Rahmen der Aufmerksamkeitsmessung im Unterricht, soll eine Messung der Aufmerksamkeit einer ganzen Klassen durchgeführt werden. Um Anhaltspunkte eines effizienten Aufbaus des eigentlichen Versuchs zu erhalten, sollen die Grenzen und Qualität bei der Gesichtsanalyse basierend auf Bildmaterial einer einzelnen Kamera aufgezeigt werden. Diese wird fest montiert um eine Frontalaufnahme aller Probanden gleichzeitig zu erhalten, wobei die gesamte Klasse im Fokus der Kamera liegt. Durch diesen Aufbau ergibt sich die Problematik mit sehr unterschiedlichen Distanzen zur Kamera und der dargestellten Größe aller Probanden im Bild.\\
Um alle Probanden im Frame zu analysieren, werden zuerst die einzelnen Gesichter im Bild gesucht, den Probanden zugeordnet und aufbereitet. Anschließend werden die Gesichter analysiert um ihre Position und Orientierung zu bestimmen. Die Augenregion wird zusätzlich behandelt, um genauere Ergebnisse bei der Bestimmung der Blickrichtung zu erhalten.\\
Die Versuche haben ergeben, das mit den heutigen HD Kameras eine gleichzeitige Analyse von mehreren Probanden im selben Frame möglich ist, die auf der Fläche eines üblichen Klassenzimmers verteilt sind. Für die Analyse kann meist nur auf den Gesichtern gearbeitet werden, da für die Bestimmung der Blickrichtung zu wenige Informationen in den kleinen Bildern vorhanden sind.