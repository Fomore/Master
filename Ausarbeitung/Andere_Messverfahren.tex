\section{Automatisierte Messverfahren von Aufmerksamkeitsparametern}
\label{other_Messverfahren}
Eine Möglichkeit zur automatischen Erfassung der Aufmerksamkeit wird in \glqq Real time detection of driver attention\grqq\cite{driverAttention} vorgestellt. Bei diesem Verfahren ist eine Kamera direkt von vorn auf den Fahrer gerichtet. Es wird anhand der Kopf und Augenposition bewertet, ob dieser aktiv auf den Verkehr achtet.\\
Für die aktuelle Problemstellung ist dieses Verfahren allerdings nicht geeignet, da es auf einen einzigen Probanden in nächster Nähe zur Kamera ausgelegt ist. Außerdem wird die Zuordnung anhand von festen Regeln bezüglich der Kopfposition durchgeführt, und ist daher nicht dynamisch genug für die Auswertung einer ganzen Schulkasse.\\ 
Ein weiteres ähnliches Verfahren wird in \glqq AggreGaze\grqq \cite{AggreGaze} präsentiert. Dabei wird eine einzige Kamera fest auf einem Bildschirm montiert, um die Blickrichtung der Passanten auf den Bildschirm zu bestimmen. Dieses Verfahren arbeitet allerdings nur auf einem recht begrenzten Bereich in dem sich die Probanden aufhalten dürfen und das Ziel der Blicke ist sehr nahe an der Kamera. Dies entspricht zwar dem grundsätzlichen Aufbau, allerdings muss der Erfassungsbereich deutlich größer ausfallen.