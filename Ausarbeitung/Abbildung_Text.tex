\label{Abbildungen}
In diesem Abschnitt werden weitere Diagramme dargestellt um einen besseren Eindruck über die Messergebnisse zu erhalten.
\subsection*{Boxplot}
Folgende Angaben gelten für alle dargestellten Boxplots.
\begin{itemize}
	\item Die schwarze Mittellinie in der Box zeigt den Median der Messwerte an.
	\item Die Box beschreibt das obere und untere Quartal der Messwerte, also jene Stellen an denen $25\%$ der Messwerte größer bzw. kleiner sind als der gewählte dargestellte Wert.
	\item \glqq Die Whiskers (gestrichelte Linie) zeigen das Maximum bzw. Minimum einer Verteilung, sofern diese nicht mehr als das 1,5-fache des Interquartilabstands vom Median abweichen\grqq 
	\cite{wiki_Boxplot}
	\item Alle Ausreißer wurden zwecks Übersichtlichkeit weggelassen
	\item Die eingezeichnete horizontale Linie stellt den Median der Messwerte aus Skalierung 1 dar, die Beschriftung gibt den Median an.
\end{itemize}
\subsection*{Anzahl der Messwerte}
Um eine Übersicht über die Anzahl der Messwerte zu erhalten ein Überblick:
\subsubsection*{Biwi Random Forests for Real Time 3D Face Analysis \cite{database_Face_Ori}}
Alle Darstellungen und Auswertungen bezüglich der verschiedenen Skalierungsverfahren haben folgende Anzahl an Messwerten bei den angegebenen Skalierungen.\\
Die geringe Anzahl der Messwerte bei $0,04$ kann keine begründete Aussage gemacht werden, außerdem weichen die Werte so sehr von de anderen ab, das sie als False-True erkannten Gesichtern angesehen werden können
$[$\autoref{img_Rot_Max}, \autoref{img_X_Pos_Skal}, \autoref{img_Y_Pos_Skal}, \autoref{img_Z_Pos_Skal}, \autoref{img_X_Rot_Skal}, \autoref{img_Y_Rot_Skal}, \autoref{img_Z_Rot_Skal}$]$\\
\begin{tabular}{|l|c|c|c|c|c|c|c|c|c|c|}
	\hline 
	&0,04&0,07&0,1&0,13&0,16&0,19&0,22&0,25&0,28&0,31-1\\
	\hline 
	Bicubic&3&1190&4545&5888&7147&8329&8991&9439&9561&$9600-9800$\\
	\hline 
	Lanczos&3&1224&4206&5696&6941&8224&8958&9400&9548&$9700-9800$\\
	\hline 
	Linear&1&776&3935&5439&6851&8019&8625&9107&9313&$9400-9800$\\
	\hline 
	Nearest-Neighbor&0&0&0&193&2081&4374&5976&7825&8595&$9200-9800$\\ 
	\hline 
\end{tabular} 
\subsubsection*{Augen-Datensatz \cite{database_Eye}}
Die einzelnen abgebildeten Boxplots basierend auf dem Augen-Datensatz \cite{database_Eye} besitzen mindestens 10.000 Messwerte.\\
$[$\autoref{ElSe_Gray_Zentrum}, \autoref{ElSe_Gray_Pupille}, \autoref{ElSe_Gray_Iris}, \autoref{ElSe_scall}$]$
\subsubsection*{Aufmerksamkeitsmessung - Werte im Versuch}
Für diese Auswertung ergeben sich folgende Werteverteilungen, dabei wurde der durchschnittliche Fehler (Mean-Error) über den gesamten Datensatz bestimmt. Die Einteilung der Boxplots erfolgte durch das natürliche Runden auf die Zehnerstelle der wahren X bzw. Y Werte. Die ungleichmäßige Verteilung der Messwerte liegt am Versuchsaufbau und ein gewisser Teil des Rauschens kann auf den Wertebereich zurückgeführt werden, die in einer Box zusammengefasst wurden ($\pm 5^\circ$)\\
Dies Angaben bezeihen sich auch  \autoref{graph_VideoSkalierung} und \autoref{graph_VideoSkalierung_Err})\\
Skalierung 1:\\
X-Mean-Error = $8,971^\circ$; Y-Mean-Error = $10,08^\circ$; EyeAVG-X-Mean-Error = $17,49^\circ$\\
\begin{tabular}{|l|c|c|c|c|c|c|c|c|c|}
	\hline 
	Winkel [Grad]&-40&-30&-20&-10&0&10&20&30&40\\
	\hline 
	X-Rotation: Anzahl&3115&1363&1278&1297&4142&1189&1343&1304&3449\\ 
	\hline 
	Y-Rotation: Anzahl&444&3328&1920&5692&2215&1804&3077&&\\
	\hline
\end{tabular}\\\\
Skalierung $0,5$:\\
X-Mean-Error = $8,927^\circ$; Y-Mean-Error = $10,07^\circ$; EyeAVG-X-Mean-Error = $17,07^\circ$\\
\begin{tabular}{|l|c|c|c|c|c|c|c|c|c|}
	\hline 
	Winkel [Grad]&-40&-30&-20&-10&0&10&20&30&40\\
	\hline 
	X-Rotation: Anzahl&2420&1068&1002&1002&3217&932&1070&1023&2720\\ 
	\hline 
	Y-Rotation: Anzahl&222&2649&1506&4372&1819&1415&2471&&\\
	\hline
\end{tabular}\\\\
Skalierung $0,25$:\\
X-Mean-Error = $8,742^\circ$; Y-Mean-Error = $9,772^\circ$; EyeAVG-X-Mean-Error = $17,07^\circ$\\
\begin{tabular}{|l|c|c|c|c|c|c|c|c|c|}
\hline 
Winkel [Grad]&-40&-30&-20&-10&0&10&20&30&40\\
\hline 
X-Rotation: Anzahl&2471&1074&1012&1018&3283&950&1077&1047&2749\\ 
\hline 
Y-Rotation: Anzahl&222&2753&1536&4452&1831&1417&2470&&\\
\hline
\end{tabular}\\\\
Skalierung $0,1$:\\
X-Mean-Error = $9,899^\circ$; Y-Mean-Error = $10,14^\circ$; EyeAVG-X-Mean-Error = $21,31^\circ$\\
\begin{tabular}{|l|c|c|c|c|c|c|c|c|c|}
\hline 
Winkel [Grad]&-40&-30&-20&-10&0&10&20&30&40\\
\hline 
X-Rotation: Anzahl&2466&1064&1002&1018&3283&950&1074&1047&2727\\ 
\hline 
Y-Rotation: Anzahl&222&2734&1517&4451&1829&1417&2461&&\\
\hline
\end{tabular}\\\\
Skalierung 0.05:\\
X-Mean-Error = $12,48^\circ$ Y-Mean-Error = $12,48$; EyeAVG-X-Mean-Error = $21,48^\circ$\\
\begin{tabular}{|l|c|c|c|c|c|c|c|c|c|}
\hline 
Winkel [Grad]&-40&-30&-20&-10&0&10&20&30&40\\
\hline 
X-Rotation: Anzahl&1231&575&550&589&1859&557&607&530&1335\\ 
\hline 
Y-Rotation: Anzahl&&1151&661&2419&1217&817&1568&&\\
\hline
\end{tabular}