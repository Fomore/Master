\section{Problemstellung}
\label{Problemstellung}
Im Rahmen dieser Arbeit sollen die Grenzen aufgezeigt werden, wie weit es technisch möglich ist Filmmaterial einer einzigen Kamera im Bezug auf Blickrichtungen bzw. Ausrichtung des Gesichts Auszuwerten und mit welchen Einschränkungen und Genauigkeiten zu rechnen sind, wenn im Bild eine gesamte Klasse dargestellt ist.\\
Eine automatisierte Auswertung der Blickrichtung wäre erstrebenswert, da dies einer der wichtigsten Indikatoren für gerichtete Aufmerksamkeit ist. Ist dies nicht möglich, reicht eine Bestimmung der Kopforientierung aus, da diese in etwa der Blickrichtung entspricht.\\
Die Messung soll den Unterricht möglichst wenig beeinträchtigen, wodurch hierfür üblicherweise verwendete Geräte, wie z.B. Eye-Tracking Brillen, nicht verwendet werden können. Zum einen ist die Anschaffung einer großen Stückzahl dieser Geräte teuer und wurde bisher nur in wenigen speziell eingerichteten Laboratorien durchgeführt wie z.B. TüDiLab \cite{TueDiLab}. Zum anderen sind die Geräte entweder Ablenkend (Brillen) oder schränken den Aktionsradius ein (Remote Tracker).\\
Die hier bestimmten Grenzen ergeben Anhaltspunkte, wie das Setup (Anzahl und Position der Kameras und deren Auflösung) für ein größeres Experiment aussehen muss, um die Aufmerksamkeit einer ganzen Klasse zu erfassen. Wären man in der Lage, solch eine qualitativ hochwertige Auswertung mit nur wenigen Kamera durchführen zu können, so ist der Aufbau und die Aufnahmen der Daten auch für technische Laien durchführbar.\\
Eine Möglichkeit für das automatische Erfassen der Aufmerksamkeit wird in \glqq Real time detection of driver attention\grqq\cite{driverAttention} vorgestellt. Bei diesem Verfahren ist eine Kamera direkt von vorn auf den Fahrer gerichtet und anhand der Kopf und Augenposition bewertet, ob dieser aktiv auf den Verkehr achtet.\\
Ein weiteres dazu passendes Verfahren wird in \glqq AggreGaze\grqq \cite{AggreGaze} präsentiert, dabei wird eine einzige Kamera fest auf einem Bildschirm montiere, um die Blickrichtung der Passanten auf den Bildschirm zu bestimmen, dieses Verfahren arbeitete allerdings nur auf einem recht begrenzen Bereich in dem sich die Probanden aufhalten dürfen und das Ziel der Blicke ist sehr nahe an der Kamera.\\
Um die Machbarkeit der Analyse zu untersuchen, wurden verschiedene Videoaufnahmen ausgewertet. Unter anderem zwei Originalaufnahmen eines Englischunterrichtes, diese zeigen die gesamte Klasse aus Richtung der Tafel und liefern Eindrücke über die verschiedenen Probleme. Allerdings besitzen sie nur eine sehr geringe Auflösung ($640\times 480$ Pixel).\\
Für die Vorversuche wurde eine Actioncam verwendet um erste Eindrücke, bezüglich der Auswirkung von Position und Zielpunkt auf die Auswertung zu erhalten.\\
Um mehr Messwerte für unterschiedlichen Zielpunkte zu erhalten wurde ein weiterer Video-Datensatz mit der Logitech-Webcam erstellt, bei der die Probanden ein bewegtes Ziel beobachten sollten. Damit besser bewertet werden kann, wie mit dem Verfahren das Ziel der Aufmerksamkeit bestimmt werden kann.\\
Außerdem soll die Anwendung auch auf schon vorhanden Aufnahmen des Englischunterrichtes arbeiten.