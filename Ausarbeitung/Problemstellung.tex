\section{Problemstellung}
\label{Problemstellung}
Die aktuellen Verfahren zur Analyse von Aufmerksamkeit im Unterricht werden meist äquivalent zu \autoref{intension} durchgeführt. Der jeweilige Schüler wird von einer Person beobachtet und dann nach bestimmten Kriterien bewertet. In die Bewertung fließt allerdings auch die Meinung/Auffassung des Bewerters ein und ist daher nicht objektiv.\\
Diese Art der Auswertung ist recht ungenau und Arbeitsintensiv, da sie von einer Person ausgeführt werden muss. Ein wichtiger Parameter ist die Blickrichtung des einzelnen Schülers, da sie meist dorthin gerichtet ist, wo auch die Aufmerksamkeit liegt.\\
Ziel dieser Arbeit ist es nun, mit möglichst geringem Aufwand an Hardware eine Bestimmung der Blickrichtung einer ganzen Klasse vorzunehmen. Die Messung soll den Unterricht möglichst wenig beeinträchtigen, wodurch Eye-Tracking Brillen nicht verwendet werden, wegen den Kosten und der Ablenkung. Auch der Aufbau soll recht einfach und für Laien anwendbar sein, somit wird nur eine einzige festmontierte Kamera vor der Klasse eingesetzt.\\
Für diese Anforderungen soll nun ein Verfahren entwickelt werden, mit dem es möglich ist, das Filmmaterial von einer gesamte Klasse auf einmal auszuwerten, um von allen Personen die Blickrichtungen bzw. die Gesichtsorientehrung während einer Schulstunde zu bestimmen.\\
\cite{MAI_Verhaltensbeobachtung}