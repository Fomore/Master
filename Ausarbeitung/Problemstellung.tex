\section{Problemstellung}
\label{Problemstellung}
Das aktuelle Verfahren zur Analyse von Aufmerksamkeit im Unterricht sieht wie folgt aus. Zyklisch wird immer ein Schüler für eine Minute beobachtet und diese dann bewertet was der Schüler getan hat. Dieses Verhalten wird nun als Gesamtergebnis des Schülers verwendet und als Bewertung des Unterrichtes über die gesamte Klasse gemittelt.\\
Diese Art der Auswertung ist recht ungenau und Arbeitsintensiv, da sie von einer Person ausgeführt wird. Ein wichtiger Parameter ist die Blickrichtung des einzelnen Schülers, da sie meist dorthin gerichtet ist, wo auch die Aufmerksamkeit liegt.\\
Ziel ist en nun mit möglichst geringem Aufwand an Hardware eine Bestimmung der Blickrichtung einer ganzen Klasse vorzunehmen. Die Messung soll den Unterricht möglichst wenig beeinträchtigen, wodurch Eye-Tracking Brillen nicht verwendet werden, wegen Kosten und Ablenkung. Auch der Aufbau soll recht einfach und für Laien anwendbar sein, somit wird nur eine festmontierte Kamera vor der Klasse eingesetzt.\\
Dazu soll ein Verfahren entwickelt werden, mit dem es möglich ist das Filmmaterial von einer gesamte Klasse auf einmal auszuwerten, um von allen die Blickrichtungen während einer Schulstunde zu bestimmen.\\
\cite{MAI_Verhaltensbeobachtung}
\begin{itemize}
	\item Es sieht für die Beurteilung eines Verhaltens als on- oder off-task drei Dimensionen vor: Blickrichtung, Körperhaltung und Tätigkeit.\\
	keine simultane Kodierung des unterrichtlichen Kontextes vorgesehen ist und daß über die Validität des Verfahrens nichts bekannt ist.\\
	von Ehrhardt, Findeisen, Marinello und Reinhartz-Wenzel (1981)
	\item Münchener Aufmerksamkeitsinventar (MAI)\\
	Es wird ein festes Zeitintervall von jeweils fünf Sekunden für die Kodierung des Schülerverhaltens und des jeweiligen Unterrichtskontextes
	vorgegeben.\\
	vgl. Helmke, 1986
	\item MAI:\\
	Die Beobachtungen in einer Unterrichtsstunde umfassen mehrere Durchgänge ("Zyklen")\\
	in Jedem Zyklus alle Schüler in einer vorher festgelegten Reihenfolge für je 5 Sekunden beobachtet und dann Aufmerksamkeits- und Kontextkodierungen.\\
	Nach jeweils vier vollständigen Zyklen folgt eine zweiminütige Pause
	\item Die Bedeutung der Aufgaben für das Beteiligungsverhalten der Schüler - Eine Videostudie zur Wirksamkeit des Unterrichtsprozesses\\ Zeitintervall $1min$ auf sichtbare einzelne Schüler\\
	Kodiersystems dienten die Arbeiten von Helmke und Kollegen 
	(Helmke, 1988; Helmke \& Renkl, 1992)\\
	Beobachtungssystemen: molecular composite-Konzept (Hoge, 1985)\\
	fünf Verhaltensindikatoren bei $\ge 3 \rightarrow $ on task
	\begin{itemize}
		\item Blickkontakt zum legitimen Sprecher oder Objekt
		\item Aktive Beteiligung an der Aufgabe
		\item keine Ausübung anderer Tätigkeiten
		\item keine Motorische Unruhe
		\item keine Themenferne Kommunikation
	\end{itemize}
\end{itemize}
