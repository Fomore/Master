\section{Problemstellung}
\label{Problemstellung}
Ziel dieser Arbeit ist es, eine automatisierte Auswertung der Blickrichtung, einem der wichtigsten Indikatoren für gerichtete Aufmerksamkeit, auf einer ganzen Klasse zu bestimmen.
Die Messung soll den Unterricht möglichst wenig beeinträchtigen, wodurch hierfür üblicherweise verwendete Geräte, wie z.B. Eye-Tracking Brillen, nicht verwendet werden können.
Zum einen ist die Anschaffung einer großen Stückzahl dieser Geräte teuer und wurde bisher nur in wenigen speziell eingerichteten Laboratorien durchgeführt (TüDiLab \cite{TueDiLab}) Zum anderen sind die Geräte entweder Ablenkend (Brillen) oder schränken den Aktionsradius ein (Remote Tracker).
Wären wir in der Lage solch eine Auswertung mit nur einer einzigen Kamera durchführen zu können, so ist der Aufbau und die Aufnahmen auch für technische Laien durchführbar.\\
Diese Arbeit untersucht, wie weit es technisch möglich ist das Filmmaterial einer Kamera, das die gesamte Klasse aufzeichnet, Auszuwerten im Bezug auf Blickrichtungen bzw. Ausrichtung des Gesichts und mit welchen Einschränkungen und Genauigkeiten zu rechnen ist.\\
\cite{MAI_Verhaltensbeobachtung}