Die größte Problematik bei der Auswertung einer ganzen Schulklasse ist, dass immer wieder Teile der Gesichter verdeckt werden, sei es durch den Arm eines anderen Schülers, die Frisur oder völlig verdeckt durch den Lehrer und ähnliches.\\
Dieser Problematik kann mit mehreren Kameras entgegen gewirkt werden, die beispielsweise an der Seite der Tafel platziert sind. Dies bietet neben der Möglichkeit einer 3D-Rekonstruktion der Szene auch die Chance das Gesicht vollständig zu erfassen.\\
Durch den großen Bereich in dem das Verfahren funktioniert ist die Positionswahl der Kameras recht frei und kann so gewählt werden, dass sie die gesamte Klasse erfassen, selten etwas verdeckt und der Unterricht dadurch wenig beeinflusst wird. Aus messtechnischer Sicht wäre die ideale Position der Kamera im Zentrum vor der Klasse, so dass die Hauptblickrichtung der Schüler in Richtung Kamera verläuft. Diese Stelle kann jedoch nicht verwendet werden da diese Position für den Unterricht (Tafel/Lehrer) benötigt wird.\\
Für die hinteren Reihen ist der Einsatz von zusätzlichen Kameras zu empfehlen, da diese Schüler recht klein dargestellt und oft durch die vorderen Reihen verdeckt werden, wenn sie von einer Kamera erfasst werden, die vor der Klasse aufgestellt wurde.\\
Mit entsprechend hochauflösenden Kameras können auch bessere Resultate auf größeren Distanzen erzielt werden, wobei ein große Unterschied eher ausbleiben wird, sollte die gesamte Klasse auf einmal aufgenommen werden, da die Schüler recht weit verteilt sind und die Auflösung durch den großen Distanzunterschied begrenzt ist.\\ 
Als Messinstrument ist eine oder mehrere Videokameras sinnvoll, da die Qualität der Auswertung auf Videos die begrenze Auflösung bei weitem wettmacht.\\
Für eine Auswertung der Aufmerksamkeit ist die erreichte Genauigkeit ausreichend, die Tendenzen sind klar erkennbar und können entsprechend interpretiert werden.\\
Da der große Erfassungsbereich nur auf Videos erreicht wird, wäre es von Vorteil, die Detektion und das Tracking soweit zu ergänzen, dass auf Profilbildern gearbeitet werden kann um Landmarks zu erkennen. Somit kann das Tracking auch begonnen werden, wenn die Probanden nicht grob in Richtung Kamera blicken und ist gegenüber Drehungen robuster.\\
Auch der Einsatz von Weitwinkelobjektiven kann nicht empfohlen werden, da zwar mit ihrer Hilfe die gesamte Klasse erfasst werden kann, aber sehr viele Bereiche im Kamerabild nur Umgebung zeigen und die Schüler entsprechend klein dargestellt sind. Eine fokussiertere Kamera würde zwar weniger Schüler erfassen, diese werden allerdings deutlich größer dargestellt und die Kamera kann passend zur Position der Schüler aufgestellt werden.\\
Abschließend kann die Aussage getroffen werden, das es mit dem aktuellen Stand der Technik möglich ist, mehrere Personen mit nur einer einzigen Kamera soweit zu Analysieren, das eine Aussage über ihre Blickrichtung gemacht werden kann.