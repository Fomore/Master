\section{Verbesserung}
Die größte Problematik bei der Auswertung einer ganzen Schulklasse, ist, dass immer wieder Teile der Gesichter verdeckt werden, sei es durch den Arm eines anderen Schülers oder der Frisur oder völlig verdeckt durch den Lehrer und ähnliches.\\
Dieser Problematik kann entgegen gewirkt werden, indem mehrere Kameras verwendet werden die beispielsweise an der Seite der Tafel platziert sind. Dies bietet neben der Möglichkeit einer 3D-Rekonstruktion der Szene auch die Chance das Gesicht vollständig erfasst wird.\\
Durch den großen Bereich in dem das Verfahren funktioniert ist die Positionswahl der Kameras recht frei und kann so gewählt werden, dass sie die gesamte Klasse erfassen, selten verdeckt und den Unterricht wenig beeinflusst.\\
Für die hinteren Reihen ist der Einsatz von zusätzlichen Kameras zu empfehle, da diese Schüler recht klein dargestellt und oft durch die vorderen Reihen verdeckt werden, sollen sie von einer Kamera erfasst werden, die vor dar Klasse aufgestellt wurde.\\
Für eine Auswertung der Aufmerksamkeit ist die erreichte Genauigkeit ausreichend, die Tendenzen sind klar erkennbar und können entsprechend interpretiert werden.\\
Da der große Erfassungsbereich nur auf Videos erreicht wird, wäre es von Vorteil die Detektion und Tracking soweit zu ergänzen, dass auf Profilbilder gearbeitet werden kann um Landmarks zu erkennen. Somit kann das Tacking auch begonnen werden, wenn die Probanden nicht grob in Richtung Kamera blicken. Das ist gegenüber einer Drehung robuster.\\
Auch der Einsatz von Weitwinkelobjektive kann nicht empfohlen werden, da zwar mit ihrer Hilfe die gesamte Klasse erfasst werden kann, aber sehr viele Bereiche im Kamerabild nur Umgebung zeigen und die Schüler entsprechend klein dargestellt sind. Eine fokussiertere Kamera würde zwar weniger Schüler erfassen, diese währen allerdings deutlich größer darstellen und die Kamera kann passend zur Position der Schüler aufgestellt werden.