Durch den großen Raumbereich in dem das Verfahren auf Basis der Kopforientierung funktioniert ist die Positionswahl der Kameras hierfür recht frei und kann so gewählt werden, dass sie die gesamte Klasse erfassen, selten etwas verdeckt und der Unterricht dadurch wenig beeinflusst wird. Aus messtechnischer Sicht wäre die ideale Position der Kamera im Zentrum vor der Klasse, so dass die Hauptblickrichtung der Schüler in Richtung Kamera verläuft. Diese Stelle kann jedoch voraussichtlich häufig nicht verwendet werden da diese Position für den Unterricht (Tafel/Lehrer) benötigt wird.\\
Die Problematik mit verdeckten Landmarks, kann mit mehreren Kameras entgegen gewirkt werden, die beispielsweise an der Seite der Tafel platziert sind. Dies bietet neben der Möglichkeit einer 3D-Rekonstruktion der Szene auch die Chance das Gesicht in allen Orientierungswinkeln vollständig zu erfassen. Außerdem helfen verschiedene Perspektiven dabei das Problem der Verdeckung eines Gesichts im Kamerabild zu lösen, da vielleicht eine andere Kamera freie Sicht hat.\\
Für die hinteren Reihen ist der Einsatz von zusätzlichen Kameras zu empfehlen, da diese Schüler recht klein dargestellt werden und oft durch die vorderen Reihen verdeckt werden, wenn sie von nur einer Kamera erfasst werden, die vor der Klasse aufgestellt wurde.\\
Mit entsprechend hochauflösenden Kameras können auch bessere Resultate auf größeren Distanzen erzielt werden, wobei ein großer Unterschied eher ausbleiben wird, sollte die gesamte Klasse auf einmal aufgenommen werden, da die Schüler recht weit verteilt sind und die Auflösung durch den großen Distanzunterschied begrenzt ist.\\ 
Als Messinstrument ist eine oder mehrere Videokameras sinnvoll, da die Qualität der Auswertung auf Videos die begrenze Auflösung bei weitem wettmacht. Dabei kann eine Kamera einen Bereich von $45^\circ$ mit $5m$ Tiefe sinnvoll erfassen.\\
Da der große Erfassungsbereich nur auf Videos erreicht wird, wäre es von Vorteil, die Detektion und das Tracking soweit zu ergänzen, dass auf Profilbildern gearbeitet werden kann um Landmarks zu erkennen. Somit kann das Tracking auch begonnen werden, wenn die Probanden nicht in Richtung Kamera blicken und wäre gegenüber Drehungen robuster. Entsprechende Verfahren existieren, z.B. Real-time Multi-view Facial Landmark Detector \cite{Uricar-etal-BWILD-2015}\\
Auch der Einsatz von Weitwinkelobjektiven kann nicht empfohlen werden, da zwar mit ihrer Hilfe die gesamte Klasse erfasst werden kann, aber sehr viele Bereiche im Kamerabild nur Umgebung zeigen und die Schüler entsprechend klein dargestellt werden. Eine fokussiertere Kamera würde zwar weniger Schüler erfassen, diese werden allerdings deutlich größer dargestellt und die Kamera kann passend zur Position der Schüler aufgestellt werden.\\
Eine weitere Problematik ist das Licht bei der Aufnahme, es gibt starke Lichtquellen wie Fenster und Schatten in den Augenhören der Probanden. Für eine gleichmäßige Beleuchtung könnten Neonlampen von oben sorgen und eine Abdunklung der Fenster um weder die Probanden noch die Kameras zu blenden.\\
Abschließend kann die Aussage getroffen werden, das es mit dem aktuellen Stand der Technik möglich ist, mehrere Personen mit nur einer einzigen Kamera soweit zu analysieren, dass eine Aussage über ihre Kopforientierung gemacht werden kann.