\section{Verbesserung der Augen}
\label{verbesserung_ElSe}
Zusätzlich zu den 64 Landmarks, die ein Gesicht beschreiben, kann von OpenFace weitere 28 Landmarks für ein Auge bestimmt werden, aus denen dann die Blickrichtung ermittelt wird.\\
Um die Position der Landmarks zu verbessern, kann auf dem Bildausschnitt der Augen der ElSe-Algotithmus eingesetzt werden. Dieser Algorithmus arbeitet auf einem Farbbild um so die Umrisse der Pupille zu berechnen.\\
Da unter den 28 Landmarks die Umrisse von Pupille und Iris beschreiben wird, müssen diese aus dem Ergebnis von ElSe abgeleitet werden. Dabei hat sich eine Veränderung des Radius mit ?? für Pupille und ?? für die Iris bewährt.\\
Allerdings muss das Auge für die Berechnung aus entsprechend vielen Pixeln bestehen, wodurch es im Originalbild mindestens mit 10 Pixeln dargestellt wird, um sinnvolle Ergebnisse zu erhalten. Da diese Berechnung unabhängig der Landmarks ausgeführt wird, empfiehlt sich das Ergebnis zu überprüfen, damit die bestimmte Ellipse auch innerhalb der Augenhöhle liegt.\\
Dabei wird jedes Auge unabhängig vom anderen betrachtet, wodurch sich verschiedene Blickrichtung ergeben. Ab einer Distanz von mehr als ??cm kann die Blickrichtung beider Augen als parallel angesehen und kann entsprechend behandelt werden. Eine Verbesserung ergibt sich, wenn beide Augen anhängig von einander bestimmt werden, damit sich der Fehler minimiert.
\subsection{To Do}
\begin{itemize}
	\item Größe für Else
	\item Grenze für Rechnung
\end{itemize}