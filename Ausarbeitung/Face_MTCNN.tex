\section{MTCNN Face Detection}
\label{MTCNN}
Multi-task Cascaded Convolutional Networks (MTCNN) ist ein Algorithmus zur Detektion von Gesichtern und Bestimmung von 5 Gesichts-Landmarks in Farbbilder. Dabei werden drei CNN auf einer Bildpyramide angewendet um zuverlässig Gesichter verschiedenster Größe zu erkennen. Des weiteren wird für die Detektion der Gesichter auch deren Ausrichtung berücksichtigt, um bessere Ergebnis zu erzielen.\\
Sein Einsatzgebiet ist die Vorverarbeitung eines Frames für die spätere Auswertung. Somit soll dieser Schritt von einem möglichst robusten Verfahren durchgeführt werden. Dabei wird auf einem hochauflösendem Bild gearbeitet mit verhältnismäßig kleinen, verschieden Großen und weit verteilten Gesichter.
\subsection{Die 3 Stufen der Verarbeitung}
Für die gute Detektionsqualität sorgt die dreistufige Verarbeitung mit verschiedenen CNN auf einer Bildpyramide. Bei der Bildpyramide handelt es sich um ein in verschiedenen Größen skaliertes Bild, damit der gesuchte Inhalt in der gewünschten Auflösung abgebildet ist, ohne etwas über den Inhalt zu wissen.\\
Dies ist von Vorteil, damit das CNN auf eine feste Größe von Gesichtern optimiert werden kann, um das Lernen nicht zusätzlich zu erschweren. So werden nur die Farbverläufen gelernt und nicht weite durch die Skalierung erschwert, wodurch das CNN auf seine jeweilige Aufgabe besser optimiert werden kann.
\begin{figure}
	\centering
	\includegraphics[width=0.6\linewidth]{img/MTCNN_Step}
	\caption{Darstellung des Funktionsablaufes von MTCNN\cite{MTCCN}}
	\label{img_MTCNN_Step}
\end{figure}
\subsubsection{Stufe 1}
Beim ersten Verarbeitungsschritt werden alle Bereiche eines Bilds gesucht, in denen möglicherweise ein Gesicht zu erkennen ist. Dazu wird für die Detektion ein CNN, dem sogenannten Proposal Network (P-Net) eingesetzt, das alle möglichen Bounding-Boxen ermittelt in denen ein Gesicht zu sehen sein könnte. Diese Bounding-Boxen werden anschließend mit einem NMS ausgedünnt, um die am stärksten überlappenden Boxen zusammen zu fassen. Dies ist notwendig, da dieses CNN zwar recht schnell arbeitet, allerdings auch mit einer sehr großen False-True-Fehlerrate (Erkennen trotz nicht vorhanden).
\subsubsection{Stufe 2}
Anschließend werden die möglichen Bereiche mittels eines weiten CNN analysiert, damit alle Nicht-Gesichtsbereiche erkannt und entfernt werden können. Dies wird von dem Refine Network (R-Net) übernommen und anschließend die möglichen Bounding-Boxen mittels NMS noch weiter reduziert.
\subsubsection{Stufe 3}
Der letzte Schritt wird von einem deutlich genaueren CNN übernommen, um ein Gesicht zu detektieren, dem sogenannten Output Network (O-Net). Womit die resultierenden exakten Boxen mit ihren jeweiligen 5 Landmarks ermittelt werden.
\subsection{Qualität}
MTCNN Face Detection ist bei der Zuverlässigkeit im Verglich zu anderen bekannten Verfahren überlegen, siehe \autoref{img_MTCNN_quality}, und zudem Echtzeit fähig auf $640\times 480$ Großen Bilder. Dabei können auch Gesichter mit einer Größe von $20\times 20$ Pixel erfolgreich erkannt werden.\\
Somit sind alle Anforderungen erfüllt um mit diesem Verfahren den vorhanden Frame für die nachfolgenden Berechnungen vorzubereiten. Ein Test bestätigt diese Annahme, siehe \autoref{img_bereich_MTCNN}.
\begin{figure}
	\centering
	\includegraphics[width=0.7\linewidth]{img/MTCNN_quality}
	\caption{Qualität der Detektion der verscheiden Verfahren im Vergleich \cite{MTCCN}}
	\label{img_MTCNN_quality}
\end{figure}
\begin{figure}
	\centering
	\begin{tabular}{|c|c|c|c|c|c|c|c|c|c|c|}
		\hline
		\includegraphics[width=1.1cm]{img_MTCNN/Img1-4_pupil1}&
		\includegraphics[width=1.1cm]{img_MTCNN/Img2-4_pupil1}&
		\includegraphics[width=1.1cm]{img_MTCNN/Img3-4_pupil1}&
		\includegraphics[width=1.1cm]{img_MTCNN/Img4-4_pupil1}&
		\includegraphics[width=1.1cm]{img_MTCNN/Img5-4_pupil1}&
		\includegraphics[width=1.1cm]{img_MTCNN/Img6-4_pupil1}&
		\includegraphics[width=1.1cm]{img_MTCNN/Img7-4_pupil1}&
		\includegraphics[width=1.1cm]{img_MTCNN/Img8-4_pupil1}&
		\includegraphics[width=1.1cm]{img_MTCNN/Img9-4_pupil1}&
		\includegraphics[width=1.1cm]{img_MTCNN/Img10-4_pupil1}&
		\includegraphics[width=1.1cm]{img_MTCNN/Img11-4_pupil1}\\
		\hline
		$1m$& $2m$& $3m$& $4m$& $5m$& $6m$& $7m$& $8m$& $9m$& $10m$& $11m$\\\hline
	\end{tabular}
	\caption{Dargestellt ist die Box und die 5 Landmarks von MTCNN-Face bei verschiedenen Distanzen des Probanden zur Kamera}
	\label{img_bereich_MTCNN}
\end{figure}
%Joint Face Detection and Alignment using Multi-task Cascaded Convolutional Networks