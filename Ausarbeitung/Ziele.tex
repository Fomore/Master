\subsection{Gesetzte Ziele}
Da die Anwendung auf Aufnehmen eines Unterrichtes arbeiten soll, ergeben sich folgende Bedingungen:
\begin{itemize}
\item Normale Brillen, Kontaktlinsen und ähnliches sind erlaubt.
\item Bewegung/Sprechen usw. der Schüler ist erlaubt.
\item Es soll gleichzeitig auf Distanzen zwischen $2.5 - 8m$ zur Kamera funktionieren, ohne das sich diese Bewegt oder gezoomt werden kann.
\item Möglichst alle Blickrichtungen der Schüler sollen möglichst genau erfasst werden.
\end{itemize}
Ein deutsches Klassenzimmer hat $55-65m^2$, da noch Abstand zur Tafel usw. beachtet werden muss ergibt sich, wenn sich die Kamera an der Tafel befindet, einen Abstand zu den Schülern von $2.5 - 8m$ zur Kamera auf einer Breiten von $6m$. Somit muss der Linsenwinkel mindestens $100^\circ$ betragen.

\subsection{Randbedingungen}
Des weiteren werden folgende Annahmen gemacht:
\begin{itemize}
\item Die Szene ist Innerhalb eines Gebäudes stattfinden, mit ausreichend gleichmäßiger Beleuchtung.
\item Die Überführung zwischen Welt- und Kamerakoordinatensystem bekannt.
\item Die Kamera befindet sich vor der Klasse, so das die Blickrichtung nach vorn von den Schülern möglichst zur Kamera ist.
\item Die Gesichter der der Schüler sind komplett sichtbar.
\end{itemize}
Natürlich sind auch alle inneren Parameter der Kamera bekannt.