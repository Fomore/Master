\section{Gesetzte Bedingungen der Anwendung}
Damit der Unterricht, wie im Szenario der Problemstellung beschreiben \ref{Problemstellung}, möglichst wenig beeinflusst wird, ergeben sich folgende Randbedingungen:
\begin{itemize}
\item Brillen, Kontaktlinsen und ähnliches sind erlaubt.
\item Die üblichen Bewegungen im Unterricht wie Sprechen, Kopfdrehungen usw. der Schüler ist gestattet.
\item Es soll gleichzeitig auf Distanzen zwischen $2.5 - 8m$ zur Kamera auf einer Breit von $6m$ funktionieren.
\item Möglichst alle Blickrichtungen der Schüler sollen so exakt wie möglich erfasst werden.
\end{itemize}
Ein deutsches Klassenzimmer hat $55-65m^2$, da noch Abstand zur Tafel usw. beachtet werden muss ergibt sich, wenn die Kamera an der Tafel befindet, einen Abstand zu den Schülern von $2.5 - 8m$ zur Kamera auf einer Breiten von $6m$. Somit muss der Linsenwinkel mindestens $100^\circ$ betragen.\\
Außerdem soll die Anwendung auf schon vorhanden Aufnehmen eines Unterrichtes arbeiten, bei denen oben genannten Bedingungen erfüllen.
%Quelle
\subsection{Randbedingungen der Anwendung}
Des weiteren werden folgende Annahmen gemacht:
\begin{itemize}
\item Die Szene ist Innerhalb eines Gebäudes, mit ausreichend gleichmäßiger Beleuchtung.
\item Die Überführung zwischen Welt- und Kamerakoordinatensystem bekannt.
\item Die Kamera befindet sich vor der Klasse, so das die Hauptblickrichtung der Schüler in ihrem Fokus liegt.
\item Die Gesichter sind komplett sichtbar und nicht verdeckt.
\end{itemize}
Natürlich sind auch alle inneren Parameter der Kamera bekannt.