\section{Das Klassenzimmer - Umgebung des Eye-Tracking}
Die Anwendung ist für den Unterricht ausgelegt, wie in der Problemstellung \autoref{Problemstellung} beschrieben und soll diesen möglichst wenig beeinflusst, ergeben sich folgende Randbedingungen:
\begin{itemize}
\item Brillen, Kontaktlinsen und ähnliches sind bei den Probanden erlaubt, ebenso Frisuren, Make-up usw.
\item Die üblichen Bewegungen im Unterricht wie Sprechen, Kopfdrehungen usw. der Schüler sind gestattet.
\item Das Verfahren soll gleichzeitig auf Distanzen zwischen $2.5 - 8m$ zur Kamera auf einer Breite von $6m$ funktionieren.
\item Möglichst alle Blickrichtungen bzw. die Gesichtsorientierung der Schüler sollen so exakt wie möglich erfasst werden.
\end{itemize}
Ein deutsches Klassenzimmer soll laut Baden-Württembergischen Schulbauempfehlungen eine Grundfläche von $54-66m^2$ aufweisen für maximal 28-32 Schülern. Da noch die Tafel usw. beachtet werden muss ergibt sich einen Abstand von $2.5 - 8m$ zwischen Kamera und Schüler auf einer Breiten von $6m$, dabei befindet sich die Kamera in der Nähe der Tafel. Somit muss der Linsenwinkel mindestens $100^\circ$ betragen, damit alle im Bild sind, mit entsprechender Schärfentiefe.\\
Außerdem soll die Anwendung auf schon vorhanden Aufnehmen eines Unterrichtes arbeiten, die oben genannten Bedingungen erfüllen.\\
\cite{bauordung}
\subsection{Randbedingungen der Anwendung}
Zusätzlich werden folgende Annahmen gemacht, die sich vor allem auf die Sitzrodung der Schüler und die Umgebung beziehen.
\begin{itemize}
\item Die Szene ist innerhalb eines Gebäudes, mit ausreichend gleichmäßiger Beleuchtung.
\item Die Überführung zwischen Welt- und Kamerakoordinatensystem ist bekannt.
\item Die Kamera befindet sich vor der Klasse, so dass die Hauptblickrichtung der Schüler in Richtung Kamera verläuft.\\
Gleichzeitig kann die Kamera jedoch nicht ohne weiteres ganz zentral angebracht werden, da dieser Raum für den Unterricht (Tafel/Lehrer) benötigt wird.
\item Die Gesichter sind komplett sichtbar und nicht verdeckt durch andere Schüler oder von der Kamera abgewannt.\\
Eine Sitzrodung, wie sie hauptsächlich im Frontalunterricht üblich ist.
\end{itemize}