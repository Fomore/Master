\section{Versuch 1 - Arbeitsbereich der Verfahren}
\label{Versuch_1}
Mit diesem Versuch soll der Zusammenhang zwischen Standort eines Probanden und Position des Blickziels untersucht werden. Dazu wird eine Klassenzimmerumgebung simuliert, in der sowohl Standort als auch die Blickziele relativ zur Kamera bekannt sind.\\
Als Messinstrument für die Versuche 1 und 2 wurde die Explorer 4K Actioncam verwendet, da sie eine hohe Auflösung bei ausreichend $FPS$ und eine $170^\circ$ Weitwinkel-Linse mit großer Schärfentiefe besitzt. Mit ihrer 2,7K Einstellung wird ein $2688 \times 1520$ Farbvideo mit $30FPS$ aufgezeichnet.
\newpage
Allerdings ist die Bildqualität durch Pixelrauschen und Ähnliches deutlich schlechter als die Verkleinerung der Originalaufnahmen in den Datensätzen.
\subsection{Versuchsaufbau}
In einem Raum wurde die Kamera in $2,06m$ Höhe $31cm$ hinter den Blickzielen so montiert, das der gesamte Raum im Fokus liegt. Als Blickziel wurden 9 Punkte auf einer Ebene markiert mit der Kamera im Zentrum. Die Anordnung der Blickziele ist in \autoref{img_aufbau_target_Test} dargestellt.\\
Als Position der Probanden wurde ein Rasterfeld mit $1m$ Kantenlänge im Raum eingezeichnet auf einer Fläche von $7 \times 11m$. Die Probanden stellten sich auf diesen Positionen auf um nacheinander alle Blickziele zu betrachten. 
\begin{figure}
	\centering
	\includegraphics[width=\linewidth]{img/Target}
	\caption{Aufbau der Blickziele in Vorsuch 1, alle Angaben gerundet in Zentimeter\\rote Punkte: Blickziele, blauer Punkt: Kamera}
	\label{img_aufbau_target_Test}
\end{figure}
\subsection{Detektion mit MTCNN}
Um die Detektionswahrscheinlichkeit des MTCNN-Face Detektors zu testen wurden diese Videos analysiert.\\
Es zeigt sich, das auf allen Positionen die Probanden erfolgreich erkannt wurden und die Boxen das Gesicht recht gut beschreiben. Allerdings ist zu erkennen, das die Landmarks unzureichend genau sind. Sie sollten die Mundwinkel, Nasenspitze und beide Augen markieren, liegen aber schon bei recht großen Bildern weit daneben, siehe \autoref{img_bereich_MTCNN}
\begin{figure}
	\centering
	\begin{tabular}{|c|c|c|c|c|c|c|c|c|c|c|}
		\hline
		\tabbild[width=0.06\linewidth]{img_MTCNN/Img1-4_pupil1}&
		\tabbild[width=0.06\linewidth]{img_MTCNN/Img2-4_pupil1}&
		\tabbild[width=0.06\linewidth]{img_MTCNN/Img3-4_pupil1}&
		\tabbild[width=0.06\linewidth]{img_MTCNN/Img4-4_pupil1}&
		\tabbild[width=0.06\linewidth]{img_MTCNN/Img5-4_pupil1}&
		\tabbild[width=0.06\linewidth]{img_MTCNN/Img6-4_pupil1}&
		\tabbild[width=0.06\linewidth]{img_MTCNN/Img7-4_pupil1}&
		\tabbild[width=0.06\linewidth]{img_MTCNN/Img8-4_pupil1}&
		\tabbild[width=0.06\linewidth]{img_MTCNN/Img9-4_pupil1}&
		\tabbild[width=0.06\linewidth]{img_MTCNN/Img10-4_pupil1}&
		\tabbild[width=0.06\linewidth]{img_MTCNN/Img11-4_pupil1}\\
		\hline
		$1m$& $2m$& $3m$& $4m$& $5m$& $6m$& $7m$& $8m$& $9m$& $10m$& $11m$\\\hline
	\end{tabular}
	\caption{Dargestellt ist die Box und die 5 Landmarks von MTCNN-Face bei verschiedenen Distanzen des Probanden zur Actioncam}
	\label{img_bereich_MTCNN}
\end{figure}
\subsection{Auswertung der Aufnahme}
Für die Analyse wurden aus dem Video jene Frames ausgewählt in denen ein Blickziel fokussiert wurde und analysiert.\\
Als erstes wurde die Einzelbildauswertung von OpenFace auf die Frames angewendet und jene Abbildungen der Kopfrotationen markiert, in denen erfolgreich ein Gesicht erkannt wurde. In \autoref{graph_Test_1_Normal} links ist der horizontale Wertebereich dargestellt in dem an der jeweiligen Position ein Gesicht erfolgreich erkannt wurde.\\
Im zweiten Teil wurden die selben Frames für die Messung verwendet, dieses mal allerdings wurde das gesamte Video analysiert. Der Winkelbereich in dem auf der horizontalen Achse an den entsprechenden Positionen ein Gesicht erkannt wurde, ist in \autoref{graph_Test_1_Normal} rechts dargestellt.\\
Um alle Verbesserungen in einer realen Umgebung auszutesten, wurde wie in \autoref{Implementierung_Ablauf} beschrieben vorgegangen und die relevanten Bildausschnitte linear vergrößert. Die Auswirkung auf den horizontalen Bereich ist in \autoref{graph_Test_1_Resize} dargestellt. Durch diese Verbesserung wird die Distanz auf der gearbeitet werden kann mehr als verdoppelt bei der Video- und Einzelbild-Analyse.
\begin{figure}
	\centering
	\input{Versuch1_Img}
		$+6m$ &
	\tabbild[width=0.048\linewidth]{img_Bereich/V1_vid_Winkel_X_-3000_5000.png}&
	\tabbild[width=0.048\linewidth]{img_Bereich/V1_vid_Winkel_X_-2000_5000.png}&
	\tabbild[width=0.048\linewidth]{img_Bereich/V1_vid_Winkel_X_-1000_5000.png}&
	\tabbild[width=0.048\linewidth]{img_Bereich/V1_vid_Winkel_X_0_5000.png}&
	\tabbild[width=0.048\linewidth]{img_Bereich/V1_vid_Winkel_X_1000_5000.png}&
	\tabbild[width=0.048\linewidth]{img_Bereich/V1_vid_Winkel_X_2000_5000.png}&\\ 
	\hline 
	$+5m$ &
	\tabbild[width=0.048\linewidth]{img_Bereich/V1_vid_Winkel_X_-3000_5000.png}&
	\tabbild[width=0.048\linewidth]{img_Bereich/V1_vid_Winkel_X_-2000_5000.png}&
	\tabbild[width=0.048\linewidth]{img_Bereich/V1_vid_Winkel_X_-1000_5000.png}&
	\tabbild[width=0.048\linewidth]{img_Bereich/V1_vid_Winkel_X_0_5000.png}&
	\tabbild[width=0.048\linewidth]{img_Bereich/V1_vid_Winkel_X_1000_5000.png}&
	\tabbild[width=0.048\linewidth]{img_Bereich/V1_vid_Winkel_X_2000_5000.png}&\\ 
	\hline 
	$+4m$ &
	\tabbild[width=0.048\linewidth]{img_Bereich/V1_vid_Winkel_X_-3000_4000.png}&
	\tabbild[width=0.048\linewidth]{img_Bereich/V1_vid_Winkel_X_-2000_4000.png}&
	\tabbild[width=0.048\linewidth]{img_Bereich/V1_vid_Winkel_X_-1000_4000.png}&
	\tabbild[width=0.048\linewidth]{img_Bereich/V1_vid_Winkel_X_0_4000.png}&
	\tabbild[width=0.048\linewidth]{img_Bereich/V1_vid_Winkel_X_1000_4000.png}&
	\tabbild[width=0.048\linewidth]{img_Bereich/V1_vid_Winkel_X_2000_4000.png}&
	\tabbild[width=0.048\linewidth]{img_Bereich/V1_vid_Winkel_X_3000_4000.png}\\ 
	\hline 
	$+3m$ &
	\tabbild[width=0.048\linewidth]{img_Bereich/V1_vid_Winkel_X_-3000_3000.png}&
	\tabbild[width=0.048\linewidth]{img_Bereich/V1_vid_Winkel_X_-2000_3000.png}&
	\tabbild[width=0.048\linewidth]{img_Bereich/V1_vid_Winkel_X_-1000_3000.png}&
	\tabbild[width=0.048\linewidth]{img_Bereich/V1_vid_Winkel_X_0_3000.png}&
	\tabbild[width=0.048\linewidth]{img_Bereich/V1_vid_Winkel_X_1000_3000.png}&
	\tabbild[width=0.048\linewidth]{img_Bereich/V1_vid_Winkel_X_2000_3000.png}&
	\tabbild[width=0.048\linewidth]{img_Bereich/V1_vid_Winkel_X_3000_3000.png}\\ 
	\hline 
	$+2m$ & &
	\tabbild[width=0.048\linewidth]{img_Bereich/V1_vid_Winkel_X_-2000_2000.png}&
	\tabbild[width=0.048\linewidth]{img_Bereich/V1_vid_Winkel_X_-1000_2000.png}&
	\tabbild[width=0.048\linewidth]{img_Bereich/V1_vid_Winkel_X_0_2000.png}&
	\tabbild[width=0.048\linewidth]{img_Bereich/V1_vid_Winkel_X_1000_2000.png}&
	\tabbild[width=0.048\linewidth]{img_Bereich/V1_vid_Winkel_X_2000_2000.png}& \\ 
	\hline 
	$+1m$ & & &
	\tabbild[width=0.048\linewidth]{img_Bereich/V1_vid_Winkel_X_-1000_1000.png}&
	\tabbild[width=0.048\linewidth]{img_Bereich/V1_vid_Winkel_X_0_1000.png}&
	\tabbild[width=0.048\linewidth]{img_Bereich/V1_vid_Winkel_X_1000_1000.png}& &\\ 
	\hline 
	& $-3m$& $-2m$ & $-1m$ &Kamera& $+1m$ & $+2m$ & $+3m$ \\ 
	\hline 
\end{tabular}
	\caption{Dargestellt ist der horizontale Winkelbereich in der sich ein Gesicht drehen kann und erkannt wurde\\
	Oben: Einzelbilder, Unten: Video}
	\label{graph_Test_1_Normal}
\end{figure}
\subsection{Ergebnis}
Es zeigt sich, dass eine Auswertung von OpenFace auf einem Video deutlich zuverlässiger arbeitet als auf Einzelbildern, vor allem der größere Arbeitsbereich bezüglich der Rotation ist von Vorteil.\\
Durch die Verwendung des Weitwinkelobjektivs, kann die gesamte Breite eines Klassenzimmers erfasst werden und der Arbeitsbereich der Auswertung ist für eine erfolgreiche Detektion und Analyse breit genug um Schüler erfassen zu können, die selbst die vorderen Eckpunkte eines Klassenzimmers betrachten.\\
Bei der Distanz zur Kamera (Tiefe) besteht Handlungsbedarf, als Ziel wurde $8m$ angesetzt und das aktuelle Verfahren endet bei $5m$ in der einfachen Ausführung. Wird der Bildbereich allerdings verbessert, so verdoppelt sich die Distanz zwischen Kamera und Person, wodurch eine Abdeckung des gesamten Klassenzimmers erreicht wird.\\
Der in \autoref{OpenFace_skal} theoretisch bestimmte Detektionsabstand von $14m$ konnte nicht erreicht werden, die erreichte Maximaldistanz liegt bei etwa $10m$, immer noch ausreichend für ein Klassenzimmer. Als Ursache kann das Pixelrauschen und die Überbeleuchtung durch das einfallende Licht der Fenster angenommen werden.\\
Auch MTCNN-Face ist als Detektor geeignet, er findet zuverlässig alle Gesichter im Frame, unabhängig ihrer Größe und Orientierung. Sogar jene die von OpenFace nicht mehr verwendet werden können. Einzige Anmerkung ist die etwas ungenaue Box, dies kann aber mit einer einfachen Verschiebung der Boxränder korrigiert werden.\\
Eine signifikante Aussage bezüglich des vertikalen Winkel kann aus diesem Aufbau nicht getroffen werden, da die Neigungswinkel zwar differenziert werden können, sie allerdings zu ähnlich sind bei stehenden Personen (beides mal fast horizontal).\\
Von Interesse ist ob ein Schüler auch erkannt werden kann, wenn dieser auf den Tisch vor sich schaut, was einen sehr großen Neigungswinkel bedeutet. Um dies aus zu testen wurde Versuch 2 durchgeführt.
\section{Versuch 2 - Arbeitsbereich bezüglich der Neigung des Kopfes}
Da ein aufmerksamer Schüler durchaus auch auf den Tisch blicken kann, z.B. beim Schreiben, soll getestet werden wie weit die Analyse in solchen Situationen funktioniert.
\subsection{Versuchsaufbau}
Für diesen Versuch wurde die Kamera auf $1,88m$ Höhe aufgestellt. Die Standorte des Probanden lagen je einem Meter weit auseinander auf einer Gerade bei $3m$ und $9m$ vor der Kamera, auf einer Breite von $8m$.\\
Als Blickziele diente die Kamera, ein Punkt $78cm$ unterhalb der Kamera, sowie einer $40cm$ über dem Boden und $50cm$ vor der Kamera. Alle anderen Blickziele befanden sich $1m$ vor den markierten Standorten auf dem Boden ($2m$ und $8m$).\\
Die Messung wurde außerhalb eines Gebäudes an einem bedecken Tag durchgeführt, wodurch eine helle schattenlose Szene entsteht.
\subsection{Auswertung}
Die algorithmische Auswertung entspricht der von Versuch 1. In \autoref{graph_Test_2_Normal} ist der vertikale Bereich dargestellt in der sich ein Gesicht neigen und erkannt werden kann. Es ist zu sehen, dass nur Gesichter auf der $3m$ Linie gefunden werden konnte, wiederum ist der Arbeitsbereich bei Verwendung der Video-Analyse etwas größer mit $60^\circ$, wobei ein Winkel von $50^\circ$ immer erfasst werden kann.\\
Durch die Verbesserung der Eingabebilder kann auch auf einer Distanz von $9m$ gearbeitet werden, siehe \autoref{graph_Test_2_Resize}, mit demselben Arbeitsbereich von mindestens $50^\circ$.
\begin{figure}[!h]
	\centering
		\begin{tabular}{|c|c|c|c|c|c|c|c|c|}
	\hline 
	$+3m$ & &
	\includegraphics[width=0.5cm]{img_Bereich/V2_img_Winkel_Y_-2000_3000.png}&
	\includegraphics[width=0.5cm]{img_Bereich/V2_img_Winkel_Y_-1000_3000.png}&
	\includegraphics[width=0.5cm]{img_Bereich/V2_img_Winkel_Y_0_3000.png}&
	\includegraphics[width=0.5cm]{img_Bereich/V2_img_Winkel_Y_1000_3000.png}&
	\includegraphics[width=0.5cm]{img_Bereich/V2_img_Winkel_Y_2000_3000.png}&
	\includegraphics[width=0.5cm]{img_Bereich/V2_img_Winkel_Y_3000_3000.png}&\\ 
	\hline 
	& $-3m$ & $-2m$ & $-1m$ &Kamera& $+1m$ & $+2m$ & $+3m$ & $+4m$ \\ 
	\hline
	\hline 
	$+3m$ &
	\includegraphics[width=0.5cm]{img_Bereich/V2_vid_Winkel_Y_-3000_3000.png}&
	\includegraphics[width=0.5cm]{img_Bereich/V2_vid_Winkel_Y_-2000_3000.png}&
	\includegraphics[width=0.5cm]{img_Bereich/V2_vid_Winkel_Y_-1000_3000.png}&
	\includegraphics[width=0.5cm]{img_Bereich/V2_vid_Winkel_Y_0_3000.png}&
	\includegraphics[width=0.5cm]{img_Bereich/V2_vid_Winkel_Y_1000_3000.png}&
	\includegraphics[width=0.5cm]{img_Bereich/V2_vid_Winkel_Y_2000_3000.png}&
	\includegraphics[width=0.5cm]{img_Bereich/V2_vid_Winkel_Y_3000_3000.png}&
	\includegraphics[width=0.5cm]{img_Bereich/V2_vid_Winkel_Y_4000_3000.png}\\ 
	\hline 
	& $-3m$ & $-2m$ & $-1m$ &Kamera& $+1m$ & $+2m$ & $+3m$ & $+4m$ \\ 
	\hline 
\end{tabular}
	\caption{Dargestellt ist der Bereich des Neigungswinkels des Probanden, in denen ein Gesicht erkannt wurde.\\
		Oben: Einzelbilder, Unten: Video}
	\label{graph_Test_2_Normal}
\end{figure}
\begin{figure}
\centering
	\input{Versuch2_res}
	\caption{Dargestellt ist der Bereich des Neigungswinkels des Probanden, in denen ein Gesicht erkannt wird bei aufbereitetem Eingabebild.\\
	Oben: Einzelbilder, Unten: Video}
\label{graph_Test_2_Resize}
\end{figure}
\begin{figure}
	\centering
	\begin{tabular}{|c|c|c|c|c|c|c|c|}
	\hline
	$+11m$ & & & &
	\includegraphics[width=1cm]{img_Bereich/V1_img_res_Winkel_X_0_11000.png}& & &\\ 
	\hline 
	$+10m$ &
	\includegraphics[width=1cm]{img_Bereich/V1_img_res_Winkel_X_-3000_10000.png} &
	\includegraphics[width=1cm]{img_Bereich/V1_img_res_Winkel_X_-2000_10000.png}&
	\includegraphics[width=1cm]{img_Bereich/V1_img_res_Winkel_X_-1000_10000.png}&
	\includegraphics[width=1cm]{img_Bereich/V1_img_res_Winkel_X_0_10000.png}&
	\includegraphics[width=1cm]{img_Bereich/V1_img_res_Winkel_X_1000_10000.png}&
	\includegraphics[width=1cm]{img_Bereich/V1_img_res_Winkel_X_2000_10000.png}&
	\includegraphics[width=1cm]{img_Bereich/V1_img_res_Winkel_X_3000_10000.png}\\ 
	\hline 
	$+9m$ &
	\includegraphics[width=1cm]{img_Bereich/V1_img_res_Winkel_X_-3000_9000.png} &
	\includegraphics[width=1cm]{img_Bereich/V1_img_res_Winkel_X_-2000_9000.png}&
	\includegraphics[width=1cm]{img_Bereich/V1_img_res_Winkel_X_-1000_9000.png}&
	\includegraphics[width=1cm]{img_Bereich/V1_img_res_Winkel_X_0_9000.png}&
	\includegraphics[width=1cm]{img_Bereich/V1_img_res_Winkel_X_1000_9000.png}&
	\includegraphics[width=1cm]{img_Bereich/V1_img_res_Winkel_X_2000_9000.png}&
	\includegraphics[width=1cm]{img_Bereich/V1_img_res_Winkel_X_3000_9000.png}\\ 
	\hline 
	$+8m$ &
	\includegraphics[width=1cm]{img_Bereich/V1_img_res_Winkel_X_-3000_8000.png} &
	\includegraphics[width=1cm]{img_Bereich/V1_img_res_Winkel_X_-2000_8000.png}&
	\includegraphics[width=1cm]{img_Bereich/V1_img_res_Winkel_X_-1000_8000.png}&
	\includegraphics[width=1cm]{img_Bereich/V1_img_res_Winkel_X_0_8000.png}&
	\includegraphics[width=1cm]{img_Bereich/V1_img_res_Winkel_X_1000_8000.png}&
	\includegraphics[width=1cm]{img_Bereich/V1_img_res_Winkel_X_2000_8000.png}&
	\includegraphics[width=1cm]{img_Bereich/V1_img_res_Winkel_X_3000_8000.png}\\ 
	\hline 
	$+7m$ &
	\includegraphics[width=1cm]{img_Bereich/V1_img_res_Winkel_X_-3000_7000.png} &
	\includegraphics[width=1cm]{img_Bereich/V1_img_res_Winkel_X_-2000_7000.png}&
	\includegraphics[width=1cm]{img_Bereich/V1_img_res_Winkel_X_-1000_7000.png}&
	\includegraphics[width=1cm]{img_Bereich/V1_img_res_Winkel_X_0_7000.png}&
	\includegraphics[width=1cm]{img_Bereich/V1_img_res_Winkel_X_1000_7000.png}&
	\includegraphics[width=1cm]{img_Bereich/V1_img_res_Winkel_X_2000_7000.png}&
	\includegraphics[width=1cm]{img_Bereich/V1_img_res_Winkel_X_3000_7000.png}\\ 
	\hline 
	$+6m$ &
	\includegraphics[width=1cm]{img_Bereich/V1_img_res_Winkel_X_-3000_6000.png} &
	\includegraphics[width=1cm]{img_Bereich/V1_img_res_Winkel_X_-2000_6000.png}&
	\includegraphics[width=1cm]{img_Bereich/V1_img_res_Winkel_X_-1000_6000.png}&
	\includegraphics[width=1cm]{img_Bereich/V1_img_res_Winkel_X_0_6000.png}&
	\includegraphics[width=1cm]{img_Bereich/V1_img_res_Winkel_X_1000_6000.png}&
	\includegraphics[width=1cm]{img_Bereich/V1_img_res_Winkel_X_2000_6000.png}&
	\includegraphics[width=1cm]{img_Bereich/V1_img_res_Winkel_X_3000_6000.png}\\ 
	\hline 
	$+5m$ &
	\includegraphics[width=1cm]{img_Bereich/V1_img_res_Winkel_X_-3000_5000.png}&
	\includegraphics[width=1cm]{img_Bereich/V1_img_res_Winkel_X_-2000_5000.png}&
	\includegraphics[width=1cm]{img_Bereich/V1_img_res_Winkel_X_-1000_5000.png}&
	\includegraphics[width=1cm]{img_Bereich/V1_img_res_Winkel_X_0_5000.png}&
	\includegraphics[width=1cm]{img_Bereich/V1_img_res_Winkel_X_1000_5000.png}&
	\includegraphics[width=1cm]{img_Bereich/V1_img_res_Winkel_X_2000_5000.png}&
	\includegraphics[width=1cm]{img_Bereich/V1_img_res_Winkel_X_3000_5000.png}\\ 
	\hline 
	$+4m$ &
	\includegraphics[width=1cm]{img_Bereich/V1_img_res_Winkel_X_-3000_4000.png}&
	\includegraphics[width=1cm]{img_Bereich/V1_img_res_Winkel_X_-2000_4000.png} &
	\includegraphics[width=1cm]{img_Bereich/V1_img_res_Winkel_X_-1000_4000.png}&
	\includegraphics[width=1cm]{img_Bereich/V1_img_res_Winkel_X_0_4000.png}&
	\includegraphics[width=1cm]{img_Bereich/V1_img_res_Winkel_X_1000_4000.png}&
	\includegraphics[width=1cm]{img_Bereich/V1_img_res_Winkel_X_2000_4000.png}&\\ 
	\hline 
	$+3m$ & &
	\includegraphics[width=1cm]{img_Bereich/V1_img_res_Winkel_X_-2000_3000.png} &
	\includegraphics[width=1cm]{img_Bereich/V1_img_res_Winkel_X_-1000_3000.png}&
	\includegraphics[width=1cm]{img_Bereich/V1_img_res_Winkel_X_0_3000.png}&
	\includegraphics[width=1cm]{img_Bereich/V1_img_res_Winkel_X_1000_3000.png}&
	\includegraphics[width=1cm]{img_Bereich/V1_img_res_Winkel_X_2000_3000.png}& \\ 
	\hline 
	$+2m$ & & &
	\includegraphics[width=1cm]{img_Bereich/V1_img_res_Winkel_X_-1000_2000.png}&
	\includegraphics[width=1cm]{img_Bereich/V1_img_res_Winkel_X_0_2000.png}&
	\includegraphics[width=1cm]{img_Bereich/V1_img_res_Winkel_X_1000_2000.png}& &\\ 
	\hline 
	$+1m$ & & &
	\includegraphics[width=1cm]{img_Bereich/V1_img_res_Winkel_X_-1000_1000.png}&
	\includegraphics[width=1cm]{img_Bereich/V1_img_res_Winkel_X_0_1000.png}&
	\includegraphics[width=1cm]{img_Bereich/V1_img_res_Winkel_X_1000_1000.png}& &\\ 
	\hline 
	& $-3m$ & $-2m$ & $-1m$ &Kamera& $+1m$ & $+2m$ & $+3m$ \\ 
	\hline 
\end{tabular}
	\begin{tabular}{|c|c|c|c|c|c|c|c|}
	\hline
	$+11m$ & & & &
	\tabbild[width=0.75cm]{img_Bereich/V1_vid_res_Winkel_X_0_11000.png}& & &\\ 
	\hline 
	$+10m$ &
	\tabbild[width=0.75cm]{img_Bereich/V1_vid_res_Winkel_X_-3000_10000.png}&
	\tabbild[width=0.75cm]{img_Bereich/V1_vid_res_Winkel_X_-2000_10000.png}&
	\tabbild[width=0.75cm]{img_Bereich/V1_vid_res_Winkel_X_-1000_10000.png}&
	\tabbild[width=0.75cm]{img_Bereich/V1_vid_res_Winkel_X_0_10000.png}&
	\tabbild[width=0.75cm]{img_Bereich/V1_vid_res_Winkel_X_1000_10000.png}&
	\tabbild[width=0.75cm]{img_Bereich/V1_vid_res_Winkel_X_2000_10000.png}&
	\tabbild[width=0.75cm]{img_Bereich/V1_vid_res_Winkel_X_3000_10000.png}\\ 
	\hline 
	$+9m$ &
	\tabbild[width=0.75cm]{img_Bereich/V1_vid_res_Winkel_X_-3000_9000.png}&
	\tabbild[width=0.75cm]{img_Bereich/V1_vid_res_Winkel_X_-2000_9000.png}&
	\tabbild[width=0.75cm]{img_Bereich/V1_vid_res_Winkel_X_-1000_9000.png}&
	\tabbild[width=0.75cm]{img_Bereich/V1_vid_res_Winkel_X_0_9000.png}&
	\tabbild[width=0.75cm]{img_Bereich/V1_vid_res_Winkel_X_1000_9000.png}&
	\tabbild[width=0.75cm]{img_Bereich/V1_vid_res_Winkel_X_2000_9000.png}&
	\tabbild[width=0.75cm]{img_Bereich/V1_vid_res_Winkel_X_3000_9000.png}\\ 
	\hline 
	$+8m$ &
	\tabbild[width=0.75cm]{img_Bereich/V1_vid_res_Winkel_X_-3000_8000.png}&
	\tabbild[width=0.75cm]{img_Bereich/V1_vid_res_Winkel_X_-2000_8000.png}&
	\tabbild[width=0.75cm]{img_Bereich/V1_vid_res_Winkel_X_-1000_8000.png}&
	\tabbild[width=0.75cm]{img_Bereich/V1_vid_res_Winkel_X_0_8000.png}&
	\tabbild[width=0.75cm]{img_Bereich/V1_vid_res_Winkel_X_1000_8000.png}&
	\tabbild[width=0.75cm]{img_Bereich/V1_vid_res_Winkel_X_2000_8000.png}&
	\tabbild[width=0.75cm]{img_Bereich/V1_vid_res_Winkel_X_3000_8000.png} \\ 
	\hline 
	$+7m$ &
	\tabbild[width=0.75cm]{img_Bereich/V1_vid_res_Winkel_X_-3000_7000.png}&
	\tabbild[width=0.75cm]{img_Bereich/V1_vid_res_Winkel_X_-2000_7000.png}&
	\tabbild[width=0.75cm]{img_Bereich/V1_vid_res_Winkel_X_-1000_7000.png}&
	\tabbild[width=0.75cm]{img_Bereich/V1_vid_res_Winkel_X_0_7000.png}&
	\tabbild[width=0.75cm]{img_Bereich/V1_vid_res_Winkel_X_1000_7000.png}&
	\tabbild[width=0.75cm]{img_Bereich/V1_vid_res_Winkel_X_2000_7000.png}&
	\tabbild[width=0.75cm]{img_Bereich/V1_vid_res_Winkel_X_3000_7000.png}\\ 
	\hline 
	$+6m$ &
	\tabbild[width=0.75cm]{img_Bereich/V1_vid_res_Winkel_X_-3000_6000.png}&
	\tabbild[width=0.75cm]{img_Bereich/V1_vid_res_Winkel_X_-2000_6000.png}&
	\tabbild[width=0.75cm]{img_Bereich/V1_vid_res_Winkel_X_-1000_6000.png}&
	\tabbild[width=0.75cm]{img_Bereich/V1_vid_res_Winkel_X_0_6000.png}&
	\tabbild[width=0.75cm]{img_Bereich/V1_vid_res_Winkel_X_1000_6000.png}&
	\tabbild[width=0.75cm]{img_Bereich/V1_vid_res_Winkel_X_2000_6000.png}&
	\tabbild[width=0.75cm]{img_Bereich/V1_vid_res_Winkel_X_3000_6000.png}\\ 
	\hline 
	$+5m$ &
	\tabbild[width=0.75cm]{img_Bereich/V1_vid_res_Winkel_X_-3000_5000.png}&
	\tabbild[width=0.75cm]{img_Bereich/V1_vid_res_Winkel_X_-2000_5000.png}&
	\tabbild[width=0.75cm]{img_Bereich/V1_vid_res_Winkel_X_-1000_5000.png}&
	\tabbild[width=0.75cm]{img_Bereich/V1_vid_res_Winkel_X_0_5000.png}&
	\tabbild[width=0.75cm]{img_Bereich/V1_vid_res_Winkel_X_1000_5000.png}&
	\tabbild[width=0.75cm]{img_Bereich/V1_vid_res_Winkel_X_2000_5000.png}&
	\tabbild[width=0.75cm]{img_Bereich/V1_vid_res_Winkel_X_3000_5000.png}\\ 
	\hline 
	$+4m$ &
	\tabbild[width=0.75cm]{img_Bereich/V1_vid_res_Winkel_X_-3000_4000.png}&
	\tabbild[width=0.75cm]{img_Bereich/V1_vid_res_Winkel_X_-2000_4000.png}&
	\tabbild[width=0.75cm]{img_Bereich/V1_vid_res_Winkel_X_-1000_4000.png}&
	\tabbild[width=0.75cm]{img_Bereich/V1_vid_res_Winkel_X_0_4000.png}&
	\tabbild[width=0.75cm]{img_Bereich/V1_vid_res_Winkel_X_1000_4000.png}&
	\tabbild[width=0.75cm]{img_Bereich/V1_vid_res_Winkel_X_2000_4000.png}&
	\tabbild[width=0.75cm]{img_Bereich/V1_vid_res_Winkel_X_3000_4000.png}\\ 
	\hline 
	$+3m$ &
	\tabbild[width=0.75cm]{img_Bereich/V1_vid_res_Winkel_X_-3000_3000.png}&
	\tabbild[width=0.75cm]{img_Bereich/V1_vid_res_Winkel_X_-2000_3000.png}&
	\tabbild[width=0.75cm]{img_Bereich/V1_vid_res_Winkel_X_-1000_3000.png}&
	\tabbild[width=0.75cm]{img_Bereich/V1_vid_res_Winkel_X_0_3000.png}&
	\tabbild[width=0.75cm]{img_Bereich/V1_vid_res_Winkel_X_1000_3000.png}&
	\tabbild[width=0.75cm]{img_Bereich/V1_vid_res_Winkel_X_2000_3000.png}&\\ 
	\hline 
	$+2m$ & &
	\tabbild[width=0.75cm]{img_Bereich/V1_vid_res_Winkel_X_-2000_2000.png}&
	\tabbild[width=0.75cm]{img_Bereich/V1_vid_res_Winkel_X_-1000_2000.png}&
	\tabbild[width=0.75cm]{img_Bereich/V1_vid_res_Winkel_X_0_2000.png}&
	\tabbild[width=0.75cm]{img_Bereich/V1_vid_res_Winkel_X_1000_2000.png}&
	\tabbild[width=0.75cm]{img_Bereich/V1_vid_res_Winkel_X_2000_2000.png}&\\ 
	\hline 
	$+1m$ & & &
	\tabbild[width=0.75cm]{img_Bereich/V1_vid_res_Winkel_X_-1000_1000.png}&
	\tabbild[width=0.75cm]{img_Bereich/V1_vid_res_Winkel_X_0_1000.png}&
	\tabbild[width=0.75cm]{img_Bereich/V1_vid_res_Winkel_X_1000_1000.png}& &\\ 
	\hline 
	& $-3m$ & $-2m$ & $-1m$ &Kamera& $+1m$ & $+2m$ & $+3m$ \\ 
	\hline 
\end{tabular}
	\caption{Dargestellt ist der horizontale Winkelbereich in denen ein Gesicht mit aufbereitetem Inhalt erkannt wurde.\\
		Links: Einzelbilder, Rechts: Video}
	\label{graph_Test_1_Resize}
\end{figure}
\subsection{Ergebnisse}
Eine Videoanalyse ist also auch bei starker Kopfneigung nach unten möglich. Die Einzelbildauswertung liefert erneut etwas schlechtere Ergebnisse als die Videoauswertung.
Dabei funktioniert das Tracking nur, wenn die Versuchsperson zuerst einmal in die Kamera blickt, um einen guten Startpunkt zu bekommen. Aus diesem Test lässt sich ableiten, das auch dann eine Analyse erfolgen kann, wenn der Schüler sich dem Tisch vor sich zuwendet.\\
Auch die stärkere gleichmäßige Beleuchtung ist hilfreich, da sie Probleme durch Gegenlicht und Schatten reduziert.
\section{Versuch 3 - Berechnung auf der Augenpartie}
In diesem Abschnitt wird die Aufarbeitung der Augenregion im Kamerabild genauer untersucht. Hierzu wird eine deutlich bessere Kamera verwendet und der ElSe-Algorithmus zur Pupillenerkennung auf den Bildbereich angewendet. Es wurde eine hochauflösende Kamera gewählt, da Versuch 1 und 2 gezeigt haben, das die Augenbereich sehr klein dargestellt werden.
Mit der neuen Kamera sind hoffentlich genügend Bildinformationen enthalten sind für eine brauchbare Auswertung. Dieser Versuch soll auch die Nutzbarkeit von Eye-Tracking in solch einem Szenario testen. Von Interesse ist die Augenpartie und die Ergebnisse des OpenFace Eye-Detektor im Vergleich zu ElSe.
\begin{figure}
	\centering
	\begin {tikzpicture}
	\node[draw,align=center] (in) at(0,0) {Eingabebild\\\includegraphics[width=0.18\linewidth]{img_Versuch_Auge/Auge_in}};
	\node[draw,align=center] (El) at(6,0)  {Ellipse\\\includegraphics[width=0.18\linewidth]{img_Versuch_Auge/Auge_EL}};
	\node[draw,align=center] (OF) at(6,-4) {Landmark\\\includegraphics[width=0.18\linewidth]{img_Versuch_Auge/Auge_OF}};
	\node[draw,align=center] (ER) at(12,0)  {Ergebnis\\\includegraphics[width=0.18\linewidth]{img_Versuch_Auge/Auge_ER}};
		
	\draw[->] (in)to node[above]{ElSe}(El);
	\draw[->] (in)to node[left]{OpenFace}(OF);
	
	\draw[->] (El)to node[above]{Iris \& Pupille}(ER);
	\draw[->] (OF)to node[right]{Augenlider}(ER);	
\end{tikzpicture}
	\caption{Dargestellt sind der Ablauf um die Landmarks des Auges zu verbessern}
	\label{graph_Auge_Verbesserung}
\end{figure}
\subsection{Versuchsaufbau}
Als Messinstrument wurde eine Sony ILCE-6000 verwendet. Diese liefert ein $6000\times 4000$ Pixel großes Farbbild bei einer Brennweite von  $16mm$. Die Standorte der Probanden relativ zur Kamera wurden analog zu Versuch 1 gewählt. Da es sich um eine Fotokamera handelt, wurde ein Datensatz von Einzelbildern erstellt. Dabei wurden nur Aufnahmen gemacht, bei denen die Probanden direkt in die Kamera schauten.
\subsection{Auswertung}
Dabei wurde ElSe in der Basiskonfiguration eingesetzt, dies bedeutet das Luminance-Verfahren, siehe \autoref{gray_Luminance} als Graukonvertierer und einem Radius der Maske von 12 Pixel.
Für die Analyse wurde zuerst mit OpenFace die Augenpartie im Eingabebild bestimmt, siehe \autoref{graph_Auge_Verbesserung} und ein Beispiel in \autoref{img_Versuch_Auge} oben. Auf diesen Eingabebildern des Augenbereiches wird nun der ElSe-Algorithmus angewendet um die Pupillenellipse zu bestimmen.
Zum Vergleich sind 28 Landmarks der Augen die OpenFace liefert ebenfalls in \autoref{img_Versuch_Auge} dargestellt. Als Ergebnis wurde aus den berechneten Ellipsen von ElSe die Landmarks der Pupille und Iris abgeleitet und im selben Farbschema dargestellt.
\begin{figure}
	\centering
	\begin{tabular}{|c|c|c|c|c|c|c|c|c|c|c|} 
		\hline 
		\tabbild[width=0.07\linewidth]{img_Versuch_Auge/Auge_2}&
		\tabbild[width=0.07\linewidth]{img_Versuch_Auge/Auge_3}&
		\tabbild[width=0.07\linewidth]{img_Versuch_Auge/Auge_6}&
		\tabbild[width=0.07\linewidth]{img_Versuch_Auge/Auge_7}&
		\tabbild[width=0.07\linewidth]{img_Versuch_Auge/Auge_10}&
		\tabbild[width=0.07\linewidth]{img_Versuch_Auge/Auge_11}&	
		\tabbild[width=0.07\linewidth]{img_Versuch_Auge/Auge_14}&
		\tabbild[width=0.07\linewidth]{img_Versuch_Auge/Auge_15}&
		\tabbild[width=0.07\linewidth]{img_Versuch_Auge/Auge_17}&
		\tabbild[width=0.07\linewidth]{img_Versuch_Auge/Auge_19}&
		\tabbild[width=0.07\linewidth]{img_Versuch_Auge/Auge_22}\\
		\hline 
		$1m$&$2m$&$3m$&$4m$&$5m$&$6m$&$7m$&$8m$&$9m$&$10m$&$11m$\\ 
		\hline 
	\end{tabular}
	\caption{Beispielergebnisse von OpenFace und ElSe bei verschiedenen Distanz.\\ Von Oben nach Unten: Augenparie, Ergebnis OpenFace, Ergebnis ElSe, ElSe-Ergebnis als Landmarks}
	\label{img_Versuch_Auge}
\end{figure}
\subsection{Ergebnis}
Die Augenpartie der Probanden an den verschiedenen Positionen ist in \autoref{Augenbereich_Versuch3} dargestellt, die sich bei der angegebenen Distanz frontal vor der Kamera befand. Es ist zu erkennen, das selbst bei einer hohen Auflösung die Augenpartie sehr klein ausfällt und deshalb nur schwierig auszuwerten ist.\\
Somit zeigt sich, dass trotz einer hohen Bildauflösung der Informationsgehalt auf größere Distanzen deutlich abnimmt, wenn mit einer einzigen Kamera der gesamte Bereich einer Klasse erfasst werden soll. Außerdem ist auch gut zu erkennen, dass eine ausreichende Beleuchtung notwendig ist, da die Augenregion oft sehr dunkel ausfällt.