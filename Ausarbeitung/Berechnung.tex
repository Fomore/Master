\section{Berechnung der Position}
\label{calc_Position}
Zur Bestimmung der Position $P=(X_{avg};Y_{avg};Z_{avg})$ des Gesichtes im Kamerakoordinaten wird die Größe, ein Skalierungsfaktor $S$, des Kopfes im Bild verwendet.\\
Da bei der Abbildung von den Koordinaten ins Bild gilt $x=f\cdot \frac{X}{Z}$ und $ y=f\cdot \frac{Y}{Z}$, somit kann die Tiefe wie folgt abgeschätzt werden.\\
Sei $P_1 = (X_1;Y_1;Z_1), P_2=(X_2;Y_2;Z_2)$ die Beschreibung der Größe $G$ eines Kopfes mit:\\
\begin{align*}
a &= \frac{\sqrt{(X_1-X_2)^2+(Y_1+Y_2)^2}}{\frac{Z_1-Z_2}{2}} =\frac{G}{Z_{avg}}\\
S &= \frac{S_G}{G}\\
\Rightarrow a\cdot f &= f\cdot\frac{G}{Z_{avg}} = S_G\\
Z_{avg} &= \frac{f}{S_G}\cdot G = \frac{f}{S}\\
X_{avg} &= \frac{x \cdot Z_{avg}}{f}\\
Y_{avg} &= \frac{y \cdot Z_{avg}}{f}\\
\end{align*}
Dies beschreibt allerdings nur eine Annäherung an die Tatsächliche Position, da mit einem Durchschnittlichen Kopfgröße gerechnet wird.
\subsection{Zusammenhang Bildposition \& Weltposition}
Als Ausgangspunkt werden die Ergebnisse des CNN eingesetzt um mit deren Hilfe wie in \autoref{calc_Position} beschreiben die Position zu bestimmen. Zur Bestimmung der Orientierung $R$ liefert auch das CNN ein Ergebnis $R_{CNN}$. Allerdings stimmt es nur im Zentrum des Bildes, da am Rand immer mehr die Orientierung der einzelnen Pixel mit berücksichtigt werden muss.\\
\begin{align*}
euler_x &= \tan^{-1}(\frac{\sqrt{X^2+Z^2}}{Z^2})\\
euler_y &= \tan^{-1}(\frac{\sqrt{Y^2+Z^2}}{Z^2})\\
R_{pos} &= R(euler_x,euler_y,0)\\
R &= R_{CNN}\cdot R{pos}
\end{align*}
Eine weitere Verbesserung kann erreicht werden, indem die gefunden 2D-Landmarks mit Hilfe des PDM in 3D zu überführen. Um anschließend die Überführung von 2D nach 3D-Koordinaten erneut zu bestimmen um die Orientierung und Position zu ermitteln. Auch bei diesem Verfahren muss die Pixelorientierung beachtete werden.