\section{Umwandlung von Farbbild nach Graubild}
Da die Berechnungen von ElSe auf Graubildern arbeitet und das Eingabebild in Farbe ist, muss es in ein Graubild umgewandelt werden.\\
Die Problematik bei der Wahl des Verfahrens liegt in der Anforderung, vor allem soll der Farbunterschied zwischen Pupille und der Umgebung maximal sein, die Pupille möglichst dunkel und das restliche Auge hell. Die Farbe der Iris erschwert die Differenzierung, wenn sie recht dunkel ist, der Grauwert zur Pupille entsprechend gering ausfällt. Andererseits, ist das Erkennen der Pupille bei sehr kleinen Bildern schwierig bis unmöglich wodurch auf der Iris gerechnet werden muss, und daher diese weiterhin erhalten bleiben sollte.\\
Nach der Umwandlung wird für die Anwendung das Graubild noch normiert, damit Mindestens ein schwarzes und ein weißes Pixel vorhanden ist.
\subsection{Luminance-Verfahren}
\label{gray_Luminance}
Dies ist ein lineares Verfahren, das der menschlichen Farbwahrnehmung entspricht. Eine Gamma-Korrektur wird bei der Umwandlung nicht verwendet.\\
Somit entsteht ein natürlicher Farbverlauf, bei dem der Farbunterschied zwischen Pupille, Iris und Auge auf einem mittleren Niveau bleibt. Außerdem ist dieses Verfahren oft Standard bei der Umwandlung von Farbbild nach Graubilder, siehe \autoref{img_Luminance}.
\[G_{Luminance} = 0.299 \cdot R + 0.587 \cdot G + 0.114 \cdot B\]
\subsection{Gleam-Verfahren}
\label{gray_Gleam}
Bei dem Gleam Verfahren wird jede Farbe (Rot,Gelb und Grün) gleich stark bewertet allerdings wird jeder Farbwert mittels einer Gamma-Korrektur verbessert und das Bild wirkt heller als bei dem Luminance-Verfahren, siehe \autoref{img_Gleam}.\\
Durch die Gamma-Korrektur wird vor allem der helle Bereich weiter erhöht, somit wird der Farbunterschied zwischen Iris und Auge vermindert, wodurch die Pupille der einzige dunkle Bereich wird.\\
Allerdings wird auch dieser Farbwert erhöht und sollte die Pupille nicht schwarz sein, sie eher ins Graue überführt wird.\\
Dieses Verfahren wurde gewählt, da es im Vergleich zu den anderen Verfahren im Test von \glqq Color-to-Grayscale: Does the Method Matter in Image Recognition?\grqq \cite{rgb_to_Gray} am besten abgeschnitten hat.
\[G_{Gleam}=\dfrac{R^{\frac{1}{2.2}} + G^{\frac{1}{2.2}} + B^{\frac{1}{2.2}}}{3}\]
\subsection{Gleam-New-Verfahren}
\label{gray_New}
Dies ist eine Variante von Gleam bei dem zuerst das gesamte Bild analysiert wird um die Parameter für die jeweilige Gamma-Korrektur zu ermitteln. Dies ist etwas aufwendiger, allerdings für die kleinen Bereiche hinnehmbar.\\
Durch die individuelle Veränderung der Farbkanäle, werden Farbunterschiede minimiert und somit alle stark farbigen Bereiche ebenfalls dunkel dargestellt. Der Kontrast zwischen der farbigen Iris und dem weißen Auge wird verbessert, siehe \autoref{img_NewGeam}.\\
Da allerdings alle Farben dunkel werden, entstehen weitere dunkle Bereiche die die Detektion der Pupille beeinträchtigen können.
\[G_{Gleam New}=\dfrac{R^{r} + G^{g} + B^{b}}{3}\]
Wobei gilt $\{r,g,b\} = \frac{\log(V_{\max})}{\log(\{R,G,B\}_{\max})}$ mit $V_{\max}$ als maximal möglicher Farbwert und $R_{\max}$ als maximal Vorhandener Rot-Farbwert, $G_{\max}$ und $B_{\max}$ äquivalent.
\subsection{Quadrat-Verfahren}
\label{gray_Quadrat}
Dies ist ein Verfahren, dass das Eingabebild verdunkelt und vom Aufbau dem Inversen von Gleam entspricht. Somit ist das gesamte Bild dunkler als bei dem Luminance-Verfahren, siehe \autoref{img_Quadrat}.
Durch die Abdunklung werden kleine Farbänderungen in den dunklen Bereichen reduziert, wodurch die Pupille sehr dunkel zu sehen sein sollte, der Farbunterschied zur Iris wird allerdings ebenfalls verringert.
\[G_{Quadrat}=\dfrac{R^2+G^2+B^2}{3}\]
\subsection{Min-Max-Verfahren}
\label{gray_MinMax}
Dabei handelt es sich eigentlich um zwei verschiedene Varianten, allerdings funktionieren beide nach dem selben Prinzip, als Grauwert wird der jeweilige Extremwert aus den einzelnen Farbkanälen gewählt.\\
Durch Verwendung der Extremwerte, wird das gesamte Bild deutlich heller bzw. dunkler und kleinere Farbänderungen werden entfernt.\\
Bei dem Max-Verfahren werden alle farbigen und helle Bereiche hell bleiben und nur gleichmäßig dunkel Bereiche bleiben dunkel wie es bei schwarz der Fall ist. Wenn der Minimalwert anstelle verwendet wird, bleiben nur gleichmäßig helle Bereiche hell, alles andre wird abgedunkelt.
\begin{align*}
G_{max} &= \max(R,G,B)\\
G_{min} &= \min(R,G,B)
\end{align*}
\subsection{Normalisierung von Graubilder}
Um ein Graubild zu erhalten, dass das volle Spektrum der möglichen Werte erfüllt, wird das Eingabebild normalisiert. Dazu wird der Maximale $G_{max}$ und Minimale $G_{min}$ im Bild gesucht um anschließend wird der neue Grau-Wert $G_{new}$ wie folgt bestimmt, dabei ist $V_{max}$ der maximal größte Wert.
\[G_{new} = G\cdot \dfrac{V_{max}+G_{min}}{G_{max}}-G_{min}\]
Da für die Anwendung ein Schwarzer Bereich gesucht wird gegen einen Hellen Hintergrund, wird für die Bestimmung der Extremwerte nicht das originale Eingenbild verwendet, sonder ein Gauß-gefiltertes.\\
Dies hat den Vorteil, das einzelne lokal auftretende Werte nicht als Extremwert verwendet werden. Dies hat zur Folge, dass die Pupille gleichmäßiger dunkler wird und Pixel die eine Reflektion darstellen ignoriert werden, wodurch das gesamte Bild stärker aufgehellt wird.
\subsubsection{Auswirkung des Gauß-Filters}
Dies ist ein Tiefpassfilter und wird verwendet um das Eingangssignal zu glätten. Dies hat in der Bildverarbeitung den Effekt, dass Details im Bild verschwimmen und das Bild unscharf wird.
\begin{figure}
	\centering
	\includegraphics[width=0.2\linewidth]{img/Farbtafel2}
	\includegraphics[width=0.2\linewidth]{img/lena}
	\includegraphics[width=0.2\linewidth]{img/Auge}
	\caption{Dies sind die Eingabebilder der verschiedenen Konverter von Farbe nach Grau. Links eine Farbpalette, Mitte Lena und Rechts ein Augenausschnitt aus dem Augendatensatz \cite{database_Eye}}
	\label{img_Gray_Einagbe}
\end{figure}
\begin{figure}
	\centering
	\includegraphics[width=0.2\linewidth]{img/Farbkarte_Normal}
	\includegraphics[width=0.2\linewidth]{img/Lena_Normal}
	\includegraphics[width=0.2\linewidth]{img/Auge_NormGray}
	\caption{Ergebnis der Umwandlung von Farb- nach Grauwert mittels Luminance-Verfahren}
	\label{img_Luminance}
\end{figure}
\begin{figure}
	\centering
	\includegraphics[width=0.2\linewidth]{img/Farbkarte_22}
	\includegraphics[width=0.2\linewidth]{img/Lena_22}
	\includegraphics[width=0.2\linewidth]{img/Auge_22Gray}
	\caption{Ergebnis der Umwandlung von Farb- nach Grauwert mittels Glean-Verfahren}
	\label{img_Gleam}
\end{figure}
\begin{figure}
	\centering
	\includegraphics[width=0.2\linewidth]{img/Farbkarte_New}
	\includegraphics[width=0.2\linewidth]{img/Lena_New}
	\includegraphics[width=0.2\linewidth]{img/Auge_NewGray}
	\caption{Ergebnis der Umwandlung von Farb- nach Grauwert mittels Gleam-New-Verfahren}
	\label{img_NewGeam}
\end{figure}
\begin{figure}
	\centering
	\includegraphics[width=0.2\linewidth]{img/Farbkarte_Quadrat}
	\includegraphics[width=0.2\linewidth]{img/Lena_Quadrat}
	\includegraphics[width=0.2\linewidth]{img/Auge_QuadratGray}
	\caption{Ergebnis der Umwandlung von Farb- nach Grauwert mittels Quadrat-Verfahren}
	\label{img_Quadrat}
\end{figure}
\begin{figure}
	\centering
	\includegraphics[width=0.2\linewidth]{img/Farbkarte_Max}
	\includegraphics[width=0.2\linewidth]{img/Lena_Max}
	\includegraphics[width=0.2\linewidth]{img/Auge_MaxGray}\\
	\includegraphics[width=0.2\linewidth]{img/Farbkarte_Min}
	\includegraphics[width=0.2\linewidth]{img/Lena_Min}
	\includegraphics[width=0.2\linewidth]{img/Auge_MinGray}
	\caption{Ergebnis der Umwandlung von Farb- nach Grauwert mittels Extremwert-Verfahren. Oben: Max-Verfahren, Unten: Min-Verfahren}
	\label{img_MinMax}
\end{figure}
